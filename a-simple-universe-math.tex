\documentclass[a4paper
,draft
]{article}

\usepackage{amsmath}
\usepackage{amsthm}
\usepackage{appendix}
\usepackage[english]{babel}

 %%%%%%%%%%%%%%%%%%v
\usepackage{combelow}
\usepackage{hyperref}
\usepackage[utf8]{inputenc}
\usepackage{newunicodechar}

\usepackage[
    backend=biber,
    style=apa
%    citestyle=authoryear,
%    citestyle=alphabetic,
%    sortcites=true,
%    style=authoryear
%    style=alphabetic
    ]{biblatex}

\DeclareLanguageMapping{english}{english-apa}
\addbibresource{design-argument.bib}

\newunicodechar{Ș}{\cb{S}}
\newunicodechar{ș}{\cb{s}}
\newunicodechar{Ț}{\cb{T}}
\newunicodechar{ț}{\cb{t}}

\title{A Simple Universe Argument}
\author{Virgil Șerbănuță\thanks{\href{mailto:design-and-chance@poarta.org}{design-and-chance@poarta.org}}}
%\date{June 2015}
 %%%%%%%%%%%%%%%%%%^


\usepackage{amsfonts}
\usepackage[obeyDraft]{todonotes}
\newcommand{\svn}[2][]{\todo[author=Virgil,color=red!25!white,#1]{#2}}
\newcommand{\tsf}[2][]{\todo[author=Traian,color=green!40!white,#1]{#2}}
\newcommand{\tsfgata}[2][]{\todo[author=Traian,color=blue!40!white,#1]{DONE - #2}}
\newcommand{\commentfootnote}[1][]{}

\def\infordinala{\omega}
\def\infordinalb{\omega_1}
\def\reale{\mathbb{R}}
\def\intregi{\mathbb{Z}}
\def\complexe{\mathbb{C}}
\def\naturale{\mathbb{N}}
\def\rationale{\mathbb{Q}}
\def\descriptions{\mathbb{K}}
\def\designer{\mathbb{D}}
\newcommand{\paper}[1]{paper}
\newcommand{\multime}[1]{\left\{ #1 \right\}}
\newcommand{\definitie}[1]{\textbf{#1}}
\newcommand{\ghilimele}[1]{``#1"}
\newcommand{\lnotat}[1]{\sim #1}
\newcommand{\citare}[1]{(\cite{#1})} % TODO: Separate this in a version with parenthesis and one without.

\newtheorem{definition}{Definition}
\newtheorem{afirmatie}{Claim}
\newtheorem{notatie}{Notation}
%\newtheorem{theorem}{Theorem}[section]
\newtheorem{lemma}{Lemma}
\newtheorem{axiom}{Axiom}
\newtheorem{note}{Note}

\begin{document}

 %%%%%%%%%%%%%%%%%%
\maketitle
 %%%%%%%%%%%%%%%%%%

 \begin{abstract}
  This \paper{} presents an argument that one can make a prediction from
  the hypothesis that our universe is not designed, i.e. that it has a high
  level of complexity of a certain kind, and that this complexity would be
  easily observable everywhere. However, this is not what we observe, which
  falsifies the hypothesis. \svn{Make sure I actually made a prediction.}
\end{abstract}

\section{Introduction}

Many people believe
that we can make various claims about the universe,
that there are certain laws approximating its behaviour fairly well,
that we can compute its age,
that we can make predictions about the distant future, and so on.
This \paper{} started as a
mathematical attempt to understand why this is the case
and what we can reasonably believe about our universe.
It turns out that it matters
whether our universe is designed or not, so it became a \paper{} about that.

In the end, the mathematical part turned out to be rather small,
containing only a few simple properties about set cardinalities,
probabilities and ordinals.
I think that the non-mathematical ideas in this paper
have an intuitive appeal even without the mathematical ones,
but presenting them separately would make it less clear why certain
conclusions can be drawn.
Although, in my opinion, the non-mathematical ideas are also somewhat obvious,
I did not manage yet to find anyone drawing the same conclusions
in the same way.

My argument can be considered a statistical approach to Aquina's fifth way.
It tries to avoid the issues that other statistical approaches (e.g. the
fine-tuning argument) have.

Sections \ref{?} to \ref{?} present the mathematical notions and results used
in this paper. Section \ref{?} introduces universe approximations. Section
\ref{?} introduces the argument's axioms, while section \ref{?} uses them
to make predictions about our universe, then verifies if they match our
observations. Sections \ref{?} and \ref{?} present this paper's relations to
previos work.

\section{The ordered universe argument}

%TODO: Urmatoarele trei alineate sunt nasoale.
The \svn{great?} Catholic theologian Thomas Aquinas,
in his \ghilimele{fifth way}, attempts to show God's existence from
the order of the universe, i.e. that almost all bodies, almost always
behave according to simple natural laws. %TODO: citation.
One can find a good exposition of this argument in
\citetitle{swinburne1968}%\citeauthor{swinburne1968} \citeyear{swinburne1968}
\parencite{swinburne1968}, but let us look at a few ideas which are
interesting in the context of this \paper{}.
%Metaphysics by Peter van Inwagen, from 2015, says that it's a powerful and sophisticated defense.
% 67 de citari https://scholar.google.ro/scholar?cites=18062876017816040453&as_sdt=2005&sciodt=1,5&hl=en

There are two types of order which are interesting, the spatial order
and the temporal order.
The former is the order that one can see in (a part of) the universe
at a given moment in time, e.g. that planets, living bodies, and other things
are ordered.
\svn{This seems a bit awkward, is there a better way to state it?}{} It is
tricky to build an argument for the existence of a designer based on it.
The latter can be seen in the behaviour of things, including the laws of
nature, and it's the one that will be used in this \paper{}.

\citetitle{swinburne1968} addresses Hume's
objections\svn{Need to introduce them. * Re-read The Argument From Design}
against Aquinas' argument, but one in particular is interesting here:
the argument is based on analogy, which limits its strength.
I think that this \paper{}'s argument, although it builds on the same
foundation, needs only an extremely weak analogy
(see Section \ref{sec:remaininghypothesis})\svn{Fix section}.
% Do I really need the analogy? If yes, remove the \ghilimele{may work}. I don't need it - see the testimony argument.

The following quote from \textcite{swinburne1968}, made when addressing
Hume's objection that the order which can be observed in this universe
is just an accident, makes a nice introduction for the argument
described in this \paper{}:
\begin{quote}
But if we say that it is chance that in 1960 matter is behaving in a
regular way, our claim becomes less and less plausible as we find that in
1961 and 1962 and so on it continues to behave in a regular way. An appeal
to chance to account for order becomes less and less plausible
the greater the order.
\end{quote}

However, there were many people who were surprised by the existence of laws
for our universe in one way or another and who didn't try to use it
as an argument for the existence of a Designer.
As an example, see this quote of \textcite{Feynman2009}:
\begin{quote}
Incidentally, the fact that there are rules at all to be checked
is a kind of a miracle; that it is possible to find a rule,
like the inverse square law of gravitation, is some sort of miracle.
It is not understood at all, but it leads to the possibility of
prediction --- that means it tells you what you would expect in
an experiment you have not yet done.
\end{quote}

% TODO: Maybe list some of Hume's issues with this argument.

\section{The fine-tuning argument and other objections}

Although the argument in this paper does not resemble the fine-tuning argument
much, it uses similar methods, which may raise similar objections. Indeed,
when writing this paper I had to carefully avoid these objections, so readers
may find it useful to have a short summary of the fine-tuning argument and its
objections which are also relevant to this paper, together with an explanation
on how they were avoided.

In the recent past people started noticing that life can exist only for
narrow ranges of various parameters of our universe, so our universe
seems fine-tuned for life.
But that seems unlikely unless there is a designer for our universe,
a designer that wanted it to have (intelligent) life.
Examples of such parameters include the early cosmic density and
the expansion speed, the strength of the weak nuclear force, and
the ratio between electromagnetism and gravity \parencite{Leslie2003}.

Although this may seem intuitively reasonable, people have found various
potential problems with this argument, some of which are presented below.
For possible replies to some of these in the fine-tuning context see,
e.g., \textcites{Leslie2003}{Swinburne2003}{Monton2006}{Kotzen2012}.
% TODO: Cite \ghilimele{Should we care about fine-tuning} by Jeffery Koperski.
% TODO: Actually check if these issues apply to my argument.

Probabilities do not apply to the values of the cosmic parameters:
if one assumes that all cosmic parameter ranges have the same probability
then the probabilities become non-normalizable and the argument
would apply to any finite range (coarse-tuning).
If the cosmic parameter ranges do not all have the same probability,
then we must consider many possible probability distributions.
For some of them the life-permitting ranges have low probability,
for others they have high probability and we don't have a good way
of choosing one of them \parencites{Manson2009}{McGrew2001}.

Indeed, the most reasonable assumption is that they all have the same
probability, i.e. $0$, which implies that, if our mainstream physics theories
provide a reasonable approximate model for our universe, then the probability
that our universe is not designed is $0$.

The current argument also analyses the other case, having unequal probabilities
for individual events, where, in order to believe that our universe is not
created, one needs to either ignore the evidence that we have about our
universe's consistency, or believe in a rather weird and implausible structure
of meta-universes, without having any proof for its existence.

There is an observation selection effect:
we can't observe a universe which is not life-permitting,
so our observations are biased.
If we take this additional piece of information into account,
the probability of our universe not being designed is not longer
small \parencite{Sober2009}.
This is particularly interesting if multiple universes
exist \parencite{Manson2009}.

All universes considered in this paper contain intelligent life. Even
more, the next source of bias, those beings' perception, was taken into account
by describing universes as they are seen by the said beings.

There is a third possible source of bias which is considered here: it can be
argued that, in a non-created universe, beings might be intelligent
only if their intelligence is useful to them. But this likely means that those
beings live in a timespace region which seems consistent from their point of
view. This is also addressed in the argument: we observe non-essential
consistency, which can only be explained by design.

The multiple universes objection raises an interesting issue in the context
of the current argument.
When multiple universes exists, the probability of an intelligent being
living in a universe having a finite observable description is still $0$, or
it needs the same implausible assumptions as the case of a single universe.
In virtually all universes this argument\svn{axiom ??? instead of
\ghilimele{this argument}} would not work, since the infinite complexity
of the universe would be easily observable.
For the intelligent beings living in the other universes, as long as they have
a hypothesis which is more plausible, it would be irrational to think that
their universe is not designed (assuming that the argument presented in
this \paper{} is correct).

Non homogenous spacetime:
another possible case similar to the above is when our universe
is much larger than the part that we can observe and, outside
of the part that we can observe now, it contains spacetime regions
with different values for the various cosmic parameters \parencite{Manson2003}.

The current argument takes into account the fact that different
spacetime regions can have different descriptions. Indeed, the fact that this
is logically possible is used to show that the timespace region of our universe
in which we live is implausible in a non-created universe.

Unknown Designer's intent:
from a natural theology point of view, one can't
know what the Designer wanted
(e.g. one can't know that a universe designer would want to create
a universe having life) \parencites{Sober2009}{Narveson2003}, so, by default,
any argument showing that the probability of our universe is small
if it's not designed would also show that the probability is small
even if it is designed.
To fix this, one would need an independent way to show that the Designer
wanted the universe to have life \parencite{Sober2003}.

Although this argument is still based on analogy, note that the weakest
analogy that still has some trace of plausibility is enough, i.e. we just need
that \ghilimele{a Designer of world wants intelligent beings} has non-zero
probability. \svn{Rephrase this - the counterargument has nothing about
analogies. Maybe add something about revelations.}

Design criteria: minds may be the product of natural laws,
but we intuitively distinguish something natural from something
designed by checking if their production involves a mind or it is only
a result of natural laws.
There is no similar way of telling if the creation of the laws of nature
themselves is natural or not \parencite{Narveson2003}.
\svn{Think carefully about this. This is not a FT counter-argument and
I should make sure that I'm understanding this correctly. Think carefully.}{}

I will rephrase this as: we distinguish designed object by being much
more likely to be produced by processes involving minds than by natural laws.
As an example, with our current understanding of physical laws, a watch could
materialize out of thin air, or it could happen that its various pieces are
the product of chance, and they came together by chance,
but that's way too implausible, while it is more likely that any watch that we
observe was designed.

In the same way, does not show that the non-design hypothesis for our universe
is logically impossible, it just shows that, under any plausible
\ghilimele{natural laws}, our universe is much less plausible than if it was
designed.

\svn{Should this be here?} Note that, under the design assumption, one may have
information from outside the universe about the designer's intent. \svn{A good
question is how can we tell that from information from inside the universe.}

\svn{10} In order to avoid this objection, the current argument works with
conceivable universes.

First, let us note that using conceivable universes makes sense. If we are
trying to make a predictiona about how a designed / non-designed universe would
look like, we would have to imagine that we are outside any universe, and we are
about to observe one that is (say) non-designed. What could we say about it?
Not much, maybe that it makes logical sense (i.e. that it can be modelled
mathematically), maybe it has something cloase to time and space, but
everything else is possible.

As a parenthesis, one could go one step further and, by using the same proof
structure as used in this paper, show that, if there is no designer of worlds,
then the hypothesis that only our universe exists is a peculiar one, so it is
extremely unlikely.

There is a second reason why we should believe that many different universes
exist: \svn{11} \svn{12} \svn{13} - Do I need this? Does it help?

\textcite[][Section \ghilimele{Why a world with human bodies is unlikely
if there is no God}]{Swinburne2003} comes sort of close to the argument
presented in this \paper{}, but while \citeauthor{Swinburne2003}
argues that human bodies are unlikely, I am arguing that, in the context
of all possible universes that could have human-like beings,
our universe is extremely unlikely.
Since the existence of human bodies is not directly related to the subject
of this \paper{}, I will not discuss that section more
than it is strictly needed.
Note that  \citeauthor{Swinburne2003} says that individual sets of laws
have non-zero probability while I'm claiming that their probability is $0$.
It seems to me that \citeauthor{Swinburne2003} implicitly assumes that
such a set has a finite number of laws, while I am explicitly removing
that constraint, so both can be right within their contexts.

\section{Possible universes and their descriptions}

% TODO: maybe delete the footnote.
Many people believe that our universe is designed and created
and that it's unreasonable to believe that any universe can exist
without being created, and I agree with them.
However, these beliefs are not shared by everyone, so it's worth thinking
about what creation means.
If our universe is created, then it's likely to be the way it is because
its Creator
\footnote{Not everybody that believes that our universe is created
thinks that God created it.
Still, I hope that they would agree that capitalizing the Creator
of this universe is reasonable.}
\svn{Do I need this footnote?}{} wanted it to have certain properties.
In order to understand why our universe works the way it does,
one would need to understand the intent of its Creator.
While that is interesting in itself, I will not try to pursue it here,
except for a few limited ideas.
% TODO: use here the \ghilimele{design implies non-continous probabilities idea}

For most of the remainder of this \paper{}, let us consider the other case.
Let us assume the hypothesis that our universe is not designed and created
and let us try to make a prediction based on it.
How would a non-created universe look like?
Would it be similar to our universe?
Maybe an infinity of universes exist and ours is just one of many,
or maybe our universe it the only one that exists.
Even if ours is the only one, one could easily imagine that it worked
in a different way, say that the speed of light would be different,
or that gravity would work differently.

There are people who claim that all logically possible universes exist,
either because they think that it simply makes sense, or because they want to
give a good account of modality, or because they need
an ad hoc\svn{Do I need the \ghilimele{ad-hoc} part?}{} hypothesis
to argue against, say, fine-tuning.
The argument in this \paper{} does not assume that more than
one universe exists, but it should work either way.

A possible universe
could have exactly the same fundamental laws as ours, but with matter
organized differently.
It could have similar laws, but with different universal constants.
It could have different fundamental particles (or whatever the basic
building blocks of our universe are, assuming that there are any).
Or it could be completely different, i.e. different in all possible ways.

It could be that our logic and reasoning are universal instruments,
but it could also be that some of these possible universes are
beyond what our reasoning can grasp and others have properties
for which our logic is flawed.
Even if that's the case, let us see if we can say anything about
the possible universes that we could understand and could model in some way.

In the following, the \definitie{possible universes} term will denote
only the possible universes which we could model.

This notion of model is not precise enough.
Let us restrict the possible universes term even more,
to the possible universes that could be modelled mathematically,
even if that may leave out some of them.
This may seem too restrictive,
especially since this paper only needs universes which can be approximated
by mathematical models.
However, for simplicity, let us, for now, consider only universes
which are fully modellable with sets of axioms that are at most countable.

Let us restrict the universes we are considering even further, to universes that
have something that remotely resembles time and space, for which
\ghilimele{the state of the universe at a given moment in time}, or something
close, makes sense, and which can plausibly contain intelligent beings.
Any such universe is, for the purpose of this paper, a conceivable universe.

To keep the exposition simple, in the following I will use
\ghilimele{state of the universe at a given moment in time},
but one should replace it with one's favourite alternative concept.

A \definitie{universe description} is a mathematical theory that has
a set of axioms which is at most countable and which allows making
predictions about the future state of the universe given its state
at a certain moment in time. A \definitie{universe region description}
is something similar, but only for a given space-time region of a universe.
In the best case, for a deterministic universe, there might exist
a description which allows one to correctly predict the entire future state
given the state at any moment in time, but a universe description
as defined here does not have to predict everything and,
even when it predicts something, it does not have to always be correct.

Note that the data available for making predictions is dependent
on who is making the prediction. As an example, if we assume that
all predictions are about things that can be perceived, then
each kind of intelligent beings (e.g. humans) will make predictions
about the universe projected through their senses. If a universe contains
multiple kinds of intelligent beings, with different kinds of
sense organs, then that universe may have descriptions which are
very different.

In order to handle this dependence on who observes the universe
in a reasonable way, in the reminder of this paper we will work with universes
that contain intelligent beings
and all predictions will be relative to what these intelligent beings
could observe. If there are multiple kinds of intelligent beings in the
universe, we'll assume that we picked one such kind. I.e. although we
will talk about universes and their descriptions, we'll actually mean
(universe, intelligent-being-kind) pairs and their corresponding descriptions.

Next, let us try to specify how good an univese description
is. First, our predictions speak about the future, but expecting them to
predict everything until the end of the universe (if any) may not be
reasonable. We may want to fix a minimum amount of time $\Delta t$
and to evaluate the predictions about things that are at most
$\Delta t$ in the future. Second, we should't expect to
be able to describe everything with full precision, so we may want to
have a precision $\eta>0$ for all the values that we predict.
Third, we shouldn't expect our predictions to always be correct, so
we should require that they are true with probability $p>0$.
The exact meaning of \ghilimele{true with probability} here is left open.

Then let us say that an \definitie{approximate universe description} with a
\definitie{level of approximation} ($\eta>0$, $p>0$ and $\Delta t>0$)
is a universe description which allows one to approximate the future
state of the universe with a precision $\eta$, with a probability
$p>0$ for a prediction to be correct
\svn{Do I need the probability of the prediction to be correct
     or do I need to cover a fraction p of the world?}
and for a limited amount of time $\Delta t$.
\svn{I must answer the following questions: exactly what does that
     probability of correct prediction mean? Is it an aggregate among
     all possible predictions? Does it mean that all of them need a p
     chance of being correct? Do I need a fraction of the universe here?
     All writable predictions are countable, so they all have a non-zero
     weighth, so the aggregate probability most likely can't be zero.
     Am I talking about provable predictions or about predictions
     which are true on models?}

\svn{200 reformulat}

For a given universe or region of a universe,
given a level of approximation, we will pick a canonical description
in the following way: Let $S$ be the set of descriptions which approximate
the universe with the given level of aproximation. If $S$ contains
at least one finite description, then we pick the shortest such
description as \ghilimele{the canonical description}, breaking ties by
using the lexicographic order. Otherwise, we simply say that the
universe (region) has an infinite description, and we will abuse the
terminology a bit by picking $S$ as the canonical description (we could
also pick a random description from the set).
If the level of approximation is obvious from the context, we will call
this canonical description \definitie{the universe's description}
or the \definitie{universe region's description}.

One could use a well-ordering on the real numbers to choose the
lowest description as the universe's description, but that will
complicate things without any benefit.

\section{Options for our universe}

The reminder of this paper will analyze what we can reasonably believe about
the following issues:
\begin{itemize}
  \item Our universe is created or not.
  \item Our universe has finite or infinite space.
  \item Our universe has finite or infinite time.
  \item Our universe has a finite or infinite description.
  \item Option 1: There is a meta-$\beta$ universe for each countable ordinal
        $\beta$ such that our universe is the meta-$0$ one and the meta-$beta$
        universe includes, directly or not, all meta-$\alpha$ universes with
        $\alpha < \beta$. Option 2: there is an ordinal $\beta$ where this
        stops being the case.
  \item Life is or is not compatible with a finite chunk of space-time having
        an infinite observable description.
\end{itemize}

\section{Axioms}

\subsection{Observing events}

\begin{axiom}
  If $P$ is a probability over the set of real numbers
  (or a set with the
  same cardinal) which can measure singletons,
  we observe $x\in\reale$, and $P(x)=0$, then $x$ is generic.
\end{axiom}

Note that this does not require $P$ to be a probability over the
Borel algebra of $\reale$, although that is, in many cases, implicitly
assumed when talking about probabilities over $\reale$.

The set of events for which $P(y)$ is
greater than $0$ is at most countable, and, if we remove them from $\reale$, we
get a set with the same cardinality as $\reale$. On this set, the probability
of all peculiar events taken together is $0$, so there is no chance of us
observing one. In other words, the probability of all generic events is $1$,
so we can be sure that we observed a generic event.

Of course, the (logical) possibility of observing a peculiar event still exists,
but, practically, we will not observe it as long as the set of our
observations is countable.

This axiom could be generalized to probabilities for which singletons are not
measurable, in those cases where we have a set of disjoint measurable events
with the same cardinality as $\reale$, but, for the purpose of this paper,
the current version should be enough.

\begin{axiom}\label{noprobability}
  If we observe $x\in\reale$, when we could have
  observed any real number, and there is no probability distribution that could
  describe how $x$ was chosen, then $x$ is generic.
\end{axiom}

Note that this axiom does not say that we do not know that probability
distribution, it says that there is no such probability distribution.
Anyone believing that this cannot happen should treat the cases where
this axiom applies as invalid.

Also note that this cannot happen when using subjective probabilities.

If there is nothing that could favor
peculiar numbers over generic ones, it's absurd to think that we could have
observed an element of such a tiny set among something infinitely larger.
Also, the case with a probability above suggests that this may be the only
reasonable assumption in this case.

\subsection{Design probability}

\svn{30} \svn{31} \svn{32} \svn{33} \svn{34}
\svn{35} \svn{36} \svn{37} \svn{38} \svn{39}
\svn{40} \svn{41} \svn{42} \svn{43} \svn{44}
\svn{202}
\svn{45} \svn{46} \svn{47}

\svn{48}

\subsection{$\reale^4$ universe}

\svn{49} \svn{50} \svn{51} \svn{52} \svn{53} \svn{54}

\subsection{Logically possible universes}

\svn{55} \svn{56} \svn{57} \svn{58} \svn{59} \svn{60}

For any approximtion level $L$, and any region that is not trivial
for $L$ (left open), there are multiple possible descriptions that
are distinct for $L$ (not defined!). By splitting an infinite timespace
into ?hypercube/hyperparalelipipedic? regions with corners in $\rationale$
and taking all possible region description combinations,
we get a set of countable universe descriptions \svn{61*}

\subsection{Neighborhood modelling}

\svn{62}
\svn{201}
\svn{
    Think about deterministic vs non-deterministic modelling.
    Think about overlapping approximate descriptions - I can't have
    equivalence classes because a description that works for
    a fraction of the universe can be equivalent to two incompatible
    descriptions for the other fraction. Does the explanation in
    \ghilimele{Logically possible universes} still work?
}

\section{Valid options for our universe}

\svn{63} \svn{64}

\subsection{Finite Descriptions and Uncountable Meta-universes}

\svn{65} \svn{66} \svn{67} \svn{68} \svn{69}
\svn{70} \svn{71} \svn{72} \svn{73} \svn{74}
\svn{75} \svn{76} \svn{77} \svn{78} \svn{79}
\svn{80} \svn{81} \svn{82}

\subsection{Peculiar description for universe regions}

\svn{83} \svn{84}
\svn{85} \svn{86} \svn{87} \svn{88} \svn{89}
\svn{90} \svn{91}

\subsubsection{Improbable consistency}

\svn{92} \svn{93} \svn{94}
\svn{95} \svn{96} \svn{97} \svn{98}
\svn{Footnote 100}
\svn{Footnote 101}

\section{Conclusion}

\svn{*.*}
\svn{
    + explicatie ca universul nostru e unul din multele posibile bazate pe QM
    means that the meta-universe is too simple.
}

\end{document}
