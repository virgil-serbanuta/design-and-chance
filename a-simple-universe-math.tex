\documentclass[a4paper
%,draft
]{article}

\usepackage{amsmath}
\usepackage{amsthm}
\usepackage{appendix}
\usepackage[english]{babel}

 %%%%%%%%%%%%%%%%%%v
\usepackage{combelow}
\usepackage{hyperref}
\usepackage[utf8]{inputenc}
\usepackage{newunicodechar}

\usepackage[
    backend=biber,
    style=apa
%    citestyle=authoryear,
%    citestyle=alphabetic,
%    sortcites=true,
%    style=authoryear
%    style=alphabetic
    ]{biblatex}

\DeclareLanguageMapping{english}{english-apa}
\addbibresource{design-argument.bib}

\newunicodechar{Ș}{\cb{S}}
\newunicodechar{ș}{\cb{s}}
\newunicodechar{Ț}{\cb{T}}
\newunicodechar{ț}{\cb{t}}

\title{A Simple Universe Argument}
\author{Virgil Șerbănuță\thanks{\href{mailto:design-and-chance@poarta.org}{design-and-chance@poarta.org}}}
%\date{June 2015}
 %%%%%%%%%%%%%%%%%%^


\usepackage{amsfonts}
\usepackage[obeyDraft]{todonotes}
\newcommand{\svn}[2][]{\todo[author=Virgil,color=red!25!white,#1]{#2}}
\newcommand{\tsf}[2][]{\todo[author=Traian,color=green!40!white,#1]{#2}}
\newcommand{\tsfgata}[2][]{\todo[author=Traian,color=blue!40!white,#1]{DONE - #2}}
\newcommand{\commentfootnote}[1][]{}

\def\infordinala{\omega}
\def\infordinalb{\omega_1}
\def\reale{\mathbb{R}}
\def\intregi{\mathbb{Z}}
\def\complexe{\mathbb{C}}
\def\naturale{\mathbb{N}}
\def\rationale{\mathbb{Q}}
\def\descriptions{D_L}
\def\designer{\mathbb{D}}
\def\our_description{OURD}
\newcommand{\paper}[1]{paper}
\newcommand{\multime}[1]{\left\{ #1 \right\}}
\newcommand{\definitie}[1]{\textbf{#1}}
\newcommand{\ghilimele}[1]{``#1"}
\newcommand{\lnotat}[1]{\sim #1}
\newcommand{\negat}[1]{\sim #1}  % TODO: This is the same as \lnotat
\newcommand{\citare}[1]{(\cite{#1})} % TODO: Separate this in a version with parenthesis and one without.

\newtheorem{definition}{Definition}
\newtheorem{afirmatie}{Claim}
\newtheorem{notatie}{Notation}
%\newtheorem{theorem}{Theorem}[section]
\newtheorem{lemma}{Lemma}
\newtheorem{axiom}{Axiom}
\newtheorem{note}{Note}

\begin{document}

 %%%%%%%%%%%%%%%%%%
\maketitle
 %%%%%%%%%%%%%%%%%%

 \begin{abstract}
  This \paper{} argues that, if our universe is not designed, it should
  have a high level of a certain kind of complexity, which should be visible
  everywhere. Since we do not observe that, our universe is probably designed.
  Even more, it is probably designed for intelligent beings.
  \svn{This is valid regardless
  of how many universes exist, whether it is one, or all logically
  possbile ones, or anything in between.}
\end{abstract}

\section{Introduction}

There is a common belief
that there are certain laws approximating our universe's behaviour fairly well,
that we can compute its age,
that we can make predictions about the distant future, and so on.
In a way, all of these are really surprising.
In the words of \textcite{Feynman2009}:
\begin{quote}
Incidentally, the fact that there are rules at all to be checked
is a kind of a miracle; that it is possible to find a rule,
like the inverse square law of gravitation, is some sort of miracle.
It is not understood at all, but it leads to the possibility of
prediction --- that means it tells you what you would expect in
an experiment you have not yet done.
\end{quote}

This \paper{} attempts to figure out statistically
why do we observe such a surprising universe and
what can we reasonably believe about it.

This argument can be considered a statistical approach to Aquina's fifth way.
It tries to avoid some of the issues that other statistical approaches
attempting to show that our universe is designed
(e.g. the fine-tuning argument) have.

To be specific, it is arguing that, in the context of all possible universes
that could have human-like beings, our universe is extremely unlikely unless
a Designer intended it, and, even more, probably the
Designer wanted to design for human-like beings.

In order to show this, we first need to specify what would count as an universe.
This is a difficult question, but, fortunately, this paper's argument also works
if we only take into account only a subset of all posssible universes that
satisfies certain constraints.
In particular, these universes should be able to sustain intelligent beings,
and, assuming that they contain intelligent beings, there should be mathematical
theories (called \ghilimele{descriptions}) that these beings could use to predict
(either fully, or as an approximation) how their univers behaves.
We allow both deterministic and non-deterministic universes.

Section \ref{sec:possible-universes} introduces these notions in a more
rigourous way, together with some of reasoning behind them.

Just defining these notions is not enough,
we also need to make some assumptions about our universe and about
the link between reality and mathematics.

First, we specify what it means to observe events with probability $0$ From
a set large enough, generalizing this to the case when there is no probability
modelling how we observe the event.
Next, we specify that our universe can be modelled by a finite-dimensional
space based on the real numbers, or by something close to that.
We also specify that it is possible to approximate the behaviour of the space
around us, which is isotropic enough.
We also need to assume that the set of possible universe descriptions
is large enough, i.e. it has the same size as the set of real numbers.

Section \ref{sec:axioms} contains axioms for all the assumptions mentioned
above, each with an argument explaining why the axiom makes sense.

Having these, we can start the main part of the argument,
which says that, in the context of all possible universes that could
have human-like beings, our universe is extremely unlikely unless a Designer
intended it, the most obvious reason for that being that the Designer wanted
to design for human beings.

The main issue is that our universe seems to behave according to relatively
simple laws that do not change with time and space.
That is completely
unplausible if it happens by chance, so there should be an explanation for it.
If our universe is not designed, then the other explanation possible is that
there is something outside of our universe that made it likely for our universe
to exist (called a meta-universe).
However, it turns out that a meta-universe that
would explain our universe is also unlikely, and would itself need an
explanation, so there should be a meta-meta-universe explaining the
meta-universe, which would also need an
explanation, and so on, until we get an infinite chain of meta-universes.
But this chain would also need an explanation, so perhaps there is a
meta-infinity universe that explains it.
Since this meta-infinity universe would also need an explanation, we can
continue this process for quite a while before it breaks,
which makes these meta-universes also unplausible.

This means that, if our universe is not designed, then it will not behave
according to simple laws, and, in principle, we should be able to notice it.
However, with our current travelling means, we can observe only a limited
part of our entire universe, so, perhaps, we live in a bubble that covers
a large amount of time and space, in which simple laws apply.

If that's the case, our universe is still unlikely among all universes
containing such bubbles, because our bubble is much larger than it needs to
be.
A conservative estimation produces a probability so low for our universe,
that almost any other explanation is preferable.

It's harder to estimate the same probabilities when there is a Designer, but
it is likely that the probability for our universe is much higher than in the
non-design case.

Section \ref{sec:valid-options} describes this argument in detail.

Section \ref{sec:options} presents
the main issues discussed in this paper.

Section \ref{sec:objections} presents some clarifications and possible
answers to various objections to this paper.
Section \ref{sec:conclusion} presents this \paper{}'s argument conclusion.

This \paper{} uses some mathematical notions related to set cardinalities,
probabilities and ordinals, which can be found in, e.g., \svn{quote}.
These are summarized in \ref{sec:background}, together with some properties
derived from them.

\section{The Ordered Universe Argument}
\label{sec:ordered-universe}

%TODO: Urmatoarele trei alineate sunt nasoale.
The great Catholic theologian Thomas Aquinas,
in his \ghilimele{fifth way}, attempts to show God's existence from
the order of the universe, i.e. that almost all bodies, almost always
behave according to simple natural laws.
One can find a good exposition of this argument in
\citetitle{swinburne1968}%\citeauthor{swinburne1968} \citeyear{swinburne1968}
\parencite{swinburne1968}, but let us look at a few ideas which are
interesting in the context of this \paper{}.
%Metaphysics by Peter van Inwagen, from 2015, says that it's a powerful and sophisticated defense.
% 67 de citari https://scholar.google.ro/scholar?cites=18062876017816040453&as_sdt=2005&sciodt=1,5&hl=en

There are two types of order one may consider for showing God's existence,
the spatial order and the temporal order.
The former is the order that can be seen in (a part of) the universe
at a given moment in time, e.g. that planets, living bodies, and other things
are ordered.
The latter can be seen in the behaviour of things, including the laws of
nature, and it's the one that will be used in this \paper{}.

While it adresses many possible objections to Aquinas' fifth way,
\citetitle{swinburne1968} leaves one of them open:
this argument is based on an analogy between the order of world and the order
produced by people, which limits its strength.
I think that this \paper{}'s argument, although it builds on the same
foundation, needs only a fairly weak analogy
(see Section \ref{sec:design-probability}).

The following quote from \textcite{swinburne1968}, made when addressing
Hume's objection that the order which can be observed in this universe
is just an accident, makes a nice introduction for the argument
described in this \paper{}:
\begin{quote}
But if we say that it is chance that in 1960 matter is behaving in a
regular way, our claim becomes less and less plausible as we find that in
1961 and 1962 and so on it continues to behave in a regular way. An appeal
to chance to account for order becomes less and less plausible
the greater the order.
\end{quote}

% TODO: Maybe list some of Hume's issues with this argument.

\section{Possible Universes and Their Descriptions}
\label{sec:possible-universes}

If our universe is designed, then it's likely to be the way it is because
its Designer wanted it to have certain properties.
In order to understand why our universe works the way it does,
one would need to understand the intent of its Designer.
While that is interesting in itself, I will not try to pursue it here,
except for a few limited ideas.

For most of the remainder of this \paper{}, let us consider the other case.
Let us assume the hypothesis that our universe is not designed
and let us try to make a prediction based on it.
How would a non-designed universe look like?
Would it be similar to our universe?
Maybe an infinity of universes exist and ours is just one of many,
or maybe our universe is the only one that exists.
Even if ours is the only one, one could easily imagine that it worked
in a different way, e.g. maybe some constant like the speed of light would be
different, or maybe gravity would work differently.

There are people who claim that all logically possible universes exist,
either because they think that it simply makes sense, or because they want to
give a good account of modality, or for other reasons.
If that's the case, it seems, at first sight, that making predictions about
non-designed universes is rather hard.
However, this \paper{} argues that there are certain things that can be said
regardless of how many universes exist, whether it is one, or all logically
possbile ones, or anything in between.
%The argument in this \paper{} does not assume that more than
%one universe exists, but it should work either way.

A possible universe
could have exactly the same fundamental laws as ours, but with matter
organized differently.
It could have similar laws, but with different universal constants.
It could have different fundamental particles (or fields, or whatever the basic
building blocks of our universe are, assuming that there are any).
Or it could be completely different, i.e. different in all possible ways.

It could be that our logic and reasoning are universal instruments,
but it could also be that some of these possible universes are
beyond what our reasoning can grasp and others have properties
for which our logic is flawed.
Even if that's the case, let us see if we can say anything about
the possible universes that we could understand and could model in some way.

In the following, the \definitie{possible universes} term will denote
only the logically possible universes which we could model (with a few
more constraints that will be added below). However, the word
\ghilimele{possible} is ambiguous, so when the distinction between logical
possibility and actual possibility is important, the
\definitie{conceivable universes} term may be used instead.

This notion of model is not precise enough.
Let us restrict the possible universes term even more,
to the possible universes that could be modelled mathematically,
even if that may leave out some of them.
This may seem too restrictive,
especially since this paper only needs universes which can be approximated
by mathematical models.
We are going to relax this when talking about approximations, but, for now,
let us consider only universes
which are modellable with sets of axioms that are at most countable.

Let us restrict the universes we are considering even further, to universes that
have something remotely resembling time and space, for which
\ghilimele{the state of the universe at a given moment in time}, or something
close, makes sense, and which can plausibly contain intelligent beings.
% And let us restrict again, to universes where there is no action at a
% distance.
Any such universe is, for the purpose of this paper, a conceivable universe.

To keep the exposition simple, in the following I will use
\ghilimele{the state of the universe at a given moment in time},
but one should replace it with one's favourite alternative concept, e.g.
with \ghilimele{the past of a hyperplane whose points only have spacelike
intervals between them}.

Let us define a \definitie{universe description} to be a
consistent mathematical theory that has
a set of axioms which is at most countable and which allows making
predictions about the future state of the universe given its state
at a certain moment in time. A \definitie{universe region description}
is something similar, but only for a given space-time region of a universe,
with extra axioms to take into account the boundary conditions when predicting.
In the best case, for a deterministic universe, there might exist
a description which allows one to correctly predict the entire future state
given the state at any moment in time, but a universe description
as defined here does not have to predict everything and,
even when it predicts something, it does not have to always be correct.

\ghilimele{Predicting}, as used above, would normally mean that one
starts from the theory and does some formal inferences and computations, having
the prediction as the result.
However, as \cite{Calude2013} shows, there are many
things that can't be proven this way.
We don't know if the state of the universe at a given moment in time is one of
those things, although we could restrict our descriptions to ones where this
is possible, at least up to a reasonable level.
Regardless, let us use a different meaning: a theory predicts something if
that something is true in all models of that theory.

Note that usually the data available for making predictions is dependent
on who is making the prediction. As an example, if we assume that
all predictions are about things that can be perceived, directly or indirectly,
then
each kind of intelligent beings (e.g. humans) will make predictions
about the universe as seen through their senses. If a universe contains
multiple kinds of intelligent beings, with different kinds of
sense organs, then that universe may have descriptions which are
very different.
Of course, things that are not observable directly can sometimes be mapped
to things that are observable, but this may not be always true.

In order to handle this dependence on who observes the universe
in a reasonable way, in the reminder of this paper we will work with universes
that contain intelligent beings,
and all predictions will be relative to what these intelligent beings
could observe.
If there are multiple kinds of intelligent beings in the
universe, whenever we are talking about its description
we will assume that we picked one such kind.
I.e. although we
will talk about universes and their descriptions, we'll actually mean
(universe, intelligent-being-kind) pairs and their corresponding descriptions.

Next, let us try to specify how good an univese description
should be. First, predictions speak about the future, but expecting to
predict everything until the end of the universe (if any) may not be
reasonable. We may want to fix an amount of time $\Delta t$,
focusing on predictions about things that are at most
$\Delta t$ in the future. Second, we should't expect to
be able to describe everything with full precision, so we may want to
have a precision $\eta>0$ for all the values that are predicted.
Third, we shouldn't expect predictions to always be correct, so
we should require that they are true with probability $p>0$.
The exact meaning of \ghilimele{true with probability} here is left open,
except that we require $p$ to be the same for each location where
predictions are being made.
Of course, we may add other similar constraints if needed.

As an example, we could ask that, out of all predictions that we can make
at a given spacetime location, a fraction of $p$ turn out to be true.
If we are making
statistical predictions, then the observed outcomes would be consistent
with a fraction $p$ of all sentences representing statistical predictions
being correct.

Then let us say that an \definitie{approximate universe description} with a
\definitie{level of approximation} $L=(\eta>0,$ $p>0$ and $\Delta t>0)$
is a universe description which allows approximating the future
state of the universe with a precision $\eta$, with a probability
$p>0$ for a prediction to be correct
and for a limited amount of time $\Delta t$.

There is a distinction that we should make.
When predicting (say) weather we can't make long-term precise predictions,
and this happens because weather is chaotic, that is, a small difference
in the start state can create large differences over time.
This would happen even if the universe would be deterministic
and we would know the laws of the universe perfectly, as long as we don't know
the full current state of the universe.
However, high precision predictions may be possible for a deterministic
universe if the full state
is taken into account and, as we mentioned, we assume that we know the full
state of the universe at a moment in time when making predictions.

For a given universe or region of a universe,
given a level of approximation, we will pick a canonical description
in the following way: Let $S$ be the set of descriptions which approximate
the universe with the given level of aproximation. If $S$ contains
at least one finite description, then we pick the shortest such
description as \ghilimele{the canonical description}, breaking ties by
using the lexicographic order. Otherwise, we simply say that the
universe (region) has an infinite description, and we will abuse the
terminology a bit by picking the entire set
$S$ as the canonical description (we could
also pick a random description from the set).
If the level of approximation is obvious from the context, we will call
this canonical description \definitie{the universe's description}
or the \definitie{universe region's description}.

One could also use a well-ordering on the real numbers to choose the
lowest description as the universe's description, but that would
complicate things without any benefit.

\section{Options for Our Universe}
\label{sec:options}

The reminder of this paper will analyze what we can reasonably believe about
the following issues:
\begin{itemize}
  \item Our universe is designed or not.
  \item Our universe has a finite or infinite description.
  \item Option 1: There is a meta-$\beta$ universe for each countable ordinal
        $\beta$ such that our universe is the meta-$0$ one and the meta-$\beta$
        universe includes, directly or not, all meta-$\alpha$ universes with
        $\alpha < \beta$.
        Option 2: there is an ordinal $\beta$, possibly $1$,
        where this stops being the case.

        Whether there is an uncountable chain of meta-universes, in which
        our universe is the first one, and in which any meta-universe
        includes, directly or not, all the previous ones.
  \item The set of possible descriptions for a finite chunk of space-time
        that are also compatible with life has at least the cardinal
        of the set of real numbers, $\reale$, or a smaller one.
\end{itemize}

\section{Axioms}
\label{sec:axioms}

\subsection{Observing Events}
In the main argument we will try to use the fact that we are observing our
universe, and the laws describing its behaviour are one element picked from
a very large set, a set having the same cardinality as the set of real numbers.

The most natural probability distributions over real numbers
or over sets with the same cardinality are the continuous
ones, i.e. probabilities for which any element of the set has probability $0$.
Practically, this means that we would lose any bet which we would make on an
single element of the set.

In order to make some sense out of this, we will use two terms,
\ghilimele{generic} and \ghilimele{peculiar},
which are defined precisely in section \ref{sec:probabilities}.
Informally,
an object is peculiar if it satisfies a peculiar predicate, and a peculiar
predicate is one that has a zero probability for any continuous probability
distribution.
As an example,
\ghilimele{has a finite number of digits} is a peculiar
predicate over real numbers, and any real number with a finite number
of digits, like $12.5$, is peculiar. An object or predicate is generic if
it is not peculiar.

Axiom \ref{ax:zero} below specifies this in a more formal way.

However, it might happen that, for some of the sets used in this \paper{},
no reasonable probability distribution can be defined.
Axiom \ref{ax:noprobability} specifies how choice works in those cases.

Throughout this paper we will implicitly use only separated probability
measures, i.e. they can measure singletons (single-element sets).

\begin{axiom}
  \label{ax:zeroisgeneric}
  If $P$ is a probability over the set of real numbers\footnote{Note
    that here, and in all the axioms in this paper, it is not required that
    $P$ is a probability over the
    Borel algebra of $\reale$, although that is, in many cases, implicitly
    assumed when talking about probabilities over $\reale$.
  }
  (or a set with the same cardinal)\footnote{
    Readers should
    keep in mind that, in most cases throughout this paper, what is being said
    about the set of real numbers is similarly valid for any set with the same
    cardinality.},
  we observe a real number $x$, and $P(x)=0$, then $x$ is generic.
\end{axiom}

The set of events for which $P(y)$ is
greater than $0$ is at most countable (see section \ref{sec:probabilities}),
and, if we remove them from $\reale$, what remains will still have
the same cardinality.

On this later set, the probability
of all peculiar events taken together is $0$, so there is no chance of us
observing one.
In other words, the probability of all generic events is $1$\footnote{This
  requires that, in the original set, the set of elements with $0$ probability
  has itself a non-zero probability. However, this is implied by us observing
  an element from this set.
},
so we can be sure that we observed a generic event.

Of course, the (logical) possibility of observing a peculiar event still exists,
but, practically, we will not observe it as long as the set of our
observations is at most countable.

\begin{axiom}\label{ax:noprobability}
  If we observe a specific real number $x$, when we could have
  observed any real number, and there is no probability distribution that could
  describe how $x$ was chosen, then $x$ is generic.
\end{axiom}

Note that this axiom does not say that we do not know that probability
distribution, it says that there is no such probability distribution.
Anyone believing that this cannot happen should treat the cases where
this axiom applies as invalid.

Also note that this cannot happen when using subjective probabilities.

If there is nothing that could favor
peculiar numbers over generic ones, it's absurd to think that we could have
observed an element of such a tiny set among something infinitely larger.
Also, the similar axiom for probabilities above suggests that this is the only
reasonable assumption in this case.

\subsection{$\reale^4$ Universe}

The universes we are interested in are universes that can be
modelled on top of a finite-dimensional space based on real numbers,
or something close enough to that.

In order to not define what \ghilimele{close enough} means, we will use
the following axiom, which is true for any finite dimensional space
based on the set of real numbers, but it also works on spaces based on,
say, the set of rational numbers, and on many similar finite dimensional spaces.

Let us define a \definitie{generalized rational number} as being either a
rational number, or one of $-\infty$ and $+\infty$. Let us denote by
\ghilimele{cuboid} a corner-based shape in the $n$-dimensional space
(we could take it to mean \ghilimele{hyperrectangle}, \ghilimele{hypercube},
or any similar shape).

\begin{axiom}\label{ax:rationalcovering}
  The set of (generalized) cuboids using the same dimensions as our space-time
  and whose corners' coordinates are generalized rational numbers (i.e. they
  belong to $\rationale\cup\multime{-\infty, +\infty}$),
  is countable and covers our universe.
\end{axiom}

Note that any spacetime based on real numbers, i.e. included
in $\reale^\alpha$, will be included in the generalized cuboid having
its corners at plus or minus infinity.

The axiom above does not require the space-time to include the cuboids or
their corners.
If, say, our universe is based on the set of integers, i.e.
it would be included in $\intregi^\alpha$,
we could consider it as being included in $\rationale^\alpha$,
where we could check if it is included in one of the cuboids mentioned
above.

The following definition could also be written as an axiom.

\begin{definition}\label{finitecuboid}
  A part of our universe is \definitie{finite} if
  it can be covered with a finite cuboid, i.e. one
  for which all corner coordinates are rational numbers.
\end{definition}

There are a lot of possible definitions for \ghilimele{finite} which are not
included here.
While this paper could probably be extended to also handle many of these,
most likely there is no point in doing so.
As an example, in $\reale^3$ one can define it as
\ghilimele{having a finite volume}, which would
mean that there are finite things that do not fit in finite cuboids.
In order to handle this, when defining a level of approximation,
we could allow ignoring what happens in a small part of the region we model.

\subsection{Logically Possible Universes}

The following axiom states that, for a given level of approximation,
there is a large set of conceivable universes, which in some narrow respects
are similar to ours, but which are, in general, wildly different.
Also, our universe belongs to this set.

\begin{axiom}\label{ax:uncountable}
  For any level of approximation $L$ above a certain minimum level (see below)
  there is a set $\descriptions$
  of universe descriptions such that the following are true:
  \begin{enumerate}
    \item $\descriptions$ has the same cardinality as
          the set of real numbers, $\reale$.
    \item For all descriptions $d$ in $\descriptions$
          there is at least one conceivable
          universe $U_d$ which
      \begin{enumerate}
        \item has a time-space or something similar enough;
        \item can plausibly contain intelligent beings that use mathematics;
        \item for the intelligent beings mentioned above and
              for the level of approximation $L$,
              $d$ is $U_d$'s description.
      \end{enumerate}
    \item A description can be used only for its universe, i.e.
          if $d$ and $d'$ are descriptions from $\descriptions$,
          then $d$ does not work
          for $U_{d'}$, the universe corresponding to $d'$.
    \item $\descriptions$ contains a description for our universe.
  \end{enumerate}
  The same is true for universe regions, except that
  $\descriptions$ may have a lower
  cardinality.
\end{axiom}

The minimum level for which this is true is left unspecified, but we should
include some common-sense restrictions, e.g. the minimum length we would
need to measure is not below Planck's length.
All levels of approximation used below will be above this
minimum level, even if this is not mentioned explicitly.

To see why this axiom is reasonable, we will first show that our universe's
description could be part of such a set, then we will identify a large set
of descriptions that fulfills all the requirements except for containing
our universe's description, then we will show that, by removing some of the
descriptions in that set and adding ours, we get a set that fullfils all
requirements.

Let us note that,
%in a similar way to axiom \ref{nonessentialhypercube},
most likely, we have an approximate description for the observable part
of our universe
given by classical mechanics, maybe with some additions.
Alternately, one could use a description based on, say, quantum field theory.

We can assume, then, that there is a description,
(possibly different from the one above), that works for our entire universe,
as we are theoretically able to observe it. In the worst case, the description
would be just a recording of what we would observe (i.e. at a given time, in
a given place, a certain event takes place).

For almost any countable axiom system that still
has an $n$-dimensional real space, $\reale^n$, as a base,
one could imagine an alternate universe
which, in the present, is exactly like ours inside (say) Mars' orbit,
but what is what is outside of Mars' orbit is described by that axiom set.

Some of these axiom sets would describe laws of nature which still allow life
to exist inside Mars' orbit, are similar enough
to ours to allow us to observe what happens outside of this orbit,
but different enough that
we would notice (e.g. gravity could work differently, depending on the region
of space in which one travels).

So we could then try to take all hypothetical universes with infinite space
or time, and we could split them in an infinite number of regions. If we take
all ways of having different enough laws for these regions,
we get a set of universes whose set of descriptions has the same cardinality
as the set of real numbers and which fulfills all conditions of the axiom
above except the last one.

To be more precise, for any approximation level $L$, and any region that
is not trivial for $L$
(the meaning of \ghilimele{trivial} is left open, but, as an example,
if we use an approximation level that does not distinguish things smaller
than a size $l$, then the region must be significantly larger than $l$),
there are multiple possible descriptions that
are distinct for $L$, i.e. there is no possible region where both descriptions
would be valid within the level $L$ (see \ref{sec:not-enough-descriptions}).
We can also ask that there same applies for the two descriptions taken together
with an approximate description that we use for the universe around us.
I.e. there is no possible region where one of the two descriptions could
be applied at the same time as our description.
By splitting an infinite timespace
into disjoint regions defined by a set of finite rational coordinates
(e.g. we could split it into cubes whose edges have length $1$ and whose
corners are integers)
and taking all possible ways of assigning descriptions to these regions
we get a set of universe descriptions with infinitely countable axioms
which we will denote by $\descriptions$.
Since the set of regions is countable, and we can assign at least two
descriptions to each region, we get a set of assignments with the same
cardinality as $\reale$ (also see \ref{sec:cardinals}).

This does not change if we identify a region that can sustain intelligent life,
whose description is fixed,
and we require that the other regions' descriptions are compatible with life
in the fixed region.

Also, let us note that, from the way we constructed it, we can add our
universe's description to this set without breaking any of the requirements.

This means that the set of descriptions mentioned in the axiom,
$\descriptions$, has the cardinality of $\reale$. See section
\ref{sec:fewuniverses} for another take on this issue.

Also see \parencite{Manson2003}, which suggests that something similar
might be happening in our universe.

\subsection{Neighborhood Modelling}

Since currently we can not observe our entire universe, saying that it has a
high level of complexity is of limited use.
The second part of our argument will try to draw some conclusions from
what we see in the observable part of our universe.
In order to do that, we need an axiom saying that, around us, perhaps in the
entire observable part of our universe, there is a relatively simple
set of laws that describe it.

\begin{axiom}\label{ax:finiteneighborhood}
  There is a large compact time-space region of our universe which
  includes our solar
  system, and there is a level of approximation $L$ such that:
  \begin{enumerate}
    \item Any cuboid included in that region has a finite approximate
          description for the level $L$ (i.e. we can make non-trivial
          approximate predictions in all such cuboids).
    \item A description for one of the cuboids also works for all other
          cuboids with the same size in the given time-space region
          (i.e. the space region is isotropic enough).
  \end{enumerate}
\end{axiom}

Usually we assume, implicitly or explicitly,
that this is true, and, even more, that a cuboid's description works
for the entire universe.
This is especially visible when, e.g., claiming
that the universe is around $14$ billion years old, that the sun will,
in some distant future, become a white dwarf, or that standard-candle supernovae
are not ilussions.
In order to believe this, one must assume that our universe has a
finite approximate description or, at least, that our
solar system/galaxy/observable part of the universe has such a description.

\section{Valid Options for Our Universe}
\label{sec:valid-options}

This section will try to develop the axioms above in order to find out what
is reasonable to believe about the issues presented in
section \ref{sec:options}.

We will focus mostly on what happens when our universe is not designed
since in this case it is easier to make predictions about our universe
and to falsify them, but we will also take a look at created universes.

\textcite[][Section \ghilimele{Why a world with human bodies is unlikely
if there is no God}]{Swinburne2003} comes sort of close to the argument
presented here, but while \citeauthor{Swinburne2003}
argues that human bodies are unlikely, I am arguing that, in the context
of all possible universes that could have human-like beings,
our universe is extremely unlikely unless a Designer intended it,
and, even more, the Designer wanted to design for human-like
beings.
Since the existence of human bodies is not directly related to the subject
of this \paper{}, I will not discuss that section more
than it is strictly needed.

Note that  \citeauthor{Swinburne2003} says that individual sets of laws
have non-zero probability while I'm claiming that their probability is $0$.
It seems to me that \citeauthor{Swinburne2003} implicitly assumes that
such a set has a finite number of laws, while I am explicitly removing
that constraint, so both can be right within their contexts.

\subsection{Peculiar Descriptions and Meta-universes}
\label{fdaumu}

\subsubsection{Informal summary}

From what we observe around us, it seems that our universe is fairly homogenous,
having a relatively simple approximate description.
In this section we will try to see how plausible it is that this description
(or any finite description that still approximates what we see around us)
applies to the entire universe, not only the space and time around us.
To do that, we will that a finite description is peculiar, which means that
there is no chance of
us observing it both in the case when we use a continuous probability
distribution over universes, and in the case when we can't use any probability
distribution.

If that's the case, then the correct probability distribution used over
universe descriptions should be discontinuous.
If our universe is not designed, then we don't normally have any reason of
using a discontinuous probability distribution, which may be enough to
show that any approximate description for our universe's description is
infinite. This should settle at least the case when this probability
distribution is a subjective one.

However, we will look deeper into this issue.
Let us assume that there is a probability distribution over universe
descriptions, and that this probability distribution has discontinuities.

First, we can ask ourselves why is there a probability distribution over
universe descriptions.
A probability distribution means that which universes exist and which don't
is not completely chaotic, there is a minimal amount of order among them.
We will call this order a \ghilimele{meta-universe}, and we will say that
this meta-universe includes ours, but keep in mind that the intuition that
many have about a meta-universe is something more complex than just
a minimal amount of order.
However, the most natural explanation for this order is that our universe
is part of something larger that corresponds to the intuition of a
meta-universe, which may contain many other universes besides ours.
To put this entire paragaph in other words, the probability distribution over
universe descriptions is something determined by a meta-universe which includes
our universe.

Now we have something that explains this probability distribution which,
as mentioned above, is not continuous.
Next, we can ask ourselves if this distribution's discontinuities are what
we would normally expect or not.
A natural question is whether we have a probability distribution describing
what we shoudl expect from these discontinuities, and what kind of probability
distribution it would be.

The set of discontinuities for a probability distribution is at most countable
(TODO: reference), so, if they are described by a continuous probability
distribution, or if there is no such probability distribution, we would
expect them to be generic.
This would still mean that our universe description is generic.

However, as above, we can try to explain a peculiar universe description
using a discontinuous probability distribution over discontinuities,
which means that there would be a
meta-meta-universe that includes both our universe and the meta-universe.

But having peculiar discontinuities for the meta-meta-universe probability
distribution is plausible only if there is a meta-meta-meta-universe
with a discontinous probability distribution.
By repeating this reasoning, we get an infinite chain of meta-universes.
But every probability distribution in this entire chain must have a
discontinuity for our universe's
description, which, using the same reasoning as above, must be generic,
unless there is something explaining it, i.e. a meta-infinity universe,
whose probability distribution has a discontinuity for our universe's
description.

In order to make this chain more precise, we will assign a number corresponding
to the meta-level of each meta-universe.
We will assign $0$ to the meta-universe including ours, $1$ to the
meta-meta-universe including (directly or indirectly) the meta-universe and our
universe, $2$ to the next level, and so on.
Then we can denote by $\omega$ the meta-infinity universe that explains why all
the meta-universes corresponding to finite numbers have a certain discontinuity.

But the $\omega$ meta-universe also needs an explanation if its probability
distribution has a peculiar discontinuity, so, again, either our universe's
description is generic, or there is a meta-universe with a discontinous
probability distribution that includes the
$\omega$ meta-universe.
We will call this the $\omega + 1$ meta-universe.

In the same way, we need meta-universes corresponding to $\omega + 2$,
$\omega + 3$, and so on.
I.e., we get another infinite chain of meta-universes, which also needs an
explanation, provided by a meta-universe that we will call $\omega + \omega$
or $2\omega$. But that one still needs an explanation, so we also need
meta-universes corresponding to $2\omega + 1$, $2\omega + 2$, ..., and,
in the end, corresponding to $3\omega$.
Next, we will need meta-universes
corresponding to $4\omega$, $5\omega$, and so on.
But to explain this infinite chain, we need another meta-universe which we
will call $\omega\omega$, or $\omega^2$.

Continuing, we will need meta-universes up to $\omega^3$, then up to
$\omega^4$, and so on.
In the end, we will need one for the entire
infinite chain, denoted by $\omega^\omega$.
However, we can't stop here, and we will not be able to stop for quite a while.

Each of these meta-universes corresponds to a statistical observation and,
as long as we make a countable number of observations, we either expect to
observe generic discontinuities, or we expect to find yet another meta-universe
above everything that we observed so far.

This construction corresponds to ordinals, see (TODO: ref) to find out more,
including the place where this process breaks, i.e. the first uncountable
ordinal.

Postulating that such an improbable and complicated construction exists does
not seem reasonable, so any description for our universe must be peculiar.

Next section will contain this argument presented in a more formal way.

\subsubsection{Main argument}

Let us denote our universe's description by
$\our_description$. Let us take $\descriptions$ be the set of descriptions from
axiom \ref{ax:uncountable}, with $\our_description$ belonging to $\descriptions$.
As mentioned above, let us also consider that, perhaps, our universe is
contained in a meta-universe, which is, perhaps, contained in
a meta-meta-universe, and so on, and let us label these meta-universe
with ordinal numbers, starting with $0$ for the first meta-universe.

As above, a meta-universe is something that contains our
universe and which influences, one way or another, which universes could
exist and in which quantity.
This influence will be represented as a probability
distribution over universe descriptions.
Meta-meta-...meta-universes are defined in a similar way.

In principle, it could happen that there is a meta-universe without
a corresponding probability distribution, but we will mostly ignore that,
since axiom \ref{ax:noprobability} applies both when there is no meta-universe
and when the meta-universe does not have a corresponding probability
distribution.

As an example, our universe could be one of the many universes in a
meta-universe that would allow only the existence of universes having the
same laws as ours, but with different fundamental constants.
The probability distribution mentioned above would be zero for all
plausible universes with different laws, and for the universes that have
the same set of laws, it would be equivalent to a
probability distribution over the fundamental constants.

Alternately, the probability over universe descriptions could be a
subjective one, i.e. it can measure what we believe about what can exist.
The text below is usually written with the objective meaning in mind, but
most conclusions should also work for subjective probabilities.

The probability distribution of a meta-universe describes, at the same time,
which universes exist, and which universes could exist.
One way of thinking about this is to think about any process that we usually
consider to be random, like rolling a die.
Before the first roll, we have no events.
Each roll adds the number on the die to the list of events.
At any point, the same probability distribution describes both how we got the
existing events, and how we will get any similar event in the future.

As another example, let us take a look at what happens if only one
universe exists.
If there is absolutely no reason
for it existing and the other ones not existing, then no meta-universe which
includes it exists.
If, somehow, this universe is contained in a meta-universe which
enforces that only one universe exists, the probability distribution will
assign $1$ to this universe's description, and $0$ to everything else.
If other universes could exists, but, somehow, they don't, we will get a
probability distribution specifying what could exist, and we would be talking
about a meta-universe in which only one universe was \ghilimele{generated}.

As mentioned at the beginning of this section, we are associating
meta-universes with ordinals. Let us consider the case
when these universe ordinals stop at some point, i.e. when there are ordinals
which do not correspond to any meta-universe.
If the smallest
ordinal not associated with a universe is countable, let us denote it by
$\alpha$.
If the smallest such ordinal is not countable,
or if there is a universe for each ordinal,
then let $\alpha$ be the lowest uncountable ordinal. $\alpha$ will
be an upper limit for any ordinal that we will consider in the following.

If $\alpha = 0$, then there is no meta-universe containing ours, so there
is no probability distribution over universe descriptions,
which means that $\our_description$ is generic (axiom \ref{ax:noprobability}).

If $\alpha > 0$ and the probability distribution over universe descriptions
given by the first meta-universe is continuous,
then the probability of all peculiar descriptions is $0$,
so the only reasonable conclusion is that our universe has a generic
approximate description (axiom \ref{ax:zeroisgeneric}).

This result should probably be good enough, and if you agree with
this, please skip to section \ref{sec:peculiarregions}.

However, one could also wonder what happens when this probability distribution
is not continuous. If the discontinuities are generic, then we can easily
show that, with probability $1$, our universe's description is generic.

However, if they are not generic, then we can ask ourselves
whether we have a probability
description for these discontinuities. Since the meta-universe containing our
universe may be contained in a meta-meta-universe, maybe the meta-meta-universe
can provide such a distribution. If the meta-meta-universe probability
distribution is continuous, then the discontinuities are generic. If not,
then we have to ask ourselves if a meta-meta-meta-universe can give us more
information.

If the meta-meta-meta-universe contains a probability distribution that tells
us what is the probability that a given number is a discontinuity for both
the meta-universe and the meta-meta-universe, then, if this distribution
is continuous (or there is no such distribution),
the discontinuities are generic. If not, we can go to the next meta level.

Let $\beta$ be ordinal where this process stops (if any such ordinal exists)
i.e. let $\beta$ the least ordinal lower than $\alpha$
for which one of the following happens:
either there is no meta-universe corresponding to it
(and no probability distribution), or the probability for our universe's
description is $0$. Note that $\beta$ may be infinite.

%In general, let us let us assume that for some ordinals $\delta$ between
%$1$ and $\alpha$ (inclusive) there is a probability distribution
%$p_\delta$ over the set of universe descriptions $\descriptions$, telling us
%the probability that a given description is a discontinuity
%for all lower ordinal probabilities, i.e. for all
%$p_\gamma$ with $\gamma < \delta$.
%Since $p_0$ is not defined by the above, we will take it to be the
%probability distribution for universe descriptions.

%So, $p_0$ will be the probability distribution over universe descriptions,
%$p_1$ (if it exists) will be the probability distribution telling us
%what's the probability that a given description is a discontinuity for $p_0$,
%$p_2$ will tell us
%what's the probability that a given description is a discontinuity for both
%$p_0$ and $p_1$, and so on.

%Let $\beta$ be the minimum ordinal (if any) which is lower than our limit
%$\alpha$ and for which either there is no probability $p_\beta$,
%or the probability of our universe's description is $0$, i.e.
%$p_\beta(\our_description) = 0$.

If $\beta$ exists, let us note that we can consider that
we made a countable number of observations on various
probability distributions, one for each ordinal/meta-universe, and that
each time we observed our universe's description.
Since one of them was made on a probability distribution
where either there is no probability distribution, or the probability
of $\our_description$ is $0$, then $\our_description$ is
generic (axioms \ref{ax:noprobability} and \ref{ax:zeroisgeneric}).

If such a $\beta<\alpha$ does not exist, then for any ordinal $\delta<\alpha$
there is a probability distributions which has $OURD$ as a discontinuity.

If $\alpha$ is countable, then,
similar to the above reasoning, $\our_description$ is generic.

This means that, if our universe is not designed, we have two options
that might be reasonable:
\begin{itemize}
    \item our universe's description, $\our_description$, is generic
    \item $\our_description$ is peculiar, $\alpha$ is uncountable and,
          for all $\delta < \alpha$, the corresponding probability distribution
          exists and, according to it, $\our_description$'s probability is
          greater than $0$.
\end{itemize}

In other words, in order to claim that $\our_description$ is peculiar,
one needs to postulate the
existence of an uncountable chain of meta-universes, all of them favouring a
peculiar $\our_description$, which, by default,
is prohibitively unlikely for any of them.
But that's not all, since, although the
current argument does not work anymore when the chain of
meta-universes becomes uncountable, intuitively the peculiarness problem
still remains: why would it suddenly become reasonable to make
only peculiar observations if we make enough of them? A few, maybe, but all
of them? Normally, when we make
observations with a continuous probability distribution, which is the default,
we expect to observe only generic elements as long as we make a
countable number of observation, and only when the observation set
becomes uncountable we expect to, perhaps, also observe some peculiar elements.

From now on, I will assume that the possible objection in the preceding
paragraph is unreasonable, which means that, practically speaking,
either $\our_description$ is generic or our universe is designed.

Since we don't know the limits of our universe, and how similar
the entire universe is to the part we can observe, let us consider next
what happens for universe regions.

\subsection{Peculiar Descriptions for Universe Regions}
\label{sec:peculiarregions}

Let us consider all generalized cuboids whose corners' coordinates are
rational numbers or $+\infty$ or $-\infty$. From axiom
\ref{ax:rationalcovering}, their set is countable.

Let us examine whet happens when the set of descriptions for a cuboid that
are compatible with life in that cuboid (and, perhaps, in the universe
around it) has the same cardinal as $\reale$: we can apply the same argument
as in section \ref{fdaumu}. The main difference would be the way we assign
ordinals: the region would correspond to ordinal $0$, 
our universe would correspond to $1$, the meta-universe would correspond
to $2$, and so on. I.e. we would treat the region as it would be an universe
and our universe as being its meta-universe.

Then, since the set of all the generalized cuboids mentioned above is
countable, all of their descriptions should be generic.

However, we usually believe that we can have finite approximate descriptions
for the observable part of our universe, or, at least, for a large part of if
(axiom \ref{ax:finiteneighborhood}).
This means that the only options
that have a chance of being plausible are that our universe is designed, or
that the set of descriptions compatible with life for a finite cuboid
has a cardinality smaller than $\reale$.

Let us focus on the latter case above.
Note that, from section \ref{fdaumu}, our universe has
a generic approximate description.
But we assumed that the region around us has a finite approximate description.
If we were to extend it to the entire universe, we would find a peculiar
description for our universe, which is a contradiction. This means that there
is at least one region in our universe which has a different description.

Let $A$ be the set of possible approximate descriptions for
a cuboid of size, say, $1\;second \times meter^3$.
As argued in the paragraph above,
unless our approximation level is extremely coarse,
$A$ will have multiple elements.
We will assume that we are working with a reasonable approximation level.
For any possible description $a$ in the set $A$,
let $P(a)$ be the probability of encountering
a cuboid with $a$ as its description in our universe.

If $A$ is finite, we should, by default, pick the uniform probability
distribution on $A$, assigning equal probabilities to all elements of $A$.
However, $A$ can have an infinite cardinal,
so we have to consider more general probabilities.

In any case, since, as mentioned above, $A$ has multiple elements,
we can't reasonably expect to have one element with probability $1$.
Then let $p_1<1$ be the probability of the description that we use for the
cuboids around us.
The probability of observing $n$
non-overlapping cuboids
with this description without observing any other description is\footnote{
  This assumes independence between the cuboids, but, given that our
  universe's description is a generic one, chosen among all possible
  ones without a Creator biasing it, this is a reasonable assumption.
}
$p_1^n$.

Even if $p_1$ is very close to $1$, $p_1^n$ converges quickly to 0.
As an example, if $p_1=0.9$ then observing $n=70$ consecutive cuboids with
the same description is enough to make $p_1^n$ go below one to one thousand
odds, $n=140$ is enough to go below one to one million,
$n=210$ goes below one to one billion.

Then, if we compare the non-design hypothesis
with another one non-zero probability,
almost always the consistency of a small spacetime region around us is enough
to make the non-design hypothesis unlikely enough to disregard\footnote{
  Let $ND$ be the hypothesis that our universe is not designed,
  $D$ be the hypothesis that it is designed, $\our_description$ be our
  universe's description,
  $our$ be the description for the region of space around us.
  Note that $P(our) \ge P(\our_description)$ since $our$ occurs in any
  universe with $\our_description$, but may also occur in others. Also, assuming
  that $\our_description$ has a non-zero probability in the design case, then
  $P(our) \ge P(\our_description)=P(\our_description|D)\cdot P(D) > 0$.
  From Bayes' rule, $P(D|our) = \frac{P(our|D)\cdot P(D)}{P(our)} > 0$.
  Similarly,
  $P(ND|our) = \frac{P(our|ND)\cdot P(ND)}{P(our)}
    = \frac{P(ND) \cdot p_1^n}{P(our)}$.
  To compare the two we have to compare
  $P(our|D)\cdot P(D)$ with $P(ND) \cdot p_1^n$.
  }.

Next, let us see hat happens if we assume the design hypothesis.

\subsection{Design Probability}
\label{sec:design-probability}

\svn{merge with unknown designer's intent.}

Let us examine the hypothesis that our universe is designed.
Since, in this case, the way our world works would based
on the designer's intent,
it is no longer obvious that, say, continuous probability distributions
should be the default.
On the other hand, do we actually have better options?

How large is the the probability of having
a large consistent time-space region when a designer is involved?
Can we repeat the previous argument to draw the same conclusion, that
the probability of observing our universe is vanishingly small?

We can separate this probability in two: the probability that a designer
would want rational beings\footnote{A designer could want a consistent universe
region with a finite
description without wanting intelligent beings, but this possibility is not
analyzed in this paper}, denoted by $p_r$, and
the probability that the universe region containing
those intelligent beings is consistent
given that it was designed for them.

I think it's safe to assume that the latter probability is positive,
and, probably, fairly high. See as an example this quote from
\textcite{Swinburne2003}, which argues that, if God exists,
there is a fairly good chance that humans can understand their universe:

\begin{quote}
  So, in order to have significant freedom and responsibility, humans need
  at any time to be situated in a \ghilimele{space} in which there is a
  region of basic control and perception, and a wider region into which
  we can extend our perception and control by learning which of our
  basic actions and perceptions have which more distant effects and causes
  when we are stationary, and by learning which of our basic actions cause
  movement into which part of the wider region.
  If we are to learn which of our basic actions done where have which
  more distant effects (including which ones move us into which parts
  of the wider region), and which distant events will have which basically
  perceptible effects, the spatial world must be governed by laws of nature.
  For only if there are such regularities will there be recipes for changing
  things and recipes for extending knowledge that creatures can learn and
  utilize.
  So humans need a spatial location in a law governed universe in which to
  exercise their capacities, and so there is an argument from our being thus
  situated to God.
\end{quote}

With that in mind, let us compare the probability that
we observe consistency under the \ghilimele{design for intelligent beings},
\ghilimele{design without intelligent beings in mind} and
\ghilimele{no design} hypotheses.

If the probability that a
designer would want rational beings, $p_r$, is $0$, then we can't
differentiate the design and non-design cases, and the
the probability of observing our world's consistency is the same
for both.

However, if $p_r$ is non-zero, then the
the probability of observing our world's consistency becomes larger in the
design case, since $p_r$ is providing some ammount of support for that
hypothesis.

Oversimplifying a bit, and assuming that the probability of having consistency
is large if the universe was designed for intelligent beings,
the ammount of support would
be roughly equal to the ratio between $p_r$ and the probability of observing
our world's consistency in the non-design case.

Since observing consistency without design has an extremely
small probability,
having almost any non-zero probability for
\ghilimele{a designer of worlds would want rational beings} would provide
overhelming support for the design hypothesis.
As an example, a probability
of $1$ in one billion billion would easily be large enough.

From a natural theology point of view, one may argue for a higher design
probability through analogy with our own intents: we would be interested
in intelligent beings, it seems to us that intelligent beings would be
interested in other intelligent beings, so maybe the Designer would be
interested in creating other intelligent beings.

How much is this analogy worth?
It's hard to tell, but, on the other hand, our options are:

\begin{itemize}
  \item Our universe is not designed,
        and the probability of observing the
        consistency that we see around us is vanishingly small.
  \item Our universe is designed, but the above analogy does not work and
        peculiar descriptions still have a zero probability.
        Again, the probability of observing the
        consistency that we see around us would be vanishingly small.
  \item The probability that the above analogy works is not vanishingly small,
        and we are observing the consistency that we would expect from a
        universe designed for intelligent beings.
\end{itemize}
For completeness, the first two options are valid only under the additional
assumption mentioned earlier:
for a given approximation level, in a universe that has intelligent life,
the set of possible approximate descriptions for a finite cuboid
must have a lower cardinality than $\reale$.

It seems to me that the only reasonable option is that the analogy is not as
bad as some people may think it is.

\svn{5. Use Swinburne's argument about why would God create human-like beings.}

\section{Objections and Clarifications}
\label{sec:objections}

This section includes various possbile objections to this argument. Since the
fine-tuning argument addresses the same problem, and it's also using a
probabilistic argument (though in a completely different way), some of the
objections below are similar to the fine-tuning ones, and it may be helpful
to compare the two approaches.

For the fine-tuning argument see, e.g., \parencite{sep-fine-tuning}.
For objections to the fine-tuning argument that are relevant here, see, e.g.,
\parencites{Manson2003}{Manson2009}{McGrew2001}{Narveson2003}{Sober2009}.
For possible answers see, e.g.
\textcites{Leslie2003}{Swinburne2003}{Monton2006}{Kotzen2012}.

%Manson2009 McGrew2001 - probabilities not applicable to cosmic parameter
%values.
%Manson2009 Sober2009 - observation selection effect
%Manson2009 - multiple universes
%Manson2003 - nonhomogenous spacetime, different fundamental constants
%Sober2009 Narveson2003 - unknown designer's intent, no difference to design

% TODO: Cite \ghilimele{Should we care about fine-tuning} by Jeffery Koperski.

\subsection{Observation Selection Effect and Multiple Universes}

In this \paper{}, we only look at universes that contain intelligent life,
and that restricts the set of possible universe descriptions.
Even more, we don't see a universe as it is, instead we see it
through the eyes of the intelligent beings inhabiting it.
It can be argued that, in a non-created universe, beings might be intelligent
only if their intelligence is useful to them.
But this likely means that those beings live in a timespace region which seems
consistent from their point of view, so maybe it's not that unlikely to see
consistency around us.

However, let us look more carefully at how much consistency we would expect.
There are possible universes with consistent regions in which
intelligent life can exist, and whose consistency ends abruptly at some
random time. There are possible universes whose consistent regions are
strictly the size needed for allowing intelligent life, and there are possible
universes with large consistent regions.

Let us assume that our existence means that some consistency is required.
Is there any non-required consistency around us, consistency which is due to
chance? To be more precise,
how large is the time-space region whose consistency is required? Well,
perhaps at very distant times in the past, the consistent region included
the entire observable universe, but, right now, there is no reason to require
full consisency outside of Earth's orbit. Even more, since non-consistency
only means
observable non-consistency, and does not require something wildly different,
it's likely that we don't even need full consistency inside of Earth's orbit.

However, as far as we know, our solar system is consistent, our galaxy is
fairly consistent, and distant galaxies are also fairly consistent.

This \paper{} argues that, since we do observe much more consistency
than we would expect, design is the right explanation.

However, there is a possible objection related to this: if multiple universes
exist, perhaps all possible ones, there will be some beings living in the
implausibly consistent ones.

While this is correct, the probability of an intelligent being
living in a fully consistent universe is still $0$.
In virtually all universes the nonhomogeneity of the universe would be
easily observable, meaning that, for the relatively few intelligent beings
living in the other universes, as long as the design hypothesis
has some plausibility, it would be unreasonable to think that
their universe is not designed (assuming that the argument presented in
this \paper{} is correct).

Also, observing a large consistent region of space is still very unlikely among
all existing universes.


\subsection{Unknown Designer's Intent}
\label{sec:unknown-designer-intent}

From a natural theology point of view, one can't
know what the Designer wanted
(e.g. one can't know that a universe designer would want to create
a universe having life) \parencites{Sober2009}{Narveson2003}, so, by default,
any argument showing that the probability of our universe is small
if it's not designed would also show that the probability is small
even if it is designed.
To fix this, one would need an independent way to show that the Designer
wanted the universe to have life \parencite{Sober2003}.
In order for this paper's argument to work, one would actually need a weaker
requirement, i.e. it would be enough to have a non-zero probability
to the hypothesis that the Designer wanted the universe to have intelligent
life.

This is an interesting objection for this approach.
If we can't assign a
non-zero probability to this hypothesis, then this argument can't explain
that the universe around us seems to be consistent.
Instead, it would just point out how extremely odd it is to observe it.
This would force us to take it as a brute fact that the universe has a
finite description (or we could just
live in a bubble of consistency whose probability becomes inimaginably
smaller with each passing second, which would be fairly similar).
In this case we would be in the right setting for the argument presented in
\cite{Swinburne2004}, which uses this consistency as an argument for God's
existence.

Returning to the objection, let us note that the design and non-design cases
have some differences.

One difference is that, for an intelligent Designer/Creator, what is being
created corresponds, one way or another, to a purpose.
To be consistent with
the approach used in this \paper{}, let us assume that the set of possible
purpose has the same cardinality as the set of real numbers, $\reale$.
We should expect a number of universes with
distinct descriptions for each purpose, but we should not expect that each
possible purpose is an actual purpose.
Given that intelligence and design
both imply choice, I think we should assume that, for each actual purpose,
there were many similar purposes that were discarded when the actual one
was choosen.

So then we could ask ourselves two things: how large is the set of actual
purposes, and whether one of them involves creating intelligent beings.

I think it's hard to estimate the likelyhood of any of these, but, for the
purpose of this \paper{}, it's probably good enough to show that, by default,
we should assign a non-zero probability to both.

First, let us note that, for real numbers, the similarity of two numbers
is usually linked to the distance between them
(i.e. it's the inverse of the distance).
So, if we are given a
non-zero distance (i.e. a non-infinite similarity) between real numbers,
and we want to pick a subset of real numbers such that no two numbers are
too similar, then that subset is at most countable.

It is not obvious that
similarity between purposes works in the same way as similarity between
real numbers, but I think there is a non-zero chance
that if Designer does not choose purposes which are too similar,
then their set is at most countable.
Arguably, that chance is fairly high.

Given a countable set of options, we can (and should) assign by default
a non-zero probability to all options.

Second, let us take a look at the probability that a Designer would want to
create intelligent beings.

Obviously, there are several purposes that involve
\ghilimele{intelligent beings}.
There are two reasons for not choosing one
of them: either the Designer did choose one or more purposes which were
too similar, so there was no point in choosing intelligent beings,
or the Designer was not interested in any pupose that involved intelligent
beings, and also not interested in any similar purpose.

Let us assume that there is a similarity threshold such that no two actual
purposes are more similar than that.
For any such threshold, we can split all possible purposes into disjoint
buckets such that all purposes in the same bucket are similar beyond that
threshold. As mentioned above, there is a non-zero
probability that we can get a set of buckets that is at most countable,
so let us examine this case in more detail.

If this set is countable, then for any bucket there is a non-zero chance that
the Designer will pick a purpose from it, so there is a non-zero chance that
the Designer will pick from a bucket containing an
\ghilimele{intelligent beings} purpose.

There is a possible purpose where the designer wants intelligent beings
for their own sake, which I will call
\ghilimele{the main intelligent beings purpose}.
If we make the similarity threshold
high enough, then there's a good chance that the bucket containing
this purpose, will contain only purposes involving intelligent beings.
If that's the case, then we
have a non-zero probability that the set of buckets is countable, and a
non-zero chance of choosing the bucket with the main intelligent beings purpose,
so we have a non-zero overall chance of having a universe designed for
intelligent beings.

Even if we can't get a bucket containing only intelligent beings purposes,
if the threshold is high enough, there's a (low, but non-zero) chance that
any purpose in an intelligent beings bucket
(todo: use more reasonable terms or define them)
will be similar enough that it will result in a universe similar to one
designed for intelligent beings.


\svn{Second, a good question is whether, in a designed world, intelligent beings
would be able to identify this design by looking at their world.
TODO: citation to whoever talked about this previously.
If the intelligent beings would see many things that they design, they
may be able to identify something as being designed by them.
However, when they observe their world, they only see one thing that may or
may not be designed, so they may not have enough experience to identify designed
worlds.
However, if the Designer's purpose was to create intelligent beings,
i.e. they are not just an accident, then there's some chance that the
things they design are analogous enough with some of the things the Designer
designs that they would be able to figure that out. TODO: rephrase.
}

\svn{
  Did I say this somewhere else?

  If the Designer intends to create intelligent beings, then the Designer will
create a universe for them.
As argued in \cite{Swinburne2004},
it's very likely that this universe is, indeed, consistent on a large scale.
But if there is a non-zero probability of observing
consistency around us, then there is a non-zero probability of observing
an universe like ours.
}

\svn{
  TODO: Why does non-zero consistency imply non-zero probability for our universe?
}

\subsection{Few Universes Exist}
\label{sec:fewuniverses}

Another possible objection is: we used a meta-universe definition that's too
restrictive. What would happen if, say, we had a meta-universe that, instead
of providing a probability distribution over universes, simply restricts the
possible universe descriptions to a set with cardinality less than $\reale$,
preventing us from applying axiom \ref{ax:zeroisgeneric}?

Note that the current argument works with conceivable universes, so the issue
is not which universes exist, but which could exist.

As a parenthesis, let us note that using conceivable universes makes sense.
If we are
trying to make a prediction about how a non-designed universe would
look like, we would have to imagine that we are outside any universe, and we are
about to observe one that is (say) non-designed. What could we say about it?
Not much, maybe that it makes logical sense (i.e. that it can be modelled
mathematically), maybe it has something close to time and space, but,
besides that, anything is possible.
I.e. we could say that it can be any
of the conceivable universes, or, perhaps,
that all conceivable universes are valid candidates.

That being said, the first possible answer to this section's issue
is that the main argument of this \paper{} can be reused here,
perhaps by slightly changing this \paper{}'s
axioms.
If we assume the continuum hypothesis, the set of descriptions mentioned
above can be a countable one or a finite one, both of which are peculiar,
so we can safely assume that this is not the case.
Without the continuum hypothesis, the cardinality of the set of descriptions
can also be somewhere between countable and the cardinality of the real numbers.
If so, then, by default, we should still use continuous
probability distributions over the set of descriptions.
If that does not seem good enough, then
the set of discontinuities would still be at most countable and,
following the same line of reasoning as the main argument, the discontinuities
should be generic.
Peculiar non-continuous events would require some explanation,
which can be provided by a Designer.

This means that, if we replace the set of real numbers by a set whose
cardinality is above countable, but at most that of the real numbers
in axioms \ref{ax:zeroisgeneric}, \ref{ax:noprobability} and
\ref{ax:uncountable}, the argument should still work.

The second possible answer is that, if there is no Designer, we should actually
expect the set of
possible descriptions for the meta-universe to have the same cardinal as
$\reale$, and that's true even
if we ask these descriptions to be \ghilimele{different enough},
i.e. different even though we work with approximations.
To see why, let us consider the following argument:

The set of all conceivable descriptions has the same cardinality as the
set of real numbers, $\reale$. This means that the
set of all sets of conceivable descriptions
has the same cardinality as the power set of $\reale$, i.e. $2^\reale$.

A claim about what is possible specifies a particular set of descriptions,
and any set of descriptions can be such a claim,
which means that the set of possible claims has the same cardinal as
the power set of $\reale$, $2^\reale$.
The cardinal numbers remain the same even if we take into account only claims
according to which our universe is possible.

Unlike in the main topic discussed in this
\paper{}, this time it's clear that only one of these claims can be true.

%The $2^\reale$ cardinal for possible claims is larger than $\reale$,
%so even more than when
%working probabilities over $\reale$, we should
%use a continuous probability distribution, i.e. the only reasonable
%probability that we can assign to any concrete claim is $0$. We will not discuss
%other cases.

Let us take all the claims in which the set of possible descriptions
is less than $\reale$.
The set of these claims (let us denote it by $S$)
has a cardinality which is less than $2^\reale$.
The set of claims not in $S$ (let us denote it by $T$) has the same cardinality
as $2^\reale$. But $T$ is infinitely larger\footnote{
  $S$ is infinite, which means that, if we take a distinct copy of $S$ for each
  element of $S$, and we put them together, we get a set that still has the same
  cardinal as $S$ (i.e. $S\times S$ has the same cardinal as $S$). In order to
  get the same cardinal as $T$ we need to take even more distinct copies of $S$,
  which means that $T$ is infinitely larger than $S$.

  How many copies? We actually need one for each element in $T$, and no lower
  cardinal would do.
} than $S$, which means that it's reasonable to assume that $S$ has probability $0$.
If so, then, with probability $1$, the set of possible descriptions has the same
cardinality as $\reale$.

We should be able to get the same results if, instead of looking at the set
of all distinct descriptions, we group descriptions based on their similarity
and we work with the set of these groups of descriptions.
As argued in section \ref{sec:not-enough-descriptions},
there are $\reale$ conceivable descriptions which are different
enough, so each of these would be in a different group, which means that
there are $\reale$ groups of conceivable descriptions.
Using the same reasoning as above, with probability $1$, the set of groups of
conceivable descriptions which are possible according to the meta-universe
rules has the same cardinality as $\reale$.
From these groups, we can extract a similar cardinal of
possible descriptions, which should be different enough since we grouped them
based on similarity.

\subsection{Not Enough Descriptions}
\label{sec:not-enough-descriptions}

When working without approximations, there are $\reale$ possible descriptions
that are essentially different. However, this is less obvious when working with
approximations, especially since that means we can probably have incompatible
descriptions for the same universe.

Let us assume (as an axiom) that, for a given level of approximation $L$, we can
find a finite cuboid size for which there is a finite set of
(full-precision)
descriptions, let us call it $D_f$, which cannot all be approximated with a
single finite approximate description.
In other words, if we have a set of cuboids whose descriptions include
the $D_f$ set, we can't
find a single finite approximate description that works for all of them
within the level of approximation $L$.

In the following, the difference between
cuboid descriptions and universe descriptions is not always explicit,
but can be inferred from the context.

Here is an incomplete example that shows why the above axiom is reasonable.
First, let us assume that we have a set of
\ghilimele{primary measurements} that we
can do for our model, such that any other measurement that we could do can be
computed from those.
Then let us take two descriptions, the first saying
\ghilimele{the value of each primary measurement is $0$},
and the second saying
\ghilimele{the value of each primary measurement oscillates quickly
  between $0$ and something large enough to be easily detected within
  the approximation level}.

Then, let us consider a hypothetical infinite universe whose timespace is $R^n$.
Let us divide it in cuboids of the size mentioned above
and let us assign the incompatible
descriptions in the set mentioned above to these cuboids in a roughly even
way.
The set of cuboids is countable, so there are $\reale$ ways of
assigning these descriptions.

Let us pick an assignment and let us pick a timespace point very close to the
time end of a space-time cuboid boundary.
Let us take that entire point's past and let us try to make a prediction based
on that.
First, based on just that past, we can't predict the type of the next cuboid,
since there will be many possible assignments that have the same past, but in
which the next cuboid will be different from the current assignment.

Still, a universe description must predict the future from the past, and
the future depends only on the next cuboid type. Then, the universe
description must allow predicting the next cuboid type, so it must include,
implicitly or explicitly,
a function predicting the next cuboid from a cuboid's past
and its context (i.e. position). Now we must find out how many such functions
we need in order to describe all possible assignments.

To make things simpler, let us assume that the number of dimensions, $n$,
is $1$ and that we have only two cuboid types.
It should be obvious that
the reminder of this example can be generalized to any number of dimensions
and to any number of cuboid types greater than $2$.

Having $n=1$, means that we have a one-to-one correspondence between
integers and cuboids, so let us identify a cuboid with its number.
Let us also identify the two cuboid types with $\multime{0, 1}$ and
let us represent a universe as a function from the set of integers, $\intregi$,
to $\multime{0, 1}$, identifying, for each cuboid, its type.

We can represent the information available for predicting the type of a future
cuboid by a pair between the cuboid and its past, let us denote it
by `(x, f)` where $x$ is the cuboid's number, $0$ or $1$,
and $f$, which represents $x$'s past, is a function from the set of natural
numbers to $\multime{0,1}$.

However, we are not trying to predict just one cuboid's future,
we are trying to predict the future of all cuboids.
Any way of making these predictions identifies with a function from the past
of cuboids to their future, i.e. from $(x, f)$ pairs to $\multime{0, 1}$.

Let us find out how many distinct functions we need.
To make things simple, let us consider only universes which, for negative
integers, have only cuboids of type $0$.

The past of cuboid $0$ is perfectly identical in all these universes, so,
in order to predict the type of the cuboid with index $0$,
we need two distinct prediction functions.
In order to predict the types of cuboids $0$ and $1$ we need four
distinct prediction functions.
In general, in order to predict the types of all cuboids between $0$ and $n$
we need $2^{n+1}$ distinct prediction functions.
And, in order to predict the type of all cuboids which are greater or equal
to $0$, we need $2^\naturale$, i.e. the power set of the natural numbers,
i.e. $\reale$ prediction functions.

As mentioned previously, intelligent beings may need some consistent space-time
around them. Even if that's the case, fixing the cuboid type for a finite
chunk of an infinite universe does not change the the cardinal for the set of
functions mentioned above.

\subsection {Multiple Universes Based on the Same Laws}

As mentioned above, our universe could be included in a meta-universe in which,
colloquially speaking, all universes have the same laws,
but different fundamental constants, and no other universes are allowed.
Of course, using this paper's definitions,
the fundamental constants would be part of a universe's description.

When using approximate universe descriptions, since our universe
seems to be based on laws that are continuous in these fundamental constants,
the set constants that are "different enough" should be at most countable.

Since no other universe is allowed, and the set of allowed approximate
descriptions is countable, the meta-universe's probability
has to be non-zero for some of these descriptions. In other words, the
meta-universe probability has some peculiar discontinuities, which, as
mentioned above, is implausible.

%\subsection{Fractions of Space}

%Note that above I used a specific definition of what I mean by
%predictions being \ghilimele{true with probability $p>0$}.
%However, it's valid to ask, say,
%what happens if we want our predictions to be true
%in only a fraction $p$ of spacetime, with probability $p$.

%Since what we mean by \ghilimele{a fraction $p$} is not fully defined,
%there may be definitions for which the argument below does not work,
%but, most likely, it works for all reasonable definitions.
%Also, the argument below presents only the idea, without working out all
%details, since there is no point of being fully precise without a precise
%hypothesis.

%Let us take, again, $\reale^n$ as a hypothetical universe.
%Let us denote by $n$ the number of chunks we get if we split the entire
%spacetime
%in chunks covering a fraction $p$, one of which may be smaller than $p$.
%As above, we can probably find a cuboid size for which we
%have $2n$ different cuboid descriptions when working with a given level
%of approximation $L$.
%Let us consider again that
%we split a universe into cuboids of this size, and that we assign
%cuboid descriptions randomly.

%We are interested in cuboid type assignments that assign the cuboid types
%such that each of them covers a fraction less than $p/2$ of the universe.

%Let us take a universe chunk covering a fraction $p$.
%We can choose 
% If that that includes at most one
%partial cuboid, then, with a single cuboid description we can cover at
%most half of it. 

%We'll further assume that when we divide the spacetime in chunks of the same
%size (cuboids are a good canditate),
%we can get at least two assignments (i.e. two assignment types)
%for the cuboids inside those
%chunks such that the two assignments do not assign the same value to the
%same cuboid in both.

%Now, any description which works in a fraction $p$ of space must make
%predictions that, roughly, work in at least two cuboid types.
%So, in order to make predictions, a definition must identify
%the cuboid type, as in section \ref{sec:fewuniverses}.
%This means that we must have at least
%$\reale$ definitions.

%\svn{Is \ghilimele{true} appropriate here? and in the entire paper?
%Truth is a word that's hard to use.
%Would it be better to use \ghilimele{verified} or something similar?}

\subsection{Complicated universe}

Above we have argued that we can divide the universe region around us in small
identical pieces, all of which have finite approximate descriptions, and
a description that works for one of them works for all of them.

Of course, in practice, we can only work with finite descriptions when
modelling our universe.
Fortunately, quantum mechanics, together with
relativity (although they may be incompatible),
seem to fit our universe almost perfectly.
Even Newtonian physics seems to be pretty good.
However, we can ask ourselves whether our timespace
region is actually much more complex, perhaps infinitely complex,
but we didn't manage to notice that yet.

This is possible, but the essential part of this paper's argument relies on
the timespace around us being consistent.
As long as it is conceivable that
a small region of space around us can have two incompatible descriptions,
then it becomes unreasonable to assign probability $1$ to any description, which
means that we are observing something having a really low probability.

\section{Conclusion}
\label{sec:conclusion}

We have the following reasonable possibilities:

\begin{enumerate}
\item The probability that a Designer would want intelligent beings
      is non-zero (see section \ref{sec:unknown-designer-intent}),
      our universe is designed (at least in part) for us,
      and the universe around us is, roughly, as homogenous as we would
      expect it to be.
\item Section \ref{sec:unknown-designer-intent} does not make sense,
      we don't know whether our universe is designed or not,
      our universe is not homogenous, but we observe some
      extremely unlikely homogeneity around us.
\end{enumerate}

The most reasonable option seems to be that our universe is designed.

\section{Mathematical Background}
\label{sec:background}

This section presents the mathematical results used in this paper. They are
usually presented without proofs or references, but most introductory courses
on the topic of each subsection should cover them.

This section also contains precise definitions for some of the concepts that
are specific to this paper
(namely \ghilimele{generic} and \ghilimele{peculiar}).

\subsection{Probabilities}
\label{sec:probabilities}

Let $A$ be a set with the same cardinality as the set of real numbers, $\reale$.
Let $F$ be the set of
all mathematical predicates of one variable over $A$ that can be written as a
finite formula.
If $f$ is such a predicate then let $A_f$ be the subset of
$A$ where $f$ is true, i.e. $A_f=\multime{a\in A\mid f(a) \mbox{ is true}}$.
Given a probability distribution over $A$, we define $P(f)=P(A_f)$.

Let $F_0$ be the set of predicates in $F$ whose probability is
$0$ for all continuous probability distributions over $A$, i.e.
$$F_0=\multime{
  f\in F
  \mid P(A_f)=0
    \mbox{ for all continuous probability distributions over } A}.
$$
As an example, any predicate $f$ which is true for a finite subset of elements,
i.e. $A_f$ is finite, would belong to $F_0$.

Let us identify by $A_0$ the
set of elements of $A$ for which at least one predicate of $F_0$ is true, i.e.
$$A_0=\multime{a \in A\mid \exists f\in F_0 \mbox{ with } f(a)\mbox{ true} }.$$
Let us also denote by $F_1$ and $A_1$ the complements of $F_0$ and $A_0$,
respectively.

Let $P$ be a continuous probability distribution.

Since $F$ is countable, $F_0$ must be at most countable. Then we can compute
$P(F_0)$ as the sum of the probability of $F_0$'s elements,
so obviously, $P(F_0)=0$,
which means that the probability of its complement, $P(F_1)$, is $1$.
From the way we defined $P(f)$, $P(F_0) = P(A_0)$, so $P(A_0)$ must be $0$
and $P(A_1)$ must be $1$, which justifies the indexes used for
these.

Let us say that an element $a\in A_1$ is \definitie{generic}
and an element $a\in A_0$ is \definitie{peculiar}. Then we could rewrite
the equalities above to $P(x\mbox{ is generic}) = 1$ and
$P(x\mbox{ is peculiar}) = 0$.

When the probability for a subset of $A$ is $0$, $P(E)=0$, where $E\subset A$,
it does not mean that observing an element of $E$ is logically impossible,
it means that,
if we make a set of (independent) observations that is at most countable
of elements of $A$, we have no chance at all of observing an element of $E$.

Let us now consider a different issue.
Let $P$ be a probability over the set of real numbers, $\reale$ (or
any infinite uncountable set).
Let $A$ be the set of real numbers with non-zero probability.
We will show that $A$ is at most countable.

Indeed, let us take all subsets of $A$ that are at most countable. The
probability of any such subset is between $0$ and $1$, so we can take
$S$ to be the supremum of the probabilty of these subsets, and
$S$ will also be between $0$ and $1$.

There is a sequence $(C_n)$ of at most countable sets such that each $C_n$
includes all the preceding ones, and $P(C_n)$ converges to the supremum, $S$.
But the union of all $C_n$ (let us call it $C$) is also countable,
and its probability is larger or equal than the probability of any $C_n$,
so the union's probability is exactly $S$.

If there would be a number $x$ of non-zero probability that is not in the
union $C$, then the union of $x$ and $C$ would still be countable, and
would have a larger probability than $S$, which is a contradiction.
This means that $C$, which is countable, contains all numbers with non-zero
probability.

\subsection{Ordinals}
\label{sec:ordinals}

Ordinals are generalizations of natural numbers. Natural numbers can be
defined by identifying each natural number with the set of natural numbers less
than it. Then $0$ is identified with the empty set $\emptyset$,
$1$ with $\multime{0}=\multime{\emptyset}$,
$2$ with $\multime{0, 1}=\multime{\emptyset, \multime{\emptyset}}$,
and so on.

Each of these natural numbers is an ordinal. To define the smallest ordinal
that is not a natural number, denoted by $\omega$, we will use the
same rule: let $\omega$ to be the set of ordinals smaller than it, i.e
the set of natural numbers, $\omega=\multime{0, 1, 2, \dots}$.

Of course, the next ordinal, called $\omega + 1$, will be the set
$\multime{0, 1, 2, \dots \omega}$ and the next one,
$\omega+2$, will be $\multime{0, 1, 2, \dots \omega, \omega+1}$. We can
continue and, in the same way, define $\omega\cdot 2=\omega+\omega$ to be
the ordinal that comes after all ordinals of the form $\omega+n$
where $n$ is a finite ordinal (i.e a natural number).

Then we can define $\omega\cdot 3$, $\omega\cdot 4$ and so on, and we can
take $\omega\cdot \omega$ to be the ordinal that comes after all the ones
defined by using the above rules.

Let us note that $\omega$ is countable, and that all the ordinals mentioned
above that come after it are also countable. By using the same kind of
reasoning as above we can produce other countable ordinals like
$\omega^\omega$ (from ordinals like $\omega\cdot\omega\cdot\dots\cdot\omega$)
and $\epsilon_0$ (from $\omega^{\omega^{\cdots^\omega}}$).

After going through many similar processes, at some point we obtain the
smallest uncountable ordinal, $\omega_1$, which is the set of all
countable ordinals.

Let us note that some ordinals, like all the finite ones except $0$,
and like $\omega+1$, can be obtained from the previous one by using
a succesor relation, i.e. $succesor(\alpha) = \alpha\cup\multime{\alpha}$.
All ordinals have a succesor, but not all are succesors, some, like
$\omega$ and $\omega\cdot 2$ can be defined only as the set of all
smaller ordinals. The former are called \ghilimele{successor ordinals},
the later are called \ghilimele{limit ordinals}. Note that $0$ is a limit
ordinal.

\subsection{Cardinals}
\label{sec:cardinals}

The set of natural numbers $\naturale$, the set of integers $\intregi$ and
the set of rational numbers $\rationale$ are countable.

The set of real numbers $\reale$ has a larger cardinality, the same cardinality
as the powerset of a countable set. Another way to put this is that
if we have a countable set, and, for each element, we can choose one of two
options,
and we take all ways of choosing options for the entire countable set,
then we get a set with the same cardinal as the set of real numbers\footnote{
    This is also valid for more than two options, as long as the cardinality
    of their set is at most countable.
  }.
That is, the set of all functions
$$
f : \naturale \longrightarrow \multime{1, 2}
$$
has the same cardinality as $\reale$.

If we have a set with the same cardinality as the set of real numbers, and
we remove a countable set, the resulting set still has the same cardinality
as the set of real numbers. As an example, $\reale$, $\reale\setminus\naturale$,
$\reale\setminus\intregi$ and $\reale\setminus\rationale$ all have the same
cardinality. In general, if $A$ and $B$ are infinite sets and
$B$'s cardinality is lower than $A$, then removing $B$ from $A$ does not change
its cardinal, i.e. $A\setminus B$ has the same cardinal as $A$.

If $A$ has an infinite cardinal, then the product set $A\times A$ has the same
cardinality as $A$.

\printbibliography

\end{document}
