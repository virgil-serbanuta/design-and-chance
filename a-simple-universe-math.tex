\documentclass[a4paper
,draft
]{article}

\usepackage{amsmath}
\usepackage{amsthm}
\usepackage{appendix}
\usepackage[english]{babel}

 %%%%%%%%%%%%%%%%%%v
\usepackage{combelow}
\usepackage{hyperref}
\usepackage[utf8]{inputenc}
\usepackage{newunicodechar}

\usepackage[
    backend=biber,
    style=apa
%    citestyle=authoryear,
%    citestyle=alphabetic,
%    sortcites=true,
%    style=authoryear
%    style=alphabetic
    ]{biblatex}

\DeclareLanguageMapping{english}{english-apa}
\addbibresource{design-argument.bib}

\newunicodechar{Ș}{\cb{S}}
\newunicodechar{ș}{\cb{s}}
\newunicodechar{Ț}{\cb{T}}
\newunicodechar{ț}{\cb{t}}

\title{A Simple Universe Argument}
\author{Virgil Șerbănuță\thanks{\href{mailto:design-and-chance@poarta.org}{design-and-chance@poarta.org}}}
%\date{June 2015}
 %%%%%%%%%%%%%%%%%%^


\usepackage{amsfonts}
\usepackage[obeyDraft]{todonotes}
\newcommand{\svn}[2][]{\todo[author=Virgil,color=red!25!white,#1]{#2}}
\newcommand{\tsf}[2][]{\todo[author=Traian,color=green!40!white,#1]{#2}}
\newcommand{\tsfgata}[2][]{\todo[author=Traian,color=blue!40!white,#1]{DONE - #2}}
\newcommand{\commentfootnote}[1][]{}

\def\infordinala{\omega}
\def\infordinalb{\omega_1}
\def\reale{\mathbb{R}}
\def\intregi{\mathbb{Z}}
\def\complexe{\mathbb{C}}
\def\naturale{\mathbb{N}}
\def\rationale{\mathbb{Q}}
\def\descriptions{D_L}
\def\designer{\mathbb{D}}
\def\our_description{OURD}
\newcommand{\paper}[1]{paper}
\newcommand{\multime}[1]{\left\{ #1 \right\}}
\newcommand{\definitie}[1]{\textbf{#1}}
\newcommand{\ghilimele}[1]{``#1"}
\newcommand{\lnotat}[1]{\sim #1}
\newcommand{\negat}[1]{\sim #1}  % TODO: This is the same as \lnotat
\newcommand{\citare}[1]{(\cite{#1})} % TODO: Separate this in a version with parenthesis and one without.

\newtheorem{definition}{Definition}
\newtheorem{afirmatie}{Claim}
\newtheorem{notatie}{Notation}
%\newtheorem{theorem}{Theorem}[section]
\newtheorem{lemma}{Lemma}
\newtheorem{axiom}{Axiom}
\newtheorem{note}{Note}

\begin{document}

 %%%%%%%%%%%%%%%%%%
\maketitle
 %%%%%%%%%%%%%%%%%%

 \begin{abstract}
  This \paper{} argues that one can make a prediction from
  the hypothesis that our universe is not designed, i.e. that it has a high
  level of complexity of a certain kind, and that this complexity would be
  easily observable everywhere. However, this is not what we observe, which
  falsifies the hypothesis.
\end{abstract}

\section{Introduction}

Many people believe
that there are certain laws approximating our universe's behaviour fairly well,
that we can compute its age,
that we can make predictions about the distant future, and so on.
In a way, all of these are really surprising.
In the words of \textcite{Feynman2009}:
\begin{quote}
Incidentally, the fact that there are rules at all to be checked
is a kind of a miracle; that it is possible to find a rule,
like the inverse square law of gravitation, is some sort of miracle.
It is not understood at all, but it leads to the possibility of
prediction --- that means it tells you what you would expect in
an experiment you have not yet done.
\end{quote}

This \paper{} attempts to figure out statistically
why do we observe such a surprising universe and
what can we reasonably believe about it.

The mathematical part of this paper is rather small,
containing only a few simple properties about set cardinalities,
probabilities and ordinals.
I think that the non-mathematical ideas in this paper
have an intuitive appeal even without the mathematical ones,
but presenting them separately would make it less clear why certain
conclusions can be drawn.
Although, in my opinion, the non-mathematical ideas are also somewhat obvious,
I did not manage yet to find anyone drawing the same conclusions
in the same way.

My argument can be considered a statistical approach to Aquina's fifth way.
It tries to avoid the issues that other statistical approaches for showing
our universe's design (e.g. the fine-tuning argument) have.

Sections \ref{sec:ordered-universe} presents this paper's relation to
previos work.
Section \ref{sec:possible-universes} introduces universe approximations
and a few other related notions.
Section \ref{sec:axioms} introduces the argument's axioms,
while section \ref{sec:valid-options} uses them to make predictions
about our universe, then verifies if they match our observations.
Section \ref{sec:objections} presents some clarifications and possible
answers to various objections to this paper.
Section \ref{sec:background}
briefly presents some mathematical background for people interested in it.

\section{The Ordered Universe Argument}
\label{sec:ordered-universe}

%TODO: Urmatoarele trei alineate sunt nasoale.
The great Catholic theologian Thomas Aquinas,
in his \ghilimele{fifth way}, attempts to show God's existence from
the order of the universe, i.e. that almost all bodies, almost always
behave according to simple natural laws. %TODO: citation.
One can find a good exposition of this argument in
\citetitle{swinburne1968}%\citeauthor{swinburne1968} \citeyear{swinburne1968}
\parencite{swinburne1968}, but let us look at a few ideas which are
interesting in the context of this \paper{}.
%Metaphysics by Peter van Inwagen, from 2015, says that it's a powerful and sophisticated defense.
% 67 de citari https://scholar.google.ro/scholar?cites=18062876017816040453&as_sdt=2005&sciodt=1,5&hl=en

There are two types of order which we may consider, the spatial order
and the temporal order.
The former is the order that can be seen in (a part of) the universe
at a given moment in time, e.g. that planets, living bodies, and other things
are ordered.
The latter can be seen in the behaviour of things, including the laws of
nature, and it's the one that will be used in this \paper{}.

\citetitle{swinburne1968} addresses Hume's
objections \svn{Need to introduce them. * Re-read The Argument From Design}
against Aquinas' argument, but one in particular is interesting here:
the argument is based on analogy, which limits its strength.
I think that this \paper{}'s argument, although it builds on the same
foundation, needs only a fairly weak analogy
(see Section \ref{sec:design-probability}).

The following quote from \textcite{swinburne1968}, made when addressing
Hume's objection that the order which can be observed in this universe
is just an accident, makes a nice introduction for the argument
described in this \paper{}:
\begin{quote}
But if we say that it is chance that in 1960 matter is behaving in a
regular way, our claim becomes less and less plausible as we find that in
1961 and 1962 and so on it continues to behave in a regular way. An appeal
to chance to account for order becomes less and less plausible
the greater the order.
\end{quote}

% TODO: Maybe list some of Hume's issues with this argument.

\section{Possible Universes and Their Descriptions}
\label{sec:possible-universes}

If our universe is designed, then it's likely to be the way it is because
its Designer wanted it to have certain properties.
In order to understand why our universe works the way it does,
one would need to understand the intent of its Designer.
While that is interesting in itself, I will not try to pursue it here,
except for a few limited ideas.
% TODO: use here the \ghilimele{design implies non-continous probabilities idea}

For most of the remainder of this \paper{}, let us consider the other case.
Let us assume the hypothesis that our universe is not designed
and let us try to make a prediction based on it.
How would a non-designed universe look like?
Would it be similar to our universe?
Maybe an infinity of universes exist and ours is just one of many,
or maybe our universe it the only one that exists.
Even if ours is the only one, one could easily imagine that it worked
in a different way, e.g. maybe some constant like the speed of light would be
different, or maybe gravity would work differently.

There are people who claim that all logically possible universes exist,
either because they think that it simply makes sense, or because they want to
give a good account of modality, or for other reasons.
If that's the case, it seems, at first sight, that making predictions about
non-designed universes is rather hard.
However, this paper argues that there are certain things that can be said
regardless of how many universes exist.
The argument in this \paper{} does not assume that more than
one universe exists, but it should work either way.

A possible universe
could have exactly the same fundamental laws as ours, but with matter
organized differently.
It could have similar laws, but with different universal constants.
It could have different fundamental particles (or fields, or whatever the basic
building blocks of our universe are, assuming that there are any).
Or it could be completely different, i.e. different in all possible ways.

It could be that our logic and reasoning are universal instruments,
but it could also be that some of these possible universes are
beyond what our reasoning can grasp and others have properties
for which our logic is flawed.
Even if that's the case, let us see if we can say anything about
the possible universes that we could understand and could model in some way.

In the following, the \definitie{possible universes} term will denote
only the possible universes which we could model.
The \ghilimele{possible} term is used with the sense
\ghilimele{logical possibility}. When the distinction between logical
possibility and actual possibility is important, the
\definitie{conceivable universes} term may be used instead.

This notion of model is not precise enough.
Let us restrict the possible universes term even more,
to the possible universes that could be modelled mathematically,
even if that may leave out some of them.
This may seem too restrictive,
especially since this paper only needs universes which can be approximated
by mathematical models.
We are going to relax this when talking about approximations, but, for now,
let us consider only universes
which are modellable with sets of axioms that are at most countable.

Let us restrict the universes we are considering even further, to universes that
have something remotely resembling time and space, for which
\ghilimele{the state of the universe at a given moment in time}, or something
close, makes sense, and which can plausibly contain intelligent beings.
% And let us restrict again, to universes where there is no action at a
% distance.
Any such universe is, for the purpose of this paper, a conceivable universe.

To keep the exposition simple, in the following I will use
\ghilimele{the state of the universe at a given moment in time},
but one should replace it with one's favourite alternative concept, e.g.
with the past of a hyperplane whose points only have spacelike intervals between
them.

Let us define a \definitie{universe description} to be a
consistent mathematical theory that has
a set of axioms which is at most countable and which allows making
predictions about the future state of the universe given its state
at a certain moment in time. A \definitie{universe region description}
is something similar, but only for a given space-time region of a universe,
with extra axioms to take into account the border state when predicting.
In the best case, for a deterministic universe, there might exist
a description which allows one to correctly predict the entire future state
given the state at any moment in time, but a universe description
as defined here does not have to predict everything and,
even when it predicts something, it does not have to always be correct.

\ghilimele{Predicting}, as used above, would normally mean that one
starts from the theory and does some formal inferences and computations, having
the prediction as the result.
However, as \cite{Calude2013} shows, there are many
things that can't be proven this way.
We don't know if the state of the universe at a given moment in time is one of
those things, although we could restrict our descriptions to ones where this
is possible, at least up to a reasonable level.
Regardless, let us use a different meaning: a theory predicts something if
that something is true in all models of that theory.

Note that usually the data available for making predictions is dependent
on who is making the prediction. As an example, if we assume that
all predictions are about things that can be perceived, directly or indirectly,
then
each kind of intelligent beings (e.g. humans) will make predictions
about the universe projected through their senses. If a universe contains
multiple kinds of intelligent beings, with different kinds of
sense organs, then that universe may have descriptions which are
very different.
Of course, things that are not observable directly can sometimes be mapped
to things that are observable, but this may not be always true.

In order to handle this dependence on who observes the universe
in a reasonable way, in the reminder of this paper we will work with universes
that contain intelligent beings,
and all predictions will be relative to what these intelligent beings
could observe.
If there are multiple kinds of intelligent beings in the
universe, whenever we are talking about its description
we will assume that we picked one such kind.
I.e. although we
will talk about universes and their descriptions, we'll actually mean
(universe, intelligent-being-kind) pairs and their corresponding descriptions.

Next, let us try to specify how good an univese description
should be. First, predictions speak about the future, but expecting to
predict everything until the end of the universe (if any) may not be
reasonable. We may want to fix an amount of time $\Delta t$,
focusing on predictions about things that are at most
$\Delta t$ in the future. Second, we should't expect to
be able to describe everything with full precision, so we may want to
have a precision $\eta>0$ for all the values that are predicted.
Third, we shouldn't expect predictions to always be correct, so
we should require that they are true with probability $p>0$.
The exact meaning of \ghilimele{true with probability} here is left open.
Of course, we may add other similar constraints if needed.

Then let us say that an \definitie{approximate universe description} with a
\definitie{level of approximation} $L=(\eta>0,$ $p>0$ and $\Delta t>0)$
is a universe description which allows approximating the future
state of the universe with a precision $\eta$, with a probability
$p>0$ for a prediction to be correct ($p$ is the same for each location
where predictions are being made)
%\svn{Do I need the probability of the prediction to be correct
%     or do I need to cover a fraction p of the world? Covering a fraction
%     of the world might actually mean that a universe with intelligent
%     beings can only have finite descriptions: intelligent => the set of
%     possible descriptions for hypercubes of size $1$ is at most countable
%     => there is one which covers a fraction of the universe that is $>1$ =>
%     I can chose that to build my universe approximation. }
and for a limited amount of time $\Delta t$.
\svn{I must answer the following questions: exactly what does that
     probability of correct prediction mean? Is it an aggregate among
     all possible predictions? Does it mean that all of them need a p
     chance of being correct?
     All writable predictions are countable, so they all have a non-zero
     weighth, so the aggregate probability most likely can't be zero.
     Am I talking about provable predictions or about predictions
     which are true on models?}

There is a distinction that we should make.
When predicting (say) weather we can't make long-term precise predictions,
and this happens because weather is chaotic, that is, a small difference
in the start state can create large differences over time.
This would happen even if the universe would be deterministic
and we would know the laws of the universe perfectly, as long as we don't know
the full current state of the universe.
However, high precision predictions may be possible for a deterministic
universe if the full state
is taken into account and, as we mentioned, we assume that we know the full
state of the universe at a moment in time when making predictions.

For a given universe or region of a universe,
given a level of approximation, we will pick a canonical description
in the following way: Let $S$ be the set of descriptions which approximate
the universe with the given level of aproximation. If $S$ contains
at least one finite description, then we pick the shortest such
description as \ghilimele{the canonical description}, breaking ties by
using the lexicographic order. Otherwise, we simply say that the
universe (region) has an infinite description, and we will abuse the
terminology a bit by picking the entire set
$S$ as the canonical description (we could
also pick a random description from the set).
If the level of approximation is obvious from the context, we will call
this canonical description \definitie{the universe's description}
or the \definitie{universe region's description}.

One could also use a well-ordering on the real numbers to choose the
lowest description as the universe's description, but that would
complicate things without any benefit.

\section{Options for Our Universe}
\label{sec:options}

The reminder of this paper will analyze what we can reasonably believe about
the following issues:
\begin{itemize}
  \item Our universe is designed or not.
  \item Our universe has a finite or infinite description.
  \item Option 1: There is a meta-$\beta$ universe for each countable ordinal
        $\beta$ such that our universe is the meta-$0$ one and the meta-$\beta$
        universe includes, directly or not, all meta-$\alpha$ universes with
        $\alpha < \beta$.
        Option 2: there is an ordinal $\beta$, possibly $1$,
        where this stops being the case.
  \item The set of possible descriptions for a finite chunk of space-time
        that are also compatible with life has at least the cardinal
        of $\reale$, or a smaller one.
\end{itemize}

\section{Axioms}
\label{sec:axioms}

Throughout this paper we will implicitly use only separated probability
measures, i.e. they can measure singletons (single-element sets).

Below we will use two terms, \ghilimele{generic} and \ghilimele{peculiar},
which are defined precisely in section \ref{sec:probabilities}. Informally,
an object is peculiar if it satisfies a peculiar predicate, and a peculiar
predicate is one that has a zero probability for any continuous probability
distribution. As an example,
\ghilimele{has a finite number of digits} is a peculiar
predicate over real numbers, and any real number with a finite number
of digits, like $12.5$, is peculiar. An object or predicate is generic if
it is not peculiar.

\subsection{Observing Events}
\begin{axiom}
  \label{ax:zeroisgeneric}
  If $P$ is a probability over the set of real numbers
  (or a set with the same cardinal)\footnote{Readers of this paper should
  keep in mind that, in most cases throughout this paper, what is being said
  about the set of real numbers is similarly valid for any set with the same
  cardinality.},
  we observe $x\in\reale$, and $P(x)=0$, then $x$ is generic.
\end{axiom}

Note that here, and in all the axioms in this paper, it is not required that
$P$ is a probability over the
Borel algebra of $\reale$, although that is, in many cases, implicitly
assumed when talking about probabilities over $\reale$.

The set of events for which $P(y)$ is
greater than $0$ is at most countable (see section \ref{sec:probabilities}),
and, if we remove them from $\reale$, we
get a set with the same cardinality as $\reale$.
On this later set, the probability
of all peculiar events taken together is $0$, so there is no chance of us
observing one.
In other words, the probability of all generic events is $1$,
so we can be sure that we observed a generic event.

Of course, the (logical) possibility of observing a peculiar event still exists,
but, practically, we will not observe it as long as the set of our
observations is at most countable.

\begin{axiom}\label{ax:noprobability}
  If we observe $x\in\reale$, when we could have
  observed any real number, and there is no probability distribution that could
  describe how $x$ was chosen, then $x$ is generic.
\end{axiom}

Note that this axiom does not say that we do not know that probability
distribution, it says that there is no such probability distribution.
Anyone believing that this cannot happen should treat the cases where
this axiom applies as invalid.

Also note that this cannot happen when using subjective probabilities.

If there is nothing that could favor
peculiar numbers over generic ones, it's absurd to think that we could have
observed an element of such a tiny set among something infinitely larger.
Also, the case with a probability above suggests that this is the only
reasonable assumption in this case.

\subsection{$\reale^4$ Universe}

The universes we are interested in are universes that can be
modelled on top of $\reale^\alpha$, where $\alpha$ is finite,
or something close enough to that.

However, the following axiom is enough, and it allowes us to not define
what \ghilimele{close enough} means.

Let us define a \definitie{generalized rational number} as being either a
rational number, or one of $-\infty$ and $+\infty$.

\begin{axiom}\label{ax:rationalcovering}
  The set of (generalized) cuboids using the same dimensions as our space-time
  and having their corners' coordinates in
  $\rationale\cup\multime{-\infty, +\infty}$,
  is countable and covers our universe.
\end{axiom}

The set of cuboids above is finite if and only if our universe uses a finite
set of coordinates.
Also, a spacetime based on real numbers, i.e. included
in $\reale^\alpha$, will be included in the generalized hypercube having
its corners at plus or minus infinity.

The axiom above does not require the space-time to include the cuboids or
their corners.
If, say, our universe is based on the set of integers, i.e.
it would be included in $\intregi^\alpha$,
we could consider it as being included in $\rationale^\alpha$,
where we could check if it is included in one of the hypercubes mentioned
above.

The following definition could also be written as an axiom.

\begin{definition}\label{finitecuboid}
  A part of our universe is \definitie{finite} if
  it can be covered with a finite cuboid,
  having its corners' coordinates in $\rationale$.
\end{definition}

There are a lot of possible definitions for \ghilimele{finite} which are not
included in the above.
While this paper could probably be extended to also handle many of these,
most likely there is no point in doing so.
As an example, in $\reale^3$ one can define it as
\ghilimele{having a finite volume}, which would
mean that there are finite things that do not fit in finite cuboids.
When defining a level of approximation, we could allow
selecting a part of the thing we model, with volume less than a fixed
constant, in which we can ignore what happens.

\subsection{Neighborhood Modelling}

\begin{axiom}\label{ax:finiteneighborhood}
  There is a large compact time-space region of our universe which
  includes our solar
  system, and there is a level of approximation $L$ such that:
  \begin{enumerate}
    \item Any hypercube included in that region has a finite approximate
          description for the level $L$.
    \item A description for one of the hypercubes also works for all other
          hypercubes with the same size in the given time-space region.
  \end{enumerate}
\end{axiom}

Many people assume, implicitly or explicitly,
that this is true, and, even more, that a hypercube's description works
for the entire universe.
This is especially visible when, e.g., claiming
that the universe is around $14$ billion years old, that the sun will,
in some distant future, become a white dwarf, or that standard-candle supernovae
are not ilussions.
This means that they assume that our universe has a finite approximate
description or, at least, that our solar system/galaxy/observable part of the
universe has such a description.
\svn{
    Think about deterministic vs non-deterministic modelling.
    Think about overlapping approximate descriptions - I can't have
    equivalence classes because a description that works for
    a fraction of the universe can be equivalent to two incompatible
    descriptions for the other fraction. Does the explanation in
    \ghilimele{Logically possible universes} still work?

    See the \ghilimele{Not enough descriptions} section.
}

\subsection{Logically Possible Universes}

\begin{axiom}\label{uncountable}
  For any level of approximation $L$ above a certain minimum level (see below)
  there is a set $\descriptions$
  of universe descriptions such that the following are true:
  \begin{enumerate}
    \item $\descriptions$ has the same cardinality as $\reale$.
    \item For all descriptions $d$ in $\descriptions$
          there is at least one conceivable
          universe $U_d$ which
      \begin{enumerate}
        \item has a time-space or something similar enough;
        \item can plausibly contain intelligent beings that use mathematics;
        \item for the intelligent beings mentioned above and
              for the level of approximation $L$,
              $d$ is $U_d$'s description.
      \end{enumerate}
    \item If $d$ and $d'$ are descriptions from $\descriptions$,
          then $d$ does not work
          for $U_{d'}$, the universe corresponding to $d'$.
    \item $\descriptions$ contains a description for our universe.
  \end{enumerate}
  The same is true for universe regions, except that
  $\descriptions$ may have a lower
  cardinality.
\end{axiom}

The minimum level for which this is true is left unspecified, but we should
include some common-sense restrictions, e.g. the length precision is above
Planck's length. All levels of approximation used below will be above this
minimum level, even if this is not mentioned explicitly.

This axioms states that, for a given level of approximation,
there is a large set of conceivable universes, which in some narrow respects
are similar to ours, but which are, in general, wildly different.
Also, our universe belongs to this set.

To see why that is reasonable, let us first note that,
%in a similar way to axiom \ref{nonessentialhypercube},
most likely, we have an approximate description for the observable part
of our universe
given by classical mechanics, maybe with some additions.
Alternately, one could use a description based on, say, quantum field theory.

It may be that there is a description (possibly different from the one above),
that works for our entire universe.
If not, some parts of the following should
be adjusted to include this possibility, but the argument should stay
essentially the same.

Next, for almost any countable axiom system that still
has $R^n$ as a base, one could imagine an alternate universe
which, in the present, is exactly like ours inside (say) Mars' orbit,
but what is what is outside of Mars' orbit is described by that axiom set.

Some of these axiom sets would describe laws of nature which still allow life
to exist inside Mars' orbit, are similar enough
to ours to allow us to observe what happens outside of this orbit,
but different enough that
we would notice (e.g. gravity could work differently, depending on the region
of space in which one travels).

For any approximation level $L$, and any region that is not trivial
for $L$
(the meaning of \ghilimele{trivial} is left open, but, as an example,
if we use an approximation level that does not measure things smaller
than a size $l$, then the region must be significantly larger than $l$),
there are multiple possible descriptions that
are distinct for $L$, i.e. there is no possible region where both descriptions
would be valid within the level $L$.
By splitting an infinite timespace
into disjoint regions defined by a finite set of rational coordinates
and taking all possible ways of assigning descriptions to these regions
we get a set of universe descriptions with infinitely countable axioms
which we will denote by $\descriptions$.
Since the set of regions is countable, and we can assign at least two
descriptions to each region, we get a set of assignments with the same
cardinality as $\reale$.

This does not change if we fix a region that can sustain intelligent life
and we require that the other regions' descriptions are compatible with life.

Each of the conceivable universes
having one of these descriptions could contain intelligent beings that use
mathematics, and so on.

This likely means that the set of descriptions mentioned in the axiom,
$\descriptions$, has the cardinality of $\reale$. See section
\ref{sec:fewuniverses} for another take on this issue.

Also see, e.g., \parencite{Manson2003}, which suggests that something similar
might be happening in our universe.

\section{Valid Options for Our Universe}
\label{sec:valid-options}

This section will try to develop the axioms above in order to find out what
is reasonable to believe about the issues presented in
section \ref{sec:options}.

We will focus mostly on what happens when our universe is not designed
since in this case it is easier to make predictions about our universe
and to falsify them, but we will also take a look at created universes.

\textcite[][Section \ghilimele{Why a world with human bodies is unlikely
if there is no God}]{Swinburne2003} comes sort of close to the argument
presented here, but while \citeauthor{Swinburne2003}
argues that human bodies are unlikely, I am arguing that, in the context
of all possible universes that could have human-like beings,
our universe is extremely unlikely unless a Designer intended it, the most
obvious reason for that being that the Designer wanted to design for human
beings.
Since the existence of human bodies is not directly related to the subject
of this \paper{}, I will not discuss that section more
than it is strictly needed.

Note that  \citeauthor{Swinburne2003} says that individual sets of laws
have non-zero probability while I'm claiming that their probability is $0$.
It seems to me that \citeauthor{Swinburne2003} implicitly assumes that
such a set has a finite number of laws, while I am explicitly removing
that constraint, so both can be right within their contexts.

\subsection{Peculiar Descriptions and Meta-universes}
\label{fdaumu}

Let us assume denote our universe's description by
$\our_description$. Let us take $\descriptions$ be the set of descriptions from
axiom \ref{uncountable}, with $\our_description\in\descriptions$.
Let us also consider that, perhaps, our universe is contained
in a meta-universe, which is, perhaps, contained in
a meta-meta-universe, and so on. Since these meta-universes included in other
meta-universes resemble ordinals, we will label them with ordinal numbers,
starting with $0$ for the first meta-universe.

For the purpose of this paper, a meta-universe is something that contains our
universe and which influences, one way or another, which universes exist and
in which quantity. This influence will be represented as a probability
distribution over universe descriptions.

As an example, our universe could be one of the many universes in a
meta-universe that contains universes having the same laws as ours, but with
different fundamental constants.
The probability distribution mentioned above would be zero for all
plausible universes with different laws, and for the universes that have
the same set of laws, it would be equivalent to a
probability distribution over the fundamental constants.

Any meta-universe without a corresponding probability distribution could
probably be handled with an axiom similar to axioms \ref{ax:zeroisgeneric}
and \ref{ax:noprobability}, but, for simplicity, will be
treated as if no meta-universe exists.

Alternately, this probability could be a subjective one, i.e. it can measure
what we believe about what can exist.

The meta-universe probability distribution reflects both what could exist, and
in which proportion in the following way: we will assume that each universe's
description was generated with a given probability distribution, but, since
it will not make any difference in the argument below, we will make no
assumptions about how many universes were generated.

As an example, let us take a look at what happens if only one universe exists.
If there is absolutely no reason
for it existing and the other ones not existing, then no meta-universe which
includes it exists.
If, somehow, this universe is contained in a meta-universe which
enforces that only one universe exists, the probability distribution will
assign $1$ to this universe's description, and $0$ to everything else.
If other universes could exists, but, somehow, they don't, we will get a
probability distribution specifying what could exist, and we would be talking
about a meta-universe in which only one universe was generated.

Returning to associating meta-universes with ordinals, let us consider the case
when these universe ordinals stop at some point, i.e. when there are ordinals
which do not correspond to any meta-universe.
If the smallest
ordinal not associated with a universe is countable, let us denote it by
$\alpha$.
If the smallest such ordinal is not countable,
or if there is an ordinal for each
universe, then let $\alpha$ be the lowest uncountable ordinal. $\alpha$ will
be an upper limit for the ordinal that we will consider in the following.

If $\alpha = 0$, then there is no meta-universe containing ours, so there
is no probability distribution over universe descriptions,
which means that $\our_description$ is generic (axiom \ref{ax:noprobability}).
\svn{Either use the proper wording everywhere, or explain that,
     for easier writing I am treating $P(X)=1$ that as
     logical necessity, but the
     reader should keep in mind that the actual meaning is closer to
     \ghilimele{practically true}.}

If $\alpha > 0$ and the probability distribution over universe descriptions
given by the first meta-universe is continuous,
then the probability of all peculiar descriptions is $0$,
so the only reasonable conclusion is that our universe has a generic
approximate description (axiom \ref{ax:zeroisgeneric}).

This result should probably be good enough, and if you agree with
this, please skip to section \ref{sec:peculiarregions}.

However, one could also wonder what happens when this probability distribution
is not continuous. If the discontinuities are generic, then we can easily
show that, with probability $1$, our universe's description is generic.

However, if they are not generic, then we can ask ourselves
whether we have a probability
description for these discontinuities. Since the meta-universe containing our
universe may be contained in a meta-meta-universe, maybe the meta-meta-universe
can provide such a distribution. If the meta-meta-universe probability
distribution is continuous, then the discontinuities are generic. If not,
then we have to ask ourselves if a meta-meta-meta-universe can give us more
information.

In general, let us let us assume that for some ordinals $\delta$ between
$1$ and $\alpha$ (inclusive) there is a probability distribution
$p_\delta$ over the set of universe descriptions $\descriptions$, telling us
the probability that a given description is a discontinuity
for all lower ordinal probabilities, i.e. for all
$p_\gamma$ with $\gamma < \delta$.
Since $p_0$ is not defined by the above, we will take it to be the
probability distribution for universe descriptions.

So, $p_0$ will be the probability distribution over universe descriptions,
$p_1$ (if it exists) will be the probability distribution telling us
what's the probability that a given description is a discontinuity for $p_0$,
$p_2$ will tell us
what's the probability that a given description is a discontinuity for both
$p_0$ and $p_1$, and so on.

Let $\beta$ be the minimum ordinal (if any) which is lower than our limit
$\alpha$ and for which either there is no probability $p_\beta$,
or the probability of our universe's description is $0$, i.e.
$p_\beta(\our_description) = 0$.

If such $\beta$ ordinal exists, let us note that we can consider that
we made a countable number of observations on various
probability distributions, one for each ordinal/meta-universe, and that
each time we observed our universe's description.
Since one of them was made on a probability distribution
where either there is no probability distribution $p_\beta$,
or $p_\beta(\our_description)=0$, then $\our_description$ is
generic (axioms \ref{ax:noprobability} and \ref{ax:zeroisgeneric}).

If such a $\beta<\alpha$ does not exist, then all $p_\delta$
probability distributions exist, with $\delta$ going up to $\alpha$ and,
perhaps, including it. If, additionally, $\alpha$ is countable, then,
similar to the above reasoning, $\our_description$ is generic.

This means that, if our universe is not designed, we have two options
that might be reasonable: our universe's description, $\our_description$, is
generic, or ($\our_description$ is peculiar, $\alpha$ is uncountable and,
for all $\delta < \alpha$,
the probability distribution $p_\delta$ exists and, according to it,
$\our_description$'s probability is greater than $0$.

In other words, in order to claim that $\our_description$ is peculiar,
one needs to postulate the
existence of an uncountable chain of meta-universes, all of them favouring a
peculiar $\our_description$, which, by default,
is prohibitively unlikely for any of them.
But that's not all, since, although the
current argument does not work anymore when the chain of
meta-universes becomes uncountable, intuitively the peculiarness problem
still remains: why would it suddenly become reasonable to make
only peculiar observations if we make enough of them? A few, maybe, but all
of them? Normally, when we make
observations with a continuous probability distribution, which is the default,
we expect to observe only generic elements as long as we make a
countable number of observation, and only when the observation set
becomes uncountable we expect to, perhaps, also observe some peculiar elements.

From now on, I will assume that the possible objection in the preceding
paragraph is unreasonable, which means that, practically speaking,
either $\our_description$ is generic or our universe is designed.

Since we don't know the limits of our universe, let us consider next
what happens for universe regions.

\subsection{Peculiar Descriptions for Universe Regions}
\label{sec:peculiarregions}

Let us consider all generalized hypercubes whose corners' coordinates are
rational numbers or $+\infty$ or $-\infty$. From axiom
\ref{ax:rationalcovering}, their set is countable.

Let us examine whet happens when the set of descriptions for a hypercube that
are compatible with life in that hypercube (and, perhaps, in the universe
around it) has the same cardinal as $\reale$: we can apply the same argument
as in section \ref{fdaumu}, by considering it as some sort of universe, and the
enclosing universe as a meta-universe.

Then, since the set of all the generalized hypercubes mentioned above is
countable, all of their descriptions should be generic.

However, we usually believe that we can have finite approximate descriptions
for the observable part of our universe, or, at least, for a large part of if
(axiom \ref{ax:finiteneighborhood}).
This means that the only options
that have a chance of being plausible are that our universe is designed, or
that the set of descriptions compatible with life for a finite hypercube
has a cardinality smaller than $\reale$.

Let us focus on the latter case above.
Note that, from section \ref{fdaumu}, our universe has
a generic approximate description.
But we assumed that the region around us has a finite approximate description.
If we were to extend it to the entire universe, we would find a peculiar
description for our universe, which is a contradiction. This means that there
is at least one region in our universe which has a different description.

Let $A$ be the set of possible approximate descriptions for
a hypercube of size, say, $1\;second \times meter^3$.
As argued in the paragraph above,
unless our approximation level is extremely coarse,
$A$ will have multiple elements.
We will assume that we are working with a reasonable approximation level.
For any description $a\in A$, let $P(a)$ be the probability of encountering
a hypercube with $a$ as its description in our universe.

If $A$ is finite, we should, by default, pick the uniform probability
distribution on $A$, assigning equal probabilities to all elements of $A$.
However, $A$ can have an infinite cardinal,
so we have to consider more general probabilities.

In any case, since, as mentioned above, $A$ has multiple elements,
we can't reasonably expect to have one element with probability $1$.
Then let $p_1<1$ be the probability of the description that we use for the
hypercubes around us.
The probability of observing $n$
non-overlapping hypercubes
with this description without observing any other description is\footnote{
  This assumes independence between the hypercubes, but, given that our
  universe's description is a generic one, randomly chosen among all possible
  ones, this is a reasonable assumption.
}
$p_1^n$.

Even if $p_1$ is very close to $1$, $p_1^n$ converges quickly to 0.
As an example, if $p_1=0.9$ then observing $n=70$ consecutive hypercubes with
the same description is enough to make $p_1^n$ go below one to one thousand
odds, $n=140$ is enough to go below one to one million,
$n=210$ goes below one to one billion.

Then, if we compare the non-design hypothesis
with another one non-zero probability,
almost always the consistency of a small spacetime region around us is enough
to make the non-design hypothesis unlikely enough to disregard it.

Next, let us see hat happens if we assume the design hypothesis.

TODO: clean up and use this footnote\footnote{
  Let $ND$ be the hypothesis that our universe is not designed,
  $D$ be the hypothesis that it is designed, $\our_description$ be our
  universe's description,
  $our$ be the description for the region of space around us.
  Note that
  $P(our) \ge P(\our_description)=P(\our_description|D)\cdot P(D) > 0$.
  Then $P(D|our) = \frac{P(our|D)\cdot P(D)}{P(our)} > 0$.
  Similarly,
  $P(ND|our) = \frac{P(our|ND)\cdot P(ND)}{P(our)}
    = \frac{P(ND)}{N^k} \cdot \frac{1}{P(our)}$.
  To compare the two we have to compare
  $P(our|D)\cdot P(D)$ with $\frac{P(ND)}{N^k}$.
  } or delete.

\subsection{Design Probability}
\label{sec:design-probability}

Let us examine the hypothesis that our universe is designed.
Since, in this case, the way our world works would based
on the designer's intent,
it is no longer obvious that, say, continuous probability distributions
should be the default.
On the other hand, do we actually have better options?

Without any other information or assumption,
we can repeat the previous argument to draw the same conclusion:
the probability of observing our universe is vanishingly small.

However, if, somehow, when a designer is involved, the probability of having
a large consistent time-space region is non-trivial, then
the probability of observing our universe becomes more reasonable (how much
more reasonable depends on the consistency's probability).

We can separate this probability in two: the probability that a designer
would want rational beings\footnote{A designer could want a consistent universe
region with a finite
description without wanting intelligent beings, but this possibility is not
analyzed in this paper}, denoted by $p_r$, and
the probability that the universe region containing
those intelligent beings is consistent
given that it was designed for them.

I think it's safe to assume that the latter probability is positive,
and, probably, fairly high. See as an example this quote from
\textcite{Swinburne2003}, which argues that, if God exists,
there is a fairly good chance that humans can understand their universe:

\begin{quote}
  So, in order to have significant freedom and responsibility, humans need
  at any time to be situated in a \ghilimele{space} in which there is a
  region of basic control and perception, and a wider region into which
  we can extend our perception and control by learning which of our
  basic actions and perceptions have which more distant effects and causes
  when we are stationary, and by learning which of our basic actions cause
  movement into which part of the wider region.
  If we are to learn which of our basic actions done where have which
  more distant effects (including which ones move us into which parts
  of the wider region), and which distant events will have which basically
  perceptible effects, the spatial world must be governed by laws of nature.
  For only if there are such regularities will there be recipes for changing
  things and recipes for extending knowledge that creatures can learn and
  utilize.
  So humans need a spatial location in a law governed universe in which to
  exercise their capacities, and so there is an argument from our being thus
  situated to God.
\end{quote}

With that in mind, let us compare the probability that
we observe consistency under the \ghilimele{design for intelligent beings},
\ghilimele{design without intelligent beings in mind} and
\ghilimele{no design} hypotheses.

If the probability that a
designer would want rational beings, $p_r$, is $0$, then we can't
differentiate the design and non-design cases, and the
the probability of observing our world's consistency is the same
for both.

However, if $p_r$ is non-zero, then the
the probability of observing our world's consistency becomes larger in the
design case, since $p_r$ is providing some ammount of support for that
hypothesis.

Oversimplifying a bit, and assuming that the probability of having consistency
is large if the universe was designed for intelligent beings,
the ammount of support would
roughly be equal with the ratio between $p_r$ and the probability of observing
our world's consistency in the non-design case.

Since observing consistency without design has an extremely
small probability,
having almost any non-zero probability for
\ghilimele{a designer of worlds would want rational beings} would provide
overhelming support for the design hypothesis.
As an example, a probability
of $1$ in one billion billion would easily be large enough.

From a natural theology point of view, one may argue for a higher design
probability through analogy with our own intents: we would be interested
in intelligent beings, it seems to us that intelligent beings would be
interested in other intelligent beings, so maybe the Designer would be
interested in creating other intelligent beings.

How much is this analogy worth?
It's hard to tell, but, on the other hand, our options are:

\begin{itemize}
  \item Our universe is not designed,
        and the probability of observing the
        consistency that we see around us would be vanishingly small.
  \item Our universe is designed, but the above analogy does not work and
        peculiar descriptions still have a zero probability.
        Again, the probability of observing the
        consistency that we see around us would be vanishingly small.
  \item The probability that the above analogy works is not vanishingly small,
        and we are observing the consistency that we would expect from a
        universe designed for intelligent beings.
\end{itemize}
For completeness, the first two options are valid only if,
for a given approximation level, if we require that a universe has life then
the set of possible approximate descriptions for a finite hypercube
has a lower cardinality than $\reale$.

It seems to me that the only reasonable option is that the analogy is not as
bad as some people may think it is.

\svn{5. Use Swinburne's argument about why would God create human-like beings.}

\section{Objections and Clarifications}
\label{sec:objections}

This section includes various possbile objections to this argument. Since the
fine-tuning argument addresses the same problem, and it's also using a
probabilistic argument (though in a completely different way), some of the
objections below are similar to the fine-tuning ones, and it may be helpful
to compare the two approaches.

For the fine-tuning argument see, e.g., \parencite{sep-fine-tuning}.
For objections to the fine-tuning argument that are relevant here, see, e.g.,
\parencites{Manson2003}{Manson2009}{McGrew2001}{Narveson2003}{Sober2009}.
For possible answers see, e.g.
\textcites{Leslie2003}{Swinburne2003}{Monton2006}{Kotzen2012}.

%Manson2009 McGrew2001 - probabilities not applicable to cosmic parameter
%values.
%Manson2009 Sober2009 - observation selection effect
%Manson2009 - multiple universes
%Manson2003 - nonhomogenous spacetime, different fundamental constants
%Sober2009 Narveson2003 - unknown designer's intent, no difference to design

% TODO: Cite \ghilimele{Should we care about fine-tuning} by Jeffery Koperski.

\subsection{Observation Selection Effect and Multiple Universes}

In this \paper{}, we only look at universes that contain intelligent life,
and that restricts the set of possible universe descriptions.
Even more, we don't see a universe as it is, instead we see it
through the eyes of the intelligent beings inhabiting it.
It can be argued that, in a non-created universe, beings might be intelligent
only if their intelligence is useful to them.
But this likely means that those beings live in a timespace region which seems
consistent from their point of view, so maybe it's not that unlikely to see
consistency around us.

However, let us look more carefully at how much consistency we would expect.
There are possible universes with consistent regions in which
intelligent life can exist, and whose consistency ends abruptly at some
random time. There are possible universes whose consistent regions are
strictly the size needed for allowing intelligent life, and there are possible
universes with large consistent regions.

Let us assume that our existence means that some consistency is required.
Is there any non-required consistency around us, consistency which is due to
chance? To be more precise,
how large is the time-space region whose consistency is required? Well,
perhaps at very distant times in the past, the consistent region included
the entire observable universe, but, right now, there is no reason to require
full consisency outside of Earth's orbit. Even more, since non-consistency
only means
observable non-consistency, and does not require something wildly different,
it's likely that we don't even need full consistency inside of Earth's orbit.

However, as far as we know, our solar system is consistent, our galaxy is
fairly consistent, and distant galaxies are also fairly consistent.

This \paper{} argues that, since we do observe much more consistency
than we would expect, design is the right explanation.

However, there is a possible objection related to this: if multiple universes
exist, perhaps all possible ones, there will be some beings living in the
implausibly consistent ones.

While this is correct, the probability of an intelligent being
living in a fully consistent universe is still $0$.
Also, seeing a large consistent region of space is still very unlikely among
all existing universes.

In virtually all universes the nonhomogeneity of the universe would be
easily observable, meaning that, for the relatively few intelligent beings
living in the other universes, as long as the design hypothesis
has some plausibility, it would be unreasonable to think that
their universe is not designed (assuming that the argument presented in
this \paper{} is correct).

\subsection{Unknown Designer's Intent}

From a natural theology point of view, one can't
know what the Designer wanted
(e.g. one can't know that a universe designer would want to create
a universe having life) \parencites{Sober2009}{Narveson2003}, so, by default,
any argument showing that the probability of our universe is small
if it's not designed would also show that the probability is small
even if it is designed.
To fix this, one would need an independent way to show that the Designer
wanted the universe to have life \parencite{Sober2003}.

However, there is an asymmetry between the two cases: a Designer can plausibly
want peculiar things, or homogeneity.
Sacred texts of religions may allow us to give a non-zero probability to
the hypothesis that a Designer exists (although some would argue that it's
very small, surely it's much larger than the probability of living in
a rather large consistent region of a non-designed universe).
Also, we can use an analogy with
ourselves in order to figure out a Designer's intent.

In the last case, we would have to
choose between two extremely improbable options
(no design vs. design, but intelligent life is due to chance),
and a design option based on an analogy with, hopefully,
non-zero chances of working.
If the two improbable options mentioned above have equal probabilities, then
any positive chance that the analogy
works is enough to give the combined design hypothesis a larger probability.
Even if they don't have equal probabilities, almost any positive change
that the analogy works gives the design hypothesis an overhelmingly larger
probability.

\subsection{Few Universes Exist}
\label{sec:fewuniverses}

Another possible objection is: we used a meta-universe definition that's too
restrictive. What would happen if, say, we had a meta-universe that, instead
of providing a probability distribution over universes, simply restricts the
possible universe descriptions to a set with cardinality less than $\reale$,
preventing us from applying axiom \ref{ax:zeroisgeneric}?

Note that the current argument works with conceivable universes, so the issue
is not which universes exist, but which could exist.

As a parenthesis, let us note that using conceivable universes makes sense.
If we are
trying to make a prediction about how a non-designed universe would
look like, we would have to imagine that we are outside any universe, and we are
about to observe one that is (say) non-designed. What could we say about it?
Not much, maybe that it makes logical sense (i.e. that it can be modelled
mathematically), maybe it has something close to time and space, but
everything else is possible.
I.e. we could say that it can be any
of the conceivable universes, or, perhaps,
that all conceivable universes are valid candidates.

That being said, the first possible answer to this section's issue
is that the main argument of this \paper{} can be reused here,
perhaps by slightly changing this \paper{}'s
axioms.
If we assume the continuum hypothesis, the meta-universe
above restricts the set of descriptions to a countable one, which is peculiar,
so we can safely assume that this is not the case.
Without the continuum hypothesis, if the set of descriptions is somewhere
between countable and $\reale$, we can probably still expect continuous
probability distributions over the set of descriptions, or, at least,
that the set of discontinuities is at most countable and that peculiar
non-continuous events require some explanation, which can be provided by
a Designer.

The second possible answer is that we should actually expect the set of
possible descriptions to have the same cardinal as $\reale$,
and that's true even
if we ask these descriptions to be \ghilimele{different enough},
i.e. different even though we work with approximations.

Before going into that, it would also be interesting to ask
how many universes should we expect to exist if there is no designer. The
argument below will talk about possible descriptions,
but it works both when asking how many descriptions are possible, and
when asking what descriptions do the existing universes have.

As above, the hypothesis that the set of descriptions is finite or countable
is peculiar, so it can be discarded.
Assuming the continuum hypothesis, the only cardinality left is $\reale$.
However, there is an argument that does not use the continuum hypothesis:

The set of all conceivable descriptions has the same cardinality as $\reale$,
so the set of all sets of conceivable descriptions
has the same cardinality as the power set of $\reale$, i.e. $2^\reale$.
Since a claim about what is possible specifies a set of descriptions, and any
set of descriptions can be such a claim,
the set of possible claims has the same cardinal as $2^\reale$.
The cardinal numbers remain the same even if we take into account only claims
according to which our universe is possible.
Unlike in the main topic discussed in this
\paper{}, this time it's clear that only one of these claims can be true.

%The $2^\reale$ cardinal for possible claims is larger than $\reale$,
%so even more than when
%working probabilities over $\reale$, we should
%use a continuous probability distribution, i.e. the only reasonable
%probability that we can assign to any concrete claim is $0$. We will not discuss
%other cases.

Let us take all the claims in which the set of possible descriptions
is less than $\reale$.
The set of these claims has a cardinality which is less than $2^\reale$.
This means that it's reasonable to assume that it has probability $0$.
If so, then, with probability $1$, the set of possible descriptions has the same
cardinality as $\reale$.

We should be able to get the same results if, instead of looking at the set
of all distinct descriptions, we group descriptions based on their similarity
and we work with the set of these groups of descriptions.
As argued in section \ref{sec:not-enough-descriptions},
there are $\reale$ conceivable descriptions which are different
enough, so each of these would be in a different group, which means that
there are $\reale$ groups of conceivable descriptions.
Using the same reasoning as above, with probability $1$, the set of groups of
conceivable descriptions which are possible according to the meta-universe
rules has the same cardinality as $\reale$.
From these groups, we can extract a similar cardinal of
possible descriptions, which should be different enough since we grouped them
based on similarity.

\subsection{Not Enough Descriptions}
\label{sec:not-enough-descriptions}

When working without approximations, there are $\reale$ possible descriptions
that are essentially different. However, this is less obvious when working with
approximations, especially since that means we can probably have incompatible
descriptions for the same universe.

Let us assume (as an axiom) that, for a given level of approximation $L$, we can
find a finite hypercube size $l$ for which there is a finite set of
(full-precision)
descriptions $D_f$ which cannot all be handled with a single finite approximate
description. To be more precise,
for any set of hypercubes whose descriptions include the $D_f$ set, we can't
find a single finite approximate description that works for all of them
within the level of approximation $L$.

In the following, the difference between
hypercube descriptions and universe descriptions is not always explicit,
but can be inferred from the context.

Here is an incomplete example that shows why the above axiom is reasonable.
First, let us assume that we have a set of
\ghilimele{primary measurements} that we
can do for our model, such that any other measurement that we could do can be
computed from those.
Then let us take two descriptions, the first saying
\ghilimele{the value of each primary measurement is $0$},
and the second saying
\ghilimele{the value of each primary measurement oscillates quickly
  between $0$ and something large enough to be easily detected within
  the approximation level}.

Then, let us consider a hypothetical infinite universe whose timespace is $R^n$.
Let us divide it in hypercubes of size $l$ and let us assign the incompatible
descriptions in the set mentioned above to these hypercubes in a roughly even
way.
The set of hypercubes is countable, so there are $\reale$ ways of
assigning these descriptions.

Let us pick an assignment and let us pick a timespace point very close to the
time end of a space-time hypercube boundary.
Let us take that entire point's past and let us try to make a prediction based
on that.
First, based on just that past, we can't predict the type of the next hypercube,
since there will be many possible assignments that have the same past, but in
which the next hypercube will be different from the current assignment.

Still, a universe description must predict the future from the past, and
the future depends only on the next hypercube type. Then, the universe
description must allow predicting the next hypercube type, so it must include,
implicitly or explicitly,
a function predicting the next hypercube from a hypercube's past
and its context (i.e. position). Now we must find out how many such functions
we need in order to describe all possible assignments.

To make things simpler, let us assume that the number of dimensions, $n$,
is $1$ and that we have only two hypercube types.
It should be obvious that
the reminder of this example can be generalized to any number of dimensions
and to any number of hypercube types greater than $2$.

Having $n=1$, means that we have a bijection between integers and hypercubes,
so let us identify a hypercube with its number.
Let us also identify the two hypercube types with $\multime{0, 1}$ and
let us represent a universe as a function from $\intregi$ to $\multime{0, 1}$,
identifying, for each hypercube, its type.

We can represent the information available for predicting the type of a future
hypercube by a pair between the hypercube and its past, let us denote it
by `(x, f)` where $x$ is the hypercube's number, and $f$, which represents
$x$'s past, is a function from the set of natural numbers to $\multime{0,1}$.

However, we are not trying to predict just one hypercube's future,
we are trying to predict the future of all hypercubes.
Any way of making these predictions identifies with a function from the past
of hypercubes to their future, i.e. from $(x, f)$ pairs to $\multime{0, 1}$.

Let us find out how many distinct functions we need.
To make things simple, let us consider only universes which, for negative
integers, have only hypercubes of type $0$.

The past of hypercube $0$ is perfectly identical in all these universes, so,
in order to predict the type of the hypercube with index $0$,
we need two distinct prediction functions.
In order to predict the types of hypercubes $0$ and $1$ we need four
distinct hypercubes.
In general, in order to predict the types of hypercubes from $0$ to $n$
we need $2^{n+1}$ distinct prediction functions.
And, in order to predict the type of all hypercubes which are greater or equal
to $0$, we need $2^\naturale$ (the power set of the natural numbers),
i.e. $\reale$ prediction functions.

As mentioned previously, intelligent beings may need some consistent space-time
around them. Even if that's the case, fixing the hypercube type for a finite
chunk of an infinite universe does not change the the cardinal for the set of
functions mentioned above.

\subsection {Multiple Universes Based on the Same Laws}

As mentioned above, our universe could be included in a meta-universe in which
all universes have the same laws, but different fundamental constants, and no
other universes are allowed.
Of course, the fundamental constants would be part of a universe's description
in the sense mentioned above.

Since we are using approximate universe descriptions, and our universe
seems to be based on laws that are continuous in these fundamental constants,
then the set constants that are "different enough" should be at most countable.

Since no other universe is allowed, and the set of allowed approximate
descriptions is countable, the meta-universe's probability
has to be non-zero for some of these descriptions. In other words, the
meta-universe probability has some peculiar discontinuities, which, as
mentioned above, is implausible.

%\subsection{Fractions of Space}

%Note that above I used a specific definition of what I mean by
%predictions being \ghilimele{true with probability $p>0$}.
%However, it's valid to ask, say,
%what happens if we want our predictions to be true
%in only a fraction $p$ of spacetime, with probability $p$.

%Since what we mean by \ghilimele{a fraction $p$} is not fully defined,
%there may be definitions for which the argument below does not work,
%but, most likely, it works for all reasonable definitions.
%Also, the argument below presents only the idea, without working out all
%details, since there is no point of being fully precise without a precise
%hypothesis.

%Let us take, again, $\reale^n$ as a hypothetical universe.
%Let us denote by $n$ the number of chunks we get if we split the entire
%spacetime
%in chunks covering a fraction $p$, one of which may be smaller than $p$.
%As above, we can probably find a hypercube size for which we
%have $2n$ different hypercube descriptions when working with a given level
%of approximation $L$.
%Let us consider again that
%we split a universe into hypercubes of this size, and that we assign
%hypercube descriptions randomly.

%We are interested in hypercube type assignments that assign the hypercube types
%such that each of them covers a fraction less than $p/2$ of the universe.

%Let us take a universe chunk covering a fraction $p$.
%We can choose 
% If that that includes at most one
%partial hypercube, then, with a single hypercube description we can cover at
%most half of it. 

%We'll further assume that when we divide the spacetime in chunks of the same
%size (hypercubes are a good canditate),
%we can get at least two assignments (i.e. two assignment types)
%for the hypercubes inside those
%chunks such that the two assignments do not assign the same value to the
%same hypercube in both.

%Now, any description which works in a fraction $p$ of space must make
%predictions that, roughly, work in at least two hypercube types.
%So, in order to make predictions, a definition must identify
%the hypercube type, as in section \ref{sec:fewuniverses}.
%This means that we must have at least
%$\reale$ definitions.

%\svn{Is \ghilimele{true} appropriate here? and in the entire paper?
%Truth is a word that's hard to use.
%Would it be better to use \ghilimele{verified} or something similar?}

\section{Conclusion}

We have the following reasonable possibilities:

\begin{enumerate}
\item The Designer intent analogy makes some sense, our universe is designed,
      and the universe is, roughly, as homogenous as we would
      expect it to be (at least around us).
\item The analogy does not make sense, we don't know whether our universe is
      designed or not, our universe is not homogenous, but we observe some
      extremely unlikely homogeneity around us.
\end{enumerate}

The most reasonable option seems to be that our universe is designed.

\section{Background}
\label{sec:background}
\subsection{Probabilities}
\label{sec:probabilities}

Let $A$ be a set with the same cardinality as $\reale$. Let $F$ be the set of
all mathematical predicates of one variable over $A$ that can be written as a
finite formula. If $f$ is such a predicate then let $A_f$ be the subset of
$A$ where $f$ is true, i.e. $A_f=\multime{a\in A\mid f(a) \mbox{ is true}}$.
Given a probability distribution over $A$, we define $P(f)=P(A_f)$.

Let $F_0$ be the set of all elements of $F$ such that their probability is
$0$ for all continuous probability distributions over $A$, i.e.
$$F_0=\multime{
  f\in F
  \mid P(A_f)=0
    \mbox{ for all continuous probability distributions over } A}.
$$
As an example, any predicate which is true for a finite subset of elements,
i.e. $A_f$ is finite, would belong to $F_0$. Let us identify by $A_0$ the
set of elements of $A$ for which at least one predicate of $F_0$ is true, i.e.
$$A_0=\multime{a \in A\mid \exists f\in F_0 \mbox{ with } f(a)\mbox{ true} }.$$
Let us also denote by $F_1$ and $A_1$ the complements of $F_0$ and $A_0$,
respectively.

Let $P$ be a continuous probability distribution.

Since $F$ is countable, $F_0$ must be at most countable. Then,
obviously, $P(F_0)=0$, so the probability of its complement, $P(F_1)$, is $1$.
Similarly, $P(A_0) = 0$ and $P(A_1) = 1$, which justifies the indexes used for
these.

Let us say that an element $a\in A_1$ is \definitie{generic}
and an element $a\in A_0$ is \definitie{peculiar}. Then we could rewrite
the equalities above to $P(x\mbox{ is generic}) = 1$ and
$P(x\mbox{ is peculiar}) = 0$.
\svn{I should actually use the \ghilimele{is finite} predicate or something
     like that.}

Obviously, $P(E)=0$, where $E\subset A$, does not mean that observing
an element of $E$ is logically impossible, it just means that,
if we make a set of (independent) observations that is at most countable
of elements of $A$, we have no chance at all of observing an element of $E$.

Let us consider a probability $P$ over the set of real numbers, $\reale$ (or
a larger one). Let $A$ be the set of real numbers with non-zero probability.
We will show that $A$ is countable.

Indeed, let us take all subsets of $A$ that are at most countable. The
probability of any such subset is between $0$ and $1$, so we can take
$S$ to be the supremum of the probabilty of these subsets, and
$S$ will also be between $0$ and $1$.

There is a sequence $(C_n)$ of at most countable sets such that each $C_n$
includes all the preceding ones, and $P(C_n)$ converges to the supremum, $S$.
But the union of all $C_n$ (let us call it $C$) is also countable,
and its probability is larger or equal than the probability of any $C_n$,
so the union's probability is exactly $S$.

If there would be a number $x$ of non-zero probability that is not in the
union $C$, then the union of $x$ and $C$ would still be countable, and
would have a larger probability than $S$, which is a contradiction.
This means that $A$ is $C$, so $A$ is at most countable

\subsection{Ordinals}
\label{sec:ordinals}

Ordinals are generalizations of natural numbers. Natural numbers can be
defined by identifying each natural number with the set of natural numbers less
than it. Then $0$ is identified with the empty set $\emptyset$,
$1$ with $\multime{0}=\multime{\emptyset}$,
$2$ with $\multime{0, 1}=\multime{\emptyset, \multime{\emptyset}}$,
and so on.

Each of these natural numbers is an ordinal. To define the smallest ordinal
that is not a natural number, denoted by $\omega$, we will use the
same rule: let $\omega$ to be the set of ordinals smaller than it, i.e
the set of natural numbers, $\omega=\multime{0, 1, 2, \dots}$.

Of course, the next ordinal, called $\omega + 1$, will be the set
$\multime{0, 1, 2, \dots \omega}$ and the next one,
$\omega+2$, will be $\multime{0, 1, 2, \dots \omega, \omega+1}$. We can
continue and, in the same way, define $\omega\cdot 2=\omega+\omega$ to be
the ordinal that comes after all $\omega+n$ where n is a natural number
\svn{Use finite ordinal instead of natural number.}.

Then we can define $\omega\cdot 3$, $\omega\cdot 4$ and so on, and we can
take $\omega\cdot \omega$ to be the ordinal that comes after all the ones
defined by using the above rules.

Let us note that $\omega$ is countable, and that all the ordinals mentioned
above that come after it are also countable. By using the same kind of
reasoning as above we can produce other countable ordinals like
$\omega^\omega$ (from ordinals like $\omega\cdot\omega\cdot\dots\cdot\omega$)
and $\epsilon_0$ (from $\omega^{\omega^{\cdots^\omega}}$).

After going through many similar processes, at some point we obtain the
smallest uncountable ordinal, $\omega_1$, which is the set of all
countable ordinals.

Let us note that some ordinals, like all the finite ones except $0$,
and like $\omega+1$, can be obtained from the previous one by using
a succesor relation, i.e. $succesor(\alpha) = \alpha\cup\multime{\alpha}$.
All ordinals have a succesor, but not all are succesors, some, like
$\omega$ and $\omega\cdot 2$ can be defined only as the set of all
smaller ordinals. The former are called \ghilimele{successor ordinals},
the later are called \ghilimele{limit ordinals}. Note that $0$ is a limit
ordinal.

Transfinite induction is a generalization of induction, where, if we can
prove that a property holds for an ordinal $\alpha$ from the fact that
it holds for all ordinals $\beta<\alpha$, then this property holds for
all ordinals.
\svn{am I using this?}

In many cases transfinite induction proofs are done separately for
succesor ordinals (maybe in the form $p(\alpha)$ implies $p(\alpha+1)$)
and for limit ordinals.
\svn{am I using this?}

\printbibliography

\end{document}
