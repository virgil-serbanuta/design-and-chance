\documentclass[a4paper
,draft
]{article}

\usepackage{amsmath}
\usepackage{amsthm}
\usepackage{appendix}
\usepackage[english]{babel}

 %%%%%%%%%%%%%%%%%%v
\usepackage{combelow}
\usepackage{hyperref}
\usepackage[utf8]{inputenc}
\usepackage{newunicodechar}

\usepackage[
    backend=biber,
    style=apa
%    citestyle=authoryear,
%    citestyle=alphabetic,
%    sortcites=true,
%    style=authoryear
%    style=alphabetic
    ]{biblatex}

\DeclareLanguageMapping{english}{english-apa}
\addbibresource{design-argument.bib}

\newunicodechar{Ș}{\cb{S}}
\newunicodechar{ș}{\cb{s}}
\newunicodechar{Ț}{\cb{T}}
\newunicodechar{ț}{\cb{t}}

\title{A Simple Universe Argument}
\author{Virgil Șerbănuță\thanks{\href{mailto:design-and-chance@poarta.org}{design-and-chance@poarta.org}}}
%\date{June 2015}
 %%%%%%%%%%%%%%%%%%^


\usepackage{amsfonts}
\usepackage[obeyDraft]{todonotes}
\newcommand{\svn}[2][]{\todo[author=Virgil,color=red!25!white,#1]{#2}}
\newcommand{\tsf}[2][]{\todo[author=Traian,color=green!40!white,#1]{#2}}
\newcommand{\tsfgata}[2][]{\todo[author=Traian,color=blue!40!white,#1]{DONE - #2}}
\newcommand{\commentfootnote}[1][]{}

\def\infordinala{\omega}
\def\infordinalb{\omega_1}
\def\reale{\mathbb{R}}
\def\intregi{\mathbb{Z}}
\def\complexe{\mathbb{C}}
\def\naturale{\mathbb{N}}
\def\rationale{\mathbb{Q}}
\def\designer{\mathbb{D}}
\newcommand{\paper}[1]{paper}
\newcommand{\multime}[1]{\left\{ #1 \right\}}
\newcommand{\definitie}[1]{\textbf{#1}}
\newcommand{\ghilimele}[1]{``#1"}
\newcommand{\lnotat}[1]{\sim #1}
\newcommand{\citare}[1]{(\cite{#1})} % TODO: Separate this in a version with parenthesis and one without.

\newtheorem{definition}{Definition}
\newtheorem{afirmatie}{Claim}
\newtheorem{notatie}{Notation}
\newtheorem{theorem}{Theorem}[section]
\newtheorem{lemma}{Lemma}
\newtheorem{axiom}{Axiom}
\newtheorem{note}{Note}

\begin{document}

 %%%%%%%%%%%%%%%%%%
\maketitle
 %%%%%%%%%%%%%%%%%%

\begin{abstract}
  This \paper{} presents an argument that, with probability $1$, the laws
  that govern a non-designed universe are infinitely complex at any level of observation. From this it follows that there is a $0$ probability of us living in a non-designed universe. 
\end{abstract}

\section{Introduction}
\subsection{Background}
\subsection{Probabilities}

Let $A$ be a set with the same cardinality as $\reale$. Let $F$ be the set of
all mathematical predicates of one variable over $A$ that can be written as a
finite formula. If $f$ is such a predicate then let $A_f$ be the subset of
$A$ where $f$ is true, i.e. $A_f=\multime{a\in A\mid f(a) \mbox{ is true}}$.
Given a probability distribution over $A$, we define $P(f)=P(A_f)$.

Let $F_0$ be the set of all elements of $F$ such that their probability is
$0$ for all continuous probability distributions over $A$, i.e.
$$F_0=\multime{
  f\in F
  \mid P(A_f)=0
    \mbox{ for all continuous probability distributions over } A}.
$$
As an example, any predicate which is true for a finite subset of elements,
i.e. $A_f$ is finite, would belong to $F_0$. Let us identify by $A_0$ the
set of elements of $A$ for which at least one predicate of $F_0$ is true, i.e.
$$A_0=\multime{a \in A\mid \exists f\in F_0 \mbox{ with } f(a)\mbox{ true} }.$$
Let us also denote by $F_1$ and $A_1$ the complements of $F_0$ and $A_0$,
respectively.

In the following we will assume that we are using a continuous probability
distribution over $A$.
\svn{Do I need this?}

Let us note that $F$ is countable, so $F_0$ is at most countable. Then,
obviously, $P(F_0)=0$, so the probability of its complement, $F_1$, is $1$.
Similarly, $P(A_0) = 0$ and $P(A_1) = 1$ which justifies the indexes used for
these. Then let us say that an element $a\in A_1$ is \definitie{generic}
and an element $a\in A_0$ is \definitie{peculiar}. Then we could rewrite
the equalities above to $P(x\mbox{ is generic}) = 1$ and
$P(x\mbox{ is peculiar}) = 0$.
\svn{I should actually use the \ghilimele{is finite} predicate or something
     like that.}

\subsection{Ordinals}

Ordinals are generalizations of natural numbers. Natural numbers can be
defined by identifying $0$ with the empty set $\emptyset$, then by saying
that each natural number is the set of natural numbers less than it. This
means that $1$ is the set $\multime{0}=\multime{\emptyset}$,
$2$ is the set $\multime{0, 1}=\multime{\emptyset, \multime{\emptyset}}$,
and so on.

Each of these natural numbers is an ordinal. To define the first ordinal
that is larger than all the natural numbers, $\omega$, we will use the
same rule: let $\omega$ to be the set of ordinals smaller than it, i.e
the set of natural numbers, $\omega=\multime{0, 1, 2, \dots}$.

Of course, the next ordinal, called $\omega + 1$, will be the set
$\multime{1, 2, \dots \omega}$ and the next one,
$\omega+2$, will be $\multime{1, 2, \dots \omega, \omega+1}$. We can
continue and, in the same way, define $2\cdot\omega=\omega+\omega$ to be
the ordinal that comes after all $\omega+n$ where n is a natural number
\svn{Use finite ordinal instead of natural number.}.

Then we can define $\omega\cdot 3$, $\omega\cdot 4$ and so on, and we can
take $\omega\cdot \omega$ to be the ordinal that comes after all the ones
defined by using the above rules.

Let us note that $\omega$ is countable, and that all the ordinals mentioned
above that come after it are also countable. By using the same kind of
reasoning as above we can produce other countable ordinals like
$\omega^\omega$ (from ordinals like $\omega\cdot\omega\cdot\dots\cdot\omega$)
and $\epsilon_0$ (from $\omega^{\omega^{\cdots^\omega}}$).

After going through many similar processes, at some point we obtain the
smallest uncountable ordinal, $\omega_1$, which is the set of all
countable ordinals.

Let us note that some ordinals, like all the finite ones except $0$,
and like $\omega+1$, can be obtained from the previous one by using
a succesor relation, i.e. $succesor(\alpha) = \alpha\cup\multime{\alpha}$.
All ordinals have a succesor, but not all are succesors, some, like
$\omega$ and $\omega\cdot 2$ can be defined only as the set of all
smaller ordinals. The former are called \ghilimele{successor ordinals},
the later are called \ghilimele{limit ordinals}. Note that $0$ is a limit
ordinal.

Transfinite induction is a generalization of induction, where, if we can
prove that a property holds for an ordinal $\alpha$ from the fact that
it holds for all ordinals $\beta<\alpha$, then this property holds for
all ordinals.

In many cases transfinite induction proofs are done separately for
succesor ordinals (maybe in the form $p(\alpha)$ implies $p(\alpha+1)$)
and for limit ordinals.

\subsection{Universe Models}

Let us consider all universes for which the
\ghilimele{state of the universe at a given moment in time},
or something close enough to it, makes sense. As an example of something
close enough, one could consider a universe with a space-time,
but in which there are some space-time points whose time can't be compared.
In such a case one might (or might not) be able to replace the state of
the universe at a given moment in time with the state of the universe
for a section through a universe's space-time that splits the universe
in three parts: the section itself, the points which are before the
section in time, and the points which are after the section in time.

To keep the exposition simple, in the following I will use
\ghilimele{state of the universe at a given moment in time},
but one should replace it with one's favourite alternative concept.

Then a \definitie{universe description} is a mathematical theory that has
a set of axioms which is at most countable and which allows making
predictions about the future state of the universe given its state
at a certain moment in time. A \definitie{universe region description}
is something similar, but only for a given space-time region of a universe.
In the best case, for a deterministic universe, there might exist
a description which allows one to correctly predict the entire future state
given the state at any moment in time, but a universe description
as defined here does not have to predict everything and,
even when it predicts something, it does not have to always be correct.

Note that the data available for making predictions is dependent
on who is making the prediction, which, in turn, means that
different universe descriptions may be useful in different context.
As an example, in a a universe like ours there may be many
kinds of beings: beings which can observe its full present state,
beings which could observe all types of things in the universe,
but which at a given time can only observe a limited amount of
space around them, and beings which are blind to many kinds of
things in the universe, and the influence of those things on the things
they can observe shows up as random noise. Each of these types of beings
may discover and may prefer different types of universe descriptions.

In order to handle this dependence on who observes the universe
in a reasonable way, in the axioms below we will work with universes
that contain intelligent beings
\svn{check that I mentioned this}
and all predictions would be relative to what these intelligent beings
could observe
\svn{say that these beings are included implicitly everywhere or
     include them explicitly everywhere}.

A universe description for which we don't know how good it is would
be less useful. Let us try to specify how good an univese description
is. An \definitie{approximate universe description} with a
\definitie{level of approximation} $\eta>0$, $p>0$ and $\Delta t>0$
is a universe description which allows one to approximate the future
state of the universe with a precision $\eta$, with a probability
$p>0$ for a prediction to be correct
\svn{Do I need the probability of the prediction to be correct
     or do I need to cover a fraction p of the world?}
and for a limited amount of time $\Delta t$.
\svn{I must answer the following questions: exactly what does that
     probability of correct prediction mean? Is it an aggregate among
     all possible predictions? Does it mean that all of them need a p
     chance of being correct? Do I need a fraction of the universe here?
     All writable predictions are countable, so they all have a non-zero
     weighth, so the aggregate probability most likely can't be zero.
     Am I talking about provable predictions or about predictions
     which are true on models?}

TODO: What is the probability of a prediction to be true? Let us say
that we have a probability distribution for all the predictions that
the intelligent beings could make and that p is the average probability
of a prediction to be true.

For a given universe
\svn{and some beings inside of it?}
or region of a universe,
\svn{did I define this?}
given a level of approximation, we will pick a canonical description
in the following way: Let $S$ be the set of descriptions which approximate
the universe with the given level of aproximation. If $S$ contains
at least one finite description, then we pick the shortest such
description as \ghilimele{the canonical description}, breaking ties by
using the lexicographic order. Otherwise, we simply say that the
universe (region) has an infinite description, and we will abuse the
terminology a bit by picking $S$ as the canonical description.
\svn{Whoa, did I just pick the entire set as the description?
     Does it actually work? If yes, then that's cool!}
If the level of approximation is obvious from the context, we will call
this canonical description \definitie{the universe's description}
or the \definitie{universe region's description}.

One could use a well-ordering on the real numbers to choose the
lowest description as the universe's description, but that will
complicate things without any benefit.

In the following, we will only talk about universes (or universe regions)
which have space and time and which have an approximate description
for a given level of approximation (or which can be split into
a countable number of regions which have such approximate descriptions).
\svn{What did I just say here?}

\subsection{Options for our universe}

The reminder of this paper will analyze what we can reasonably believe about
our universe in in a mix of the following cases:
\begin{itemize}
  \item Our universe is created or not.
  \item Our universe has finite or infinite space.
  \item Our universe has finite or infinite time.
  \item Our universe has a finite or infinite description.
  \item Option 1: There is a meta-$\beta$ universe for each countable ordinal
        $\beta$ such that our universe is the meta-$0$ one and the meta-$beta$
        universe includes, directly or not, all meta-$\alpha$ universes with
        $\alpha < \beta$. Option 2: there is an ordinal $\beta$ where this
        stops being the case.
  \item Life is or is not compatible with a finite chunk of space-time having
        an infinite observable description.
\end{itemize}

\section{Argument}

\subsection{Axioms}
\subsubsection{Observing events}
\begin{axiom}
  We will assume that, if $P$ is a probability over $\reale$
  (which does not have to be a probability over the Borel algebra of $\reale$)
  for which singletons are measurable (can be generalized to having a set of
  disjoint measurable events with the same cardinality as $\reale$), then,
  if we observe $x\in\reale$ and $P(x)=0$, then $x$ is generic.
\end{axiom}

Note that the set of events for which $P(y)$ is
greater than $0$ is at most countable, and, if we remove them from $\reale$, we
get a set with the same cardinality as $\reale$. On this set, the probability
of all peculiar events taken together is $0$, so there is no chance of us
observing one. Alternatively, the probability of all generic events is $1$,
so we can be sure that we observed a generic event.

Of course, the possibility of observing a peculiar event still exists, but,
practically, we will never observe it.

\begin{axiom}
  We will also assume that, if we observe $x\in\reale$, when we could have
  observed any real number, and there is no probability distribution that could
  describe how $x$ was chosen, then $x$ is generic.
\end{axiom}

Note that this axiom does not say that we do not know that probability
distribution, it says that there is no such probability distribution.
Anyone believing that this cannot happen should treat the cases where
this axiom applies as invalid.

Also note that this cannot happen when using subjective probabilities.

If there is nothing that could favor
peculiar numbers over generic ones, it's absurd to think that we could have
observed an element of such a tiny set among something infinitely larger.
Also, the case with a probability above suggests that this may be the only
reasonable assumption in this case.

\begin{note}
  If we divide $\reale$ into (disjoint) countable subsets, we still get $\reale$
  such subsets, so the above can be generalized to observing a countable set
  of events.
\end{note}

\subsubsection{Design probability}
\begin{axiom}
  The probability that our universe is designed is not $0$.
\end{axiom}

There are a few ways to go about this:

1. Simply stating that design means that non-continuous distributions
are used and that it's the only thing that can explain this $0$
probability. I wouldn't find this to be very convincing because,
even if it's a non-continuous distribution, the same reasoning as
above could be used.

2. State that there is a non-zero probability that someone would design
a universe for beings with finite mathematical power, and that there is
a non-zero probability that this means that the universe can be
understood by those beings. This is better, but it's still not clear
why there is a non-zero probability for designing for beings with
finite mathematical power.

3. Say that things may be different if we had any hint about
the existence of the $\infordinalb$ set of peculiar universes.
However, we don't, but we have a hint about a Designer that designed
the universe for us, which means that axiom \ref{noreason} does not apply.
It's not clear why this hint would help to change a $0$ probability
in a non-zero one.

4. Make a difference between a person and a dead meta-universe
that makes it clear that peculiarness is expected. One way to do that
would be to start with the first point, namely that a Designer would
have an intent when doing designs and that he attempts to pick
the option which he expected to fulfill that intent in the best way possible.
Therefore, optimally there is one design per intent, or, at least,
a small number of designs, depending on how good the Designer is.
There is a question of how big is the number of intents.
So, if the intents are countable and the designer is somewhat capable,
then we get a discontinuous probability distribution over
universe descriptions.

This, in itself, is not enough, we need to know why would a Designer
have an intent which requires a peculiar universe.

Now, it's true that we may not always be able to distinguish
between a Designer and a blind force, and that we can only do that
when the Design is peculiar, but still, that happens only when there's
a non-zero chance that the Designer would do a peculiar design,
so we need a separate argument for that.
\svn{This may be fit for the introduction or something similar.}

Still, there are some reasons for this non-zero chance.

One is that variety is interesting and that this is a special
type of variety, therefore maybe there is an intent which requires
a peculiar universe. If the peculiar universes would be countable
then we could distribute this non-zero probability
such that all of them have a non-zero probability.
If not, it's just our hope.
\svn{How is it?}

Another one is that a Designer's intent is about beings which can reason.
If the Designer's power of reasoning is infinite, which is not
an unreasonable assumption, then there are two options: the intent
is also about beings with infinite reasoning power, or the intent
is about beings with finite reasoning power. Now, if it happens
that there are enough beings with infinite reasoning
power\footnote{Say, $3$ such beings, but any number would do.}, then the
Designer may also intend to design beings with finite reasoning power.
If the Designer's power of reasoning is finite, then it's likely that
the only option is to design other beings with finite reasoning power.
\svn{Maybe reword this as "there is a non-zero chance that...".}

But if the Designer designs beings with finite reasoning power,
then the Designer may also design an environment where these beings
can use their reasoning power effectively, i.e. an environment which,
at the level an which these beings usually function,
can be mostly understood by them, i.e. it has a finite
approximate description which allows the prediction of a non-trivial
number of things.

5. Use Swinburne's argument about why would God create human-like beings.

6. Make an argument that there is a non-zero probability that
a Designer may want variation, so He may want to create different
kinds of things: things of higher complexity and things of lower complexity,
so there is a discontinuity for peculiar things.

\subsubsection{$\reale^4$ universe}

This set of axioms attempts to say that our universe can be modelled
on top of something close to $\reale^n$, without specifying an exact
mathematical object.

\begin{axiom}\label{unitycovering}
  For all $q\in\rationale$, the subset of hypercubes of size $q$ and corners
  having rational coordinates is countable and covers our universe.
\end{axiom}

This axiom states that our universe can be covered by a set of hypercubes
of a given size. While this could certainly be false, it is implied by some
basic beliefs that we have about our universe, namely that there is a
level of approximation for which the space and time anywhere in the universe
can be approximated locally by an open set of $R^n$.

\subsubsection{Logically possible universes}
\subsubsection{Universe Covering}
Is this needed?
\subsubsection{Neighborhood modelling}
\subsubsection{Turtles all the way down}

\subsection{Valid options for our universe}

This section will try to develop the axioms above in order to find out whether
it is reasonable to believe that: our universe is (or is not) created,
space and time are infinite, the laws that we use to describe the behaviour
or the spacetime around us apply in the entire universe, our universe
is contained in an uncountable chain of meta-universes, life is
compatible with a finite chunk of space-time having an infinite
observable description.

We will focus mostly on what happens when our universe is not created.

\subsection{Finite Descriptions and Uncountable Meta-universes}
\label{fdaumu}

Let us assume that our universe has a finite description $x\in\reale$.
\svn {Mention that we identify descriptions with real numbers, finite
      descriptions with natural numbers, but we don't keep the
      operations/topological structure. Alternatively, don't use
      $\reale$ here since it's misleading.}
Let us
assume that our universe is contained in a meta-universe, which is contained in
a meta-universe, and so on, these meta-universes corresponding to ordinals.
If there are ordinals which do not correspond to any meta-universe, and
the smallest one is countable, let us denote it by $\alpha$.
Otherwise, let $\alpha$ be the lowest uncountable ordinal.

If $\alpha = 0$ then there is no probability distribution over universe
descriptions, so $x$ is generic.
\svn{Did I explain what a meta-universe is? And how it is related to
     probability? I.e. If there is no meta-universe, there is no probability
     distribution}
\svn{Mention this axiom.}

Otherwise, if $\alpha > 0$, but at level $0$ there is no probability
distribution over universe descriptions, $x$ is generic.
\svn{I said that our universe is the meta-$0$ one, but here it is
     the meta-$-1$ one.}
\svn{Explain that probability $0$ does not mean logical impossibility, it
     means that if we bet on that an infinite number of times
     (the infinite is countable), we will lose each time. So, assuming that
     we have a non-zero probability alternative, we should bet on that.
     For easier writing I will treat that as logical implication, but the
     reader should keep in mind that the actual meaning is closer to
     \ghilimele{practical implication}.}

Otherwise, let us assume that at some levels $\beta$ with
$0 < \beta \le \alpha$ there is a probability
$p_\beta$ over $\reale$, with $p_\beta(y)$ being the probability that $y$ is
a discontinuity for all $p_\gamma$ with $\gamma < \beta$.

Let $\beta$ be the minimum (if any) for which $\beta < \alpha$ and
$p_\beta$ does not exist or $p_\beta(x) = 0$.

% If such a $\beta$ exists, then let us denote by $P$ the joint probalility for
% all the ordinals up to $\beta$ and including it.
% Then $p_\gamma(x) = P(x|\gamma)$. We have $P(x|\gamma)>0$ for all
% $\gamma<\beta$ and $P(x|\beta) = 0$. Then $P(x) = 0$

If such a $\beta$ exists, then, although in all the lower universes
$\gamma < \beta$ we may be able to bet on a peculiar description for our
universe, at level $\beta$ we can be sure that a bet on peculiar descriptions
would fail. In other words, at level $\beta$ the only reasonable bet would be
that we are in a chain of (meta-)universes where our universe's description is
generic.
% Since we only placed an at-most-countable number of bets until now,
%we don't expect certain failure.
%\svn{Why we don't expect failure?}

If such a $\beta$ does not exist, up to $\alpha$, without including it,
$p_\beta$ exists and $p_\beta(x) > 0$.

If $\alpha$ is countable and there is no meta-universe above $\alpha$,
then, similar to the above reasoning, as far as $\alpha$ is concerned,
$x$ is generic.

This means that, if our universe is not created, the only option that
has a chance of being reasonable is that $\alpha$ is uncountable and,
for all $\beta < \alpha$, $p_\beta$ exists and $p_\beta(x) > 0$.

Note that, in order to claim that $x$ is peculiar, one needs to postulate the
existence of an uncountable chain of meta-universes, all of them favouring a
peculiar $x$, which, by default, is unlikely for any of them. Also, although the
current argument does not work anymore with an uncountable chain of
meta-universes, intuitively the peculiarness problem still remains.

From now on, I will assume that the possible objection in the preceding
paragraph is unreasonable, which means that, practically speaking,
either $x$ is not peculiar or our universe is designed.

Next, let us ask consider if a universe with finite space-time can have an
infinite description that is compatible with intelligent life. If this is not
possible, then either our universe is designed, or it has an infinite
space-time, or it has a finite spacetime with an infinite description.

%Ok. For the countable ordinals: if one of them implies that $x$ is generic,
%then it is. For the uncountable ordinals: if only those imply that $x$ is
%generic, it may not matter. Why? One argument goes like this: the ordinals up
%to an ordinal tell us that it's plausible that $x$ is peculiar. However, that
%ordinal tells us that it's implausible, which also means that the ordinals below
%are implausible. If we're selecting from a set that's small enough,
%it's implausible that we get implausible elements. Small enough means countable.
%However, above countable ordinals, we expect to also get implausible results.
%This is also suggested by the generalization above.


\subsubsection{Peculiar description for universe regions}

From axiom ??, time and space are close to $\reale^4$.
\svn{Add this axiom. Mention that we can cover the entire space with a
     countable set of cuboids of size 1, each of which can be specified in a
     finite way, e.g. cuboids which have coordinates rational numbers.
     Mention that the cuboids are rectangular ones. Mention also that
     we are also considering cuboids with coordinates in
     $\rationale\cup\multime{-\infty, +\infty}$
     Actually, what I need is that cuboids with generalized rational corner
     coordinates cover the entire space. This can be generalized to remove
     rational numbers from the condition and to use other shapes, but that does
     not matter.}

Let us take a countable covering with (generalized) cuboids of our universe
with their corners' coordinates in $\rationale\cup\multime{-\infty, +\infty}$.
Let us take all finite unions of such cuboids. Their set is countable.

If a finite chunk of spacetime can have an infinite approximate
description which is compatible with inteligent life in the universe, then the
argument in section \ref{fdaumu} also applies to a union of cuboids that covers
finite chunks of space (we consider it as a mini-universe inside
our universe), so either each such union has a generic description,
or finite chunks can only have peculiar descriptions, or our universe
is designed.

However, we believe that we can have peculiar/finite approximate descriptions
for the region of universe around us, which would leave us with the following
two options: finite unions of cuboids that cover a finite chunk of space can
only have finite descriptions, or our universe is designed.

However, from section \ref{fdaumu}, our universe should have an
infinite description, which means that either our universe is infinite,
or it is designed.

If our universe is infinite, unions of cuboids that cover an infinite
amount of time-space can have infinite descriptions. Applying the argument
in section \ref{fdaumu}, we get that either each such union has a generic
description, or our universe is designed.

To summarize, either (finite unions of cuboids that cover finite chunk of space
can have infinite descriptions) or (our universe is infinite, finite unions of
cuboids that cover an infinite chunk of spacetime have only infinite
descriptions, while finite unions of cuboids that cover a finite chunk of
space can have only finite descriptions) or (our universe is designed)

Let us divide the spacetime around us into spacetime hypercubes of size, say,
$1 second \times meter^3$. Let us assume that our universe is not created
and that finite unions of cuboids that cover a finite amount of space
can have infinite approximate descriptions. Then each such hypercube should
have a generic approximate description. However, we generally believe that
we can describe the space around us fairly well using a finite description
(axiom ??), so we will disregard this possibility. Even if we are unwilling
to disregard it, the same ideas as in the next section can be used to
disregard it.

\subsubsection{Improbable consistency}

Let us analyze what happens if finite unions of cuboids covering a finite
spacetime can only have finite approximate descriptions.
Then let $A_1$ be the set of possible approximate descriptions for
the hypercube of size, say, $1 second \times meter^3$.
Unless our approximation is extremely coarse, $A_1$ will have multiple elements.
Let $P$ be the probability of encountering an element of $A_1$ in our universe.
Since we don't know $P$, by default we could choose equal probabilities for
all elements of $A_1$. However, let us consider what actually happens.
\svn{why is there a probability?}
It would be unreasonable to assume that the probability is zero for all
elements except one. This is suggested by two things: first, there are too many
possible universes where this does not happen; second, our universe has an
infinite description.
\svn{can I prove this?}
Then, if the description that we use for the universe around us
has probability $p_1 < 1$, then the probability of observing $n$ hypercubes
with this description without observing any other description is $p_1^n$.

Even if $p_1$ is very close to $1$, $p_1^n$ converges very quickly to $0$,
so the consistency of a small spacetime region around us is enough
to make this hypothesis unlikely enough (compared to the others, assuming
that at least one other has a non-zero probability) to disregard it.

\section{Conclusion}

We have, then, the following reasonable possibilities:

\begin{enumerate}
\item Our universe is created.
\item Our timespace is infinite and each infinite chunk has an infinite
      description. At the same time, one of the following would be true:
      \begin{enumerate}
      \item Any approximate description for a limited chunk of space around us
            is infinite (assuming that this is possible and compatible with
            life). This is not what we currently observe around us.
            This also means that we can't know much about the past and we can't
            predict much about the future. The observable laws of the
            universe would vary a lot around us.
      \item Any approximate description for a limited chunk of space around us
            is finite (assumes that infinite descriptions are impossible or
            incompatible with life). At the same time, one of the following
            would be true:
            \begin{enumerate}
            \item We are in a very unlikely chunk of our universe, where,
                  given a reasonable decomposition in small pieces (e.g.
                  hypercubes of size 1), all pieces around us that we observe
                  have the same approximate description. The probability
                  of this happening converges quickly to $0$ when the number
                  of observed pieces goes up.
            \item The universe pieces around us do not have the same
                  approximate description. As above, this also means that
                  we can't know much about the past and we can't
                  predict much about the future. This is also not what we
                  observe.
            \end{enumerate}
      \end{enumerate}
\end{enumerate}

%%%%%%%%%%%%%%%%%%%%%%%%%%%%%%%%%%%%%%%%%%%%%%%%%%%%
%%%%%%%%%%%%%%%%%%%%%%%%%%%%%%%%%%%%%%%%%%%%%%%%%%%%
%%%%%%%%%%%%%%%%%%%%%%%%%%%%%%%%%%%%%%%%%%%%%%%%%%%%
%%%%%%%%%%%%%%%%%%%%%%%%%%%%%%%%%%%%%%%%%%%%%%%%%%%%

\section{Introductive notions}

\subsection{Probabilities}

\subsection{Ordinals}

\subsection{Main lemma}

Let $\alpha$ be an ordinal.

For each $\beta\le\alpha$, let us assume that:
\begin{enumerate}
  \item $E_\beta$ is a non-empty set with unspecified elements.
        $E$ is the disjoint union of
        all $E_\beta$.
  \item There is a descendant-ancestor $d$ relation on $E$ ($d(e, f)$ is read as
        \ghilimele{$e$ descends from $f$} or \ghilimele{$f$ is $e$'s ancestor})
        that satisfies the following:
        \begin{enumerate}
          \item An element has at least one ancestor on each higher level:

                For all $\gamma\le\alpha$, $\beta<\gamma$,
                $e_\beta\in E_\beta$,
                there is an $e_\gamma\in E_\gamma$ such that
                $d(e_\beta, e_\gamma)$.
          \item An element has no ancestor on its level, or below:

                For all $\gamma\le\alpha$, $\beta\le\gamma$,
                $e_\beta\in E_\beta$,
                $e_\gamma\in E_\gamma$, it is not the case that
                $d(e_\gamma, e_\beta)$.
          \item Transitivity:

                For all $\gamma\le\alpha$, $\beta\le\alpha$, $\delta\le\alpha$,
                $e_\beta\in E_\beta$, $e_\gamma\in E_\gamma$,
                $e_\delta\in E_\delta$, if $d(e_\beta, e_\gamma)$ and
                $d(e_\gamma, e_\delta)$, then $d(e_\beta, e_\delta)$.
        \end{enumerate}
  \item $E^\beta_\gamma$ is defined as
        $$E^\beta_\gamma=\multime{
          (e_\delta)_{\gamma\le \delta < \beta}
            \in\prod_{\gamma\le \delta < \beta}E_\delta
          \mid
          \forall \gamma\le\delta<\epsilon\le\beta, d(e_\delta, e_\epsilon)}.$$
  \item $D_\beta$ is a non-empty set of probability distributions over $\reale$.
  \item $p_\beta$ is a partial function from $E_\beta$ to $D_\beta$, providing
        a probability distribution over $\reale$ for some of the $e\in E_\beta$.
        When $\beta$ is obvious from the context, we will denote $p_\beta(e)$
        by $p_e$.
  \item If $\beta > 0$ then, for each $e\in E_\beta$ we have a way of choosing
        tuples $(e_\gamma)_{\gamma<\beta}\in E^\beta_0$ in such a way that,
        if $p_\gamma{e_\gamma}$ is defined for all $\gamma<\beta$,
        $p_e(x)$ gives us the probability of $x$ being a discontinuity for
        all $(p_{e_\gamma})_{\gamma<\beta}$.
  \item For $0 < \gamma < \beta$ and $e_\beta\in E_\beta$, if
        $p_{e_\delta}$ is defined for all $\gamma\le\delta<\beta$ and $x$ is a
        discontinuity for
        $(p_{e_\delta})_{\gamma\le\delta<\beta}\in E^\beta_\gamma$, then
        $p_{e_\beta}(x \mid (e_\delta)_{\gamma\le\delta<\beta})
          = p_{e_\gamma}(x)$.
\end{enumerate}

As an example, let us fix $e_\alpha$ in $E_\alpha$ and let us assume that we
have a probability $P = P^{e_\alpha}$ over $E^\alpha_0$. We
define $P^\beta$ to be $P$ restricted to
$E^\beta_0$, i.e. $P^\beta(F) = P(E^\alpha_\beta \times F)$.
We define $p_\beta$ inductively by:
\begin{enumerate}
  \item $p_0(e_0)$ are probabilities over $\reale$.
  \item Assuming that $\beta > 0$ and that we have defined $p_\gamma$
        for all $\gamma < \beta$, let
        $$F_\gamma(H) =
          \multime{e\in E_\gamma
            \mid p_e \mbox { has a discontinuity in }H}$$
        and let
        $$F^\gamma_\delta(H)
          = \prod_{\delta <= \epsilon < \gamma}F_\epsilon(H).$$
        We define $p_{\beta}$ in the following way:
        $$p_{e_{\beta}}(H) =
          P^{\beta + 1}(F^{\beta}_0(H) \mid e_{\beta}).$$
\end{enumerate}
The above is well-defined if each of the sets $F_\gamma(H)$ is
measurable. Strictly speaking, we only need $F_\gamma(x)$ to be measurable for
all $x$, which is true in various cases, e.g. when all subsets of $E_\gamma$ are
measurable.

\begin{lemma}
  Let $\alpha$ be an ordinal. For all $\beta < \alpha$, let $d$,
  $E_\beta$, $E^\beta_\gamma$, $D_\beta$,
  $p_\beta$, be as above. Let us also assume that
  \begin{enumerate}
    \item we have a property $Q$ over $\reale$;
    \item there is a tuple $(e_\gamma)_{\gamma \le \beta}$ such that if
          $\delta < \gamma \le \beta$ then $d(e_\gamma, e_\delta)$;
    \item If $p_\beta(e_\beta)(x) = 0$ or $p_\beta(e_\beta)$ is not defined
          then $Q(x)$.
  \end{enumerate}

  Then $\neg Q(x)$ implies $p$ defined up to $\alpha$ and
  $p_\beta(e_\beta)(x) > 0$ for all $\beta$ up to $\alpha$.
\proof
  This is true in a stupid way from item 3 above. Do I really need all this
  silly framework? The framework is needed in order to make item 3 work,
  I guess.
\end{lemma}


\subsection{Universe models}


\section{Valid options and axioms for our universe}


\subsection{Proof axioms}

\begin{axiom}
Any finite universe has a non-zero probability of being designed.
\end{axiom}

Unless otherwise specified, below I will analyse what happens when our
universe is not designed.

\subsection {The actual proof}

\section{Case analysis}

\subsection{Meta-universes}

Let us consider what happens if we find that our universe's description $d$ is peculiar. From the axiom above, there is a probablility distribution that allows us to compute $P(d)$. But this means that there exists something outside of our universe. Maybe other universes exist, maybe there are no other universes, but there is something that influences why our universe is peculiar instead of generic. We will call this the meta-$1$ universe. If this meta-$1$ universe is not part of a meta-$2$ universe then the set of discontinuities for $d1$ is generic, so $d$ is generic, contradiction.

By transfinite induction, there is such a meta-$\alpha$ universe for any countable ordinal $\alpha$. Note that the set of all countable ordinals is an uncountable ordinal.

Postulating that many meta universes is considered unreasonable, especially since they are not generic ones, so any case analysis that ends with our universe's descripton being peculiar (meaning that there is an uncountable number of meta universes) will be discarded.

\subsection{Finite universe description}

If our universe's description is finite, then it's peculiar, hence we will
discard this hypothesis. From now on, our universe's description will be
infinite.

\subsection{Infinite description for finite chunks of universe}

As far as we can observe, our universe has the same finite description
everywhere. Let us take a finite space-time chunk $K$ outside the area needed
for our life. Then $K$'s description would be peculiar, which we will discard
as mentioned above.

The only remaining hypothesis is that a finite space-time chunk can only have
a finite description.

\subsection{Finite space or time}

As a consequence of the above, our universe must have either infinite space,
or infinite time.

\subsection{Remaining hypotheses}

The only remaining option for a non-created universe is that is infinite in
time, space, or both, it has an infinite description, no infinite local
descriptions are compatible with our existence, and our universe's
description is generic among all descriptions compatible with our existence.

We will say that a region of space-time is \definitie{consistent} if the
decription of any piece of that region is the same as the region's description.

It is likely that our existence would not be possible without some (finite)
time-space region around us having at least some consistency. Let us assume
that the time-space occupied by our bodies needs very high consistency, much
of Earth's and Sun's timespace needs high consistency, and that this
consistency requirement drops as we co further away from us. Of course,
there are many (possible or real) universes where this consistency does
not exist. By chance, some would have consistent regions which are large
enough that intelligent life exists. For almost all\footnote{This is an
intuitive \ghilimele{almost all}, not a statistical one.} of those,
inconsistencies would be readily observable, all life would end at
random points of inconsistency, and so on. The question we would like to
answer is what is more likely: that we are in a universe which, a far as we can tell, is consistent for huge spans of time and space around us, or that we are in a designed universe?

If we consider a space-time hypercube of size $1$, say $1$ cubic meter for $1$ second, and assuming some approximation parameters \svn{define these before this section}, there is a finite set of possible descriptions, say $D_{c1}$. Let's denote its size by $N$. Let us assume some approximation requirements for which $N$ is at least $2$ (note that for reasonable parameters $N$ is very large). Let us aldo denote by $D_{OURS}$ the description that we would give to our universe (and )

Then, by default, the probability that any such hypercube has a given description is $1/N$. The probability that $k$ such hypercubes have the same description is $1/N^k$. This converges very quickly to $0$, so it does not take much to go below any target probability\footnote{If $N=2$, $k=10$ is enough to go below one to one thousand odds, $k=20$ is enough to go below one to one million, $k=30$ goes below one to one billione}. Therefore, unless the probability of our universe being created is inimaginbly close to $0$, the region of universe that we can observe and which we believe to have the same observable description as ours is enough to discard the hypothesis that it was not designed\footnote{Let $ND$ be the hypothesis that our universe id not designed, $D$ be the hypothesis that it is designed, $OUR$ be our universe, $our$ be the region of space around us that has the same description. Note that $P(our) \ge P(OUR)=P(OUR|D)\cdot P(D) > 0$. Then $P(D|our) = \frac{P(our|D)\cdot P(D}{P(our)}) > 0$. Similarly, $P(ND|our) = \frac{P(our|ND)\cdot P(ND}{P(our)}) = \frac{P(ND)}{N^k} \cdot \frac{1}{P(our)}$. To compare the two we have to compare $P(our|D)\cdot P(D)$ with $\frac{P(ND)}{N^k}$.}. 

But, although a probability of $1/N$ is a sensible default, someone may prefere
a more skewed probability distribution. It is unresonable to use a distribution
which assigns a $0$ probability to any of the possible cases, so let $q > 0$ be
the probability of a given hypercube having the same description as our
timespace region. Again, $q^k$ decreases very quickly, so unless $q$ or the
probability of our universe being desined are extremely skewed, a very small
region of space around us is enough to make us discard the not designed hypothesis.

\section{Axioms}

In the following, OURS could be either \ghilimele{our universe}, \ghilimele{our galaxy from some time around its beginnings until some time around its possible end} or \ghilimele{the observable region of our universe extended as much as possible, in a sane way, through time}, or anything similar. To make it easier to talk about it, let us pick the last option.

Whenever we are talking about descriptions of OURS, we, humans, are the intelligent beings which may use these descriptions.

There is a level of approximation $L$ and a rational number $r$ \svn{Do I need $r > \eta$?}such that the following axioms hold.

\begin{enumerate}
\item \label{countablecovering} Given a positive rational size $q>r$, OURS can be covered by a set that is at most countable of time-space hypercubes of size $q$.
  \begin{itemize}
    \item Note that this axiom is stronger than needed, one could use open sets of size at most $q$ instead of hypercubes.
  \end{itemize}

\item \label{nonessentialhypercube} There is a time-space region of OURS which is non-trivial (i.e. it contains a non-empty hypercube of size greater than $r$), let us call it $R_{ours}$, with the following properties:
  \begin{enumerate}
    \item One can specify its limits precisely enough (i.e. within the level of approximation $L$) using a finite mathematical predicate. As an example, one could choose a time-space region in that contains the space from Mars' orbit to Pluto's, through all the past and future time where its position can be specified reasonably using its current position as a basis.
    \item We could use a finite description $d_{ours}$ to describe this region within the level of approximation $L$. As an example one could use classical mechanics, maybe with some chemistry and some quantum physics (deterministic) approximations to describe our solar system fairly well. \svn{These examples are repeated in the "Why are these axioms reasonable" section. Maybe I should delete them here.}
    \item $R_{ours}$ is outside the part of our universe which is essential for our existence, e.g. outside of Earth's orbit around Sun. If we choose that \svn{what does \ghilimele{that} mean here?}as our time-space region, when looking back in time, what happens in this region may become essential for us, say at some time $t_0$. This simply means that we will need to start the time-space region after that time $t_0$.
  \end{enumerate}
\item \label{uncountable} There is a set $D$ of descriptions of universe regions such that the following are true:
  \begin{enumerate}
    \item $D$ has the same cardinality as $\reale$.
    \item For all elements $d\in D$ there is at least one conceivable universe $U_d$ which
      \begin{enumerate}
        \item contains intelligent beings that use mathematics,
        \item has a time-space and a region $R_d$ of $U_d$ that can be specified, within the level of approximation $L$, using a finite number of words.
        \item for the intelligent beings mentioned above and for the level of approximation $L$, $d$ is a description of $R_d$.
      \end{enumerate}
    \item $d_{ours}\in D$
  \end{enumerate}
\item \label{infinitechain} The probability that there is a meta-universe for each ordinal in $\infordinalb$\footnote{The set of all countable ordinals and the lowest uncountable ordinal}\svn{Is this the right cardinal?}, such that the following are true, is $0$:
  \begin{enumerate}
    \item our universe is the $0$-th meta universe;
    \item each universe corresponding to an ordinal $\alpha$\svn{I should put the next paranthesis outside of the axioms somehow.} (the $\alpha$-th meta-universe or the $(meta-)^\alpha$ universe) determines, directly or indirectly, in all lower ordinals, the part of the laws of universes which is relevant to us in one way or another;
    \item for all $\alpha$, the laws of the $\alpha$-th meta-universe that define the laws of the lower meta-universes are not generic.
  \end{enumerate}
  \begin{itemize}
    \item Note that the term meta-universe here is used loosely, it just means
          that there is something outside of a universe, which is not a
          designer, but which is a reason (as in axiom \ref{noreason})
          for the laws of our universe.
    \item Note that, if all possible universes exist, the class of all these universes could be considered a $meta-$universe.
  \end{itemize}
\item \label{noreason} If $A$ has the same cardinality as $\reale$ and one observes an element $a\in A$, and there is absolutely no reason for $a$ being observed over any other element, then $a$ is generic.
  \begin{itemize}
    \item Note that \ghilimele{there being a reason}, is used here in a very generic way, it just means that there is something that influences this observation. In particular, it does not necessarily mean that $a$ is determined by a proces. E.g. observing the time interval between two particles being emmited by a radioactive object has some reason behind it, i.e. the way our universe works means that some ranges of values have higher probabilities than others. Or maybe the way our universe works means that all possible values have equal probability, or that all possible value ranges have probabilities proportional with their size: This also means that \ghilimele{there is a reason} for observing an element.
    \item Note that the only case relevant to this paper when when we might observe something for which there is no reason, would be when we are observing our universe: if our universe is not designed, it might be that there is nothing above our universe which would have an influence over how our universe is.
  \end{itemize}
\item \label{ourplausible} $P(R_{ours} \mid Designer) > 0$ and $P(Designer)>0$.
\end {enumerate}

\section{Results}

\begin{theorem}
  If there is no Designer then the probability that the region of space mentioned in axiom \ref{nonessentialhypercube}, $R_{ours}$, exists, is $0$.
\end{theorem}

\begin{proof}

First, let us note that the set of finite descriptions is countable, while, from axiom \ref{uncountable}, the total set of descriptions $D$ is not countable. This means that finite descriptions are peculiar.

Let $H$ be the hypercube in axiom \ref{nonessentialhypercube} and let $q$ be a rational number such that $H$ is of size greater than $q$. According to axiom \ref{countablecovering}, such a rational number exists.

The laws of the hypercube are determined by the laws of our universe ($U$), i.e. the laws of our universe specify which of the possible sets of laws work in a specific hypercube. While it may be that the same observable laws apply in the same way accross the entire universe, this is likely a special case, so in a generic conceivable universe we will have a probability distribution over the entire set of descriptions. If a hypercube's laws are chosen using a continuous distribution, then they are generic with probability $1$. However, $H$'s laws are peculiar, which means that the probability of having such a hypercube is $0$. Note that this is true even if $U$'s laws say that every possible hypercube exists: $H$'s probability among all $U$'s hypercubes is still $0$.\svn{For this it is enough to say that there are two hypercubes with the same laws, not two hypercubes where I can predict enough things.}

Let us see what happens if they are chosen using a discontinous distribution $d_0$ which belongs to our universe, which we will call the $(meta-)^0$universe from now on. Let us note that $d_0$'s discontinuities are at most countable. If these discontinuities are generic, then $H$'s laws have a $0$ probability of being peculiar. If there is no $(meta-)^1$universe, i.e. there is no reason for these discontinuities, then, from axiom \ref{noreason}, $d_0$'s discontinuities are, indeed, generic.

The only case when they may not be generic is if they are determined by the $(meta-)^1$universe. Let us assume that at least one discontinuity is peculiar, and let us call it $x_0$. Let $d_1$ be the distribution probability which specifies for which points $d_0$ is discontinuous. If $d_1$ is continuous, then the probability of $x_0$ being peculiar is $0$. If $d_1$ has discontinuities, they may be generic, which means that $H$'s probability is still $0$. As above, if there is no $(meta-)^2$universe, then these discontinuities are indeed generic, so let us assume that there is a $(meta-)^2$universe and that one of $d_1$'s discontinuities is peculiar. Then we can repeat the same reasoning.

In general, if there is an $n$ such that the $(meta-)^{n}$universe is not included in anything that may influence the way its laws are, then, by axiom \ref{noreason}, $d_n$'s discontinuities are generic, which means that $H$'s probability is $0$. If it is included in a $(meta-)^{n+1}$universe and $d_n$ does not have a peculiar discontinuity, then $H's$ probability is $0$. To have a non-zero probability we need both a $(meta-)^{n+1}$universe and a peculiar discontinuity for $d_n$. But, in a similar way, this means that there is a $(meta-)^{n+2}$universe and $d_{n+1}$ must have a peculiar discontinuity.

So the only way in which $H$'s probability was not proven as being $0$ above is when there is an infinite chain of $(meta-)^{n}$universes. But let us note that this entire infinite chain of universes has a countable number of peculiar discontinuities. Then either its probability is $0$ because of axiom \ref{noreason}, or there is an universe corresponding to $\infordinala$\footnote{The lowest infinite ordinal} which determines the laws of the previous universes. But if we denote by $d_{\infordinala}$ the probability distribution that describes the previous universes' discontinuities, then, as above, either the previous universes' probability is 0, or $d_{\infordinala}$ has peculiar discontinuities.

Note that, as long as $\alpha$ is a countable ordinal, if the discontinuities of all ordinals lower than $\alpha$ are peculiar, then either the probability of having these discontinuities is $0$ or $d_\alpha$ has peculiar discontinuities. So, by transfinite induction, for all ordinals lower than $\infordinalb$, $d_\alpha$ has peculiar discontinuities. From axiom \ref{infinitechain}, the probability of this happening is $0$.

Then $H$'s probability is $0$ for all cases.
\end{proof}

\begin{theorem}
With probability $1$, our universe is designed.
\end{theorem}

\begin{proof}

Let $A$ and $B$ be disjoint events.

\begin{equation*}
\begin{split}
P(A \lor B \mid C) &= \frac{P((A \lor B) \land C)}{P(C)} \\
                   &= \frac{P((A \land C) \lor (B \land C))}{P(C)} \\
                   &= \frac{P(A \land C)}{P(C)} + \frac{P(B \land C)}{P(C)} \\
                   &= P(A\mid C) + P(B\mid C). 
\end{split}
\end{equation*}

If $P(A \lor B) = 1$:

\begin{equation*}
\begin{split}
P(C) &= P(C \land (A \lor B)) = P(C \land A) + P(C \land B) \\
      &= P(C \mid A) P(A) + P(C \mid B) P(B).
\end{split}
\end{equation*}

\svn{Remove the above equations.}

Let $\designer$ be the hypothesis that our universe is designed.

\begin{equation*}
\begin{split}
P(R_{ours}) &= P(R_{ours} \mid \designer) P(\designer) + P(R_{ours} | \lnot \designer) P(\lnot \designer) \\
            &= P(R_{ours} \mid \designer) P(\designer) > 0.
\end{split}
\end{equation*}

$$P(\lnot \designer \mid R_{ours}) = \frac{P(R_{ours} \mid \lnot \designer) P(\lnot \designer)}{P(R_{ours})} = 0.$$

\begin{equation*}
\begin{split}
1 &= P(\designer \lor \lnot \designer \mid R_{ours}) \\
  &= P(\designer \mid R_{ours}) + P(\lnot \designer \mid R_{ours}) \\
  &= P(\designer \mid R_{ours}).
\end{split}
\end{equation*}

\end{proof}

\section{Why the axioms hold}

\subsection{Axiom \ref{countablecovering}}



% While this could certainly be false, it would mean that almost everything that we believe about OURS is wrong. In particular, everything that we believe about the distant past or future would likely be wrong by large margins. E.g. the dinosaurs could as well have dissapeared 1000 years ago or 1000 billion years ago.

\subsection{Axiom \ref{nonessentialhypercube}}

This axiom states that there is a region of space and time for which we can, in principle, approximate both its position and its behaviour, and which would allow us to live if its behaviour was diffferent. Note that we don't actually have to provide an approximate behaviour for whatever is in that region, we just have to be able to build a mathematical model which would approximately predict that region's behaviour in a non-zero fraction of cases.
\svn{But predicting depends on what is in that box. If we fill it with whatever weird particles, black holes, and so on, maybe things get harder. OTOH, maybe it works if we can predict at all.}

One example of a region that fits axiom \ref{nonessentialhypercube} would be a time-space region $R_{ours}$ that includes our solar system from Mars' orbit to Pluto's, through all the time where its position can be specified reasonably using its current position as a basis. One could use classical mechanics, maybe with some chemistry and some quantum physics (deterministic) approximations to describe our solar system fairly well. Also, in recent-ish history, this $R_{ours}$ seems non-essential for our existence. We can extend this region in the past and future until it becomes essential.

Even if, somehow, this region is essential for our existence, we could go outside of our solar sistem, or even outside of our galaxy to find a good region.

\subsection{Axiom \ref{uncountable}}

This axioms states that, for a given level of approximation, there is a large set of conceivable universes, in some respects, are similar to OURS, but which are, at the same time, wildly different. Also, OURS belongs to this set.

To see why that is reasonable, let us first note that, in a similar way to axiom \ref{nonessentialhypercube}, OURS could have an approximate universe description given by classical mechanics, maybe with some additions.

Also, for almost any countable axiom system that still has $R^n$ as a base, one could imagine an alternate universe in which what is inside (say) Mars' orbit stays the same as in ours, but what is what is outside of Mars' orbit is described by that axiom set. All of these countable axiom sets form a set of descriptions with the same cardinality as $\reale$, and each of these imaginary universes would contain intelligent beings that use mathematics, and so on.

\subsection{Axiom \ref{infinitechain}}

This axiom assigns a zero probability for the existence of a non-countable, well-ordered sequences of universes, for which we don't have any kind of confirmation, no matter how weak. There is no chance of a revelation, or any other means of having even the weakest hint about its existence. Not having any confirmation might not be enough for assigning a zero probability to it, but this axiom describes something which is infinitely peculiar, and this infinite is uncountable. Negating this axiom would require a (very) strong faith in something for which we don't have any reason to believe.

To get a hint of how strong this faith must be, let us consider only the universes corresponding to finite ordinals. If there is a value $p < 1$ such that the probability of a peculiar $meta^n-$universe existing if a peculiar $meta^{n-1}-$universe exists is at most $p$, then the probability of all of them existing is lower than $p^n$ for any $n$, so it is $0$.

In order to have a non-zero probability, we need the probability of a $meta^n-$universe existing to converge at $1$, and it can't simply converge to $1$, it needs to converge fast enough so that the product of these probabilities converges to something greater that $0$. While that, in itself, is a very strong claim, it's not enough. We need to believe in an incomparably stronger claim, namely the total probability of an uncountable set of peculiar universes to be greater than $0$.

\subsection{Axiom \ref{noreason}}

I think that believing in one of the strongest claims one could make, i.e. that there is a chance of $a$ being peculiar, without there being any reason for that, and without us having any reason to believe it, is a very good example of an irrational belief.

To take an example, let us assume that we, at some point in the future, somehow find out the perfect mathematical model for the universe and that we know that it is perfect and that there is no simpler model. Let us assume that this model (or any equivalent one) has a free parameter which could be any real number.

Given all these, I think that it is completely unreasonable to believe that this free parameter is (say) an integer without it being determined, or at least influenced, from outside the universe. To generalize, since the set of real numbers which we can uniquely identify without using the "is the value of a certain universe parameter" property is peculiar, we can be fully confident that this parameter would not be in this set. Instead, it would be part of this set's complement, which is infinitely larger and contains almost all real numbers.

%I think that believing in an extremely strong claim, i.e. that there is a chance of $a$ being peculiar, without there being any reason for that, and without us having any reason to believe it, is a very good example of an irrational belief. It could only be better if we would have reasons to believe otherwise.

%But we do actually have reasons to believe otherwise. While there are better examples, consider that there are people claiming that they had a revelation about our world being created. One may believe that the probability that these people are lying is (very) high, but I don't think anyone can reasonably believe that this probability is $1$.

\subsection{Axiom \ref{ourplausible}}


\section{On models for our universe.}

There are a few options: finite vs infinite universe, created vs non-created, complex everywhere vs complex only far away vs simple.

1. If finite allows continuum complexity, then our universe is created.

2. If finite does not allow continuum complexity, then, on the non-created branch, a universe can only be infinite and infinitely complex. By default there is nothing saying that this complexity should be close to us.

\section{On models for our universe - 2}

I need to show that the hypercube can have a continuum of optimal descriptions. This is not trivial, especially since the size of the hypercube may be smaller than the measuring error.

I need to show the same for the entire universe. In the following, I'll assume that I did.

On the other hand, if a finite hypercube can have only a countable set of optimal descriptions for a given approximation level, then finitness implies that our universe is designed, so either our universe is infinite, or it is designed.

Also, an infinite observable universe with a finite observable description for a given approximation level implies design, so either our universe is infinite and has an infinite observable description for any approximation level, or it is designed.

Let us assume that it is infinite, with an infinite observable description. We can split it into countable hypercubes of equal size. Obviously, we'll have a countable set of descriptions for them. We'll take any reasonable probability distribution on them. These hypercubes could be arranged in any way, and we have a continuum of arrangements. That's likely to not be useful here. However, let us take $pl$ to be the probability of having a hypercube of size $l$ that approximately follows the laws that we see. Let us assume that we split our space into hypercubes, and that we have the directions up, down, left, right, front, back. Taking any such hypercube, the probability of it following our laws is $pl$. If we take $n$ hypercubes, the probability of all following our laws is $pl^n$. So the probability of our universe not being created converges exponentially at $0$.

\section{Larger region than needed}

Wasn't this already handled above?

\section{What happens when a hypercube does not have a continuum of possible optimal descriptions}

TODO: this example is about infinity, not continuum. One can have an infinite number of descriptions, but it's less clear that this is compatible with our life. As an example, let us consider a hypercube that periodically emits light for half of the period, then stops for the second half. The light intensity is, in principle, unlimited, but, arguably, above a certain value, depending on its distance to us, our life may become impossible.

While this example does not mean in itself that it's impossible to have a continuum of descriptions,

If the universe is infinite, and the same set of rules stands, then one can make a similar argument as the above (TODO: expand). But we can't prove that the univers id infinite.

If only a finite region of space around us has the same laws, then we can take a probability distribution over the set of laws and we can say that for each hypercube (i.e. cubic milimeter-milisecond) ouside the area essential for our life there is a probability $p$ of having the same set of laws as in our region. There are two questions here: 1. Is $p$ really a constant? 2. What is the value of $p$?

We don't have any reason to believe otherwise, so, by default, $p$ shuld be a constant, with a value of $\frac{1}{possible optimal description count}$. Then, although we can't compute it since we don't have a prior probability, the probability that there is a Creator increases extremely fast with each hypercube that has the same approximate set of laws.

However, instead of assuming that we have a constant probability, let us see if there is a deeper reason for that.

First, since the set of hypercubes that we can observe is finite, let us number then from $1$ to $N$.

Let us denote by $L$ the number of hypercubes that have the same set of laws as the ones in which we live. This is a constant of our universe. We should model it with a number randomly distributed in {0, .., N}. $P(L=n)$

Hm.... If there is no universe above ours, there is no reason for hypercubes to be coordinated, so, looking from ouside, we should expect a generic probability $p$ that one of them has the same set of laws as our hypercubes. Since $p$ is generic, $p$ is not $1$. Therefore the probablilty of our universe, if not designed, is less or equal to $p^N$

\end{document}
