\documentclass[a4paper
,draft
]{article}

\usepackage{amsmath}
\usepackage{amsthm}
\usepackage{appendix}
\usepackage[english]{babel}

 %%%%%%%%%%%%%%%%%%v
\usepackage{combelow}
\usepackage{hyperref}
\usepackage[utf8]{inputenc}
\usepackage{newunicodechar}

\usepackage[
    backend=biber,
    style=apa
%    citestyle=authoryear,
%    citestyle=alphabetic,
%    sortcites=true,
%    style=authoryear
%    style=alphabetic
    ]{biblatex}

\DeclareLanguageMapping{english}{english-apa}
\addbibresource{design-argument.bib}

\newunicodechar{Ș}{\cb{S}}
\newunicodechar{ș}{\cb{s}}
\newunicodechar{Ț}{\cb{T}}
\newunicodechar{ț}{\cb{t}}

\title{A Simple Universe Argument}
\author{Virgil Șerbănuță\thanks{\href{mailto:design-and-chance@poarta.org}{design-and-chance@poarta.org}}}
%\date{June 2015}
 %%%%%%%%%%%%%%%%%%^


\usepackage{amsfonts}
\usepackage[obeyDraft]{todonotes}
\newcommand{\svn}[2][]{\todo[author=Virgil,color=red!25!white,#1]{#2}}
\newcommand{\tsf}[2][]{\todo[author=Traian,color=green!40!white,#1]{#2}}
\newcommand{\tsfgata}[2][]{\todo[author=Traian,color=blue!40!white,#1]{DONE - #2}}
\newcommand{\commentfootnote}[1][]{}

\def\infordinala{\omega}
\def\infordinalb{\omega_1}
\def\reale{\mathbb{R}}
\def\intregi{\mathbb{Z}}
\def\complexe{\mathbb{C}}
\def\naturale{\mathbb{N}}
\def\rationale{\mathbb{Q}}
\def\descriptions{\mathbb{K}}
\def\designer{\mathbb{D}}
\newcommand{\paper}[1]{paper}
\newcommand{\multime}[1]{\left\{ #1 \right\}}
\newcommand{\definitie}[1]{\textbf{#1}}
\newcommand{\ghilimele}[1]{``#1"}
\newcommand{\lnotat}[1]{\sim #1}
\newcommand{\negat}[1]{\sim #1}  % TODO: This is the same as \lnotat
\newcommand{\citare}[1]{(\cite{#1})} % TODO: Separate this in a version with parenthesis and one without.

\newtheorem{definition}{Definition}
\newtheorem{afirmatie}{Claim}
\newtheorem{notatie}{Notation}
%\newtheorem{theorem}{Theorem}[section]
\newtheorem{lemma}{Lemma}
\newtheorem{axiom}{Axiom}
\newtheorem{note}{Note}

\begin{document}

 %%%%%%%%%%%%%%%%%%
\maketitle
 %%%%%%%%%%%%%%%%%%

 \begin{abstract}
  This \paper{} presents an argument that one can make a prediction from
  the hypothesis that our universe is not designed, i.e. that it has a high
  level of complexity of a certain kind, and that this complexity would be
  easily observable everywhere. However, this is not what we observe, which
  falsifies the hypothesis. \svn{Make sure I actually made a prediction.}
\end{abstract}

\section{Introduction}

Many people believe
that there are certain laws approximating our universe's behaviour fairly well,
that we can compute its age,
that we can make predictions about the distant future, and so on.
In a way, all of these are really surprising
\svn{Maybe move the Feinmann quote from below to here},
so this \paper{} started as a
mathematical attempt to understand why this is the case
and what we can reasonably believe about our universe.
It turns out that it matters
whether our universe is designed or not, so it became a \paper{} about that.

In the end, the mathematical part turned out to be rather small,
containing only a few simple properties about set cardinalities,
probabilities and ordinals.
I think that the non-mathematical ideas in this paper
have an intuitive appeal even without the mathematical ones,
but presenting them separately would make it less clear why certain
conclusions can be drawn.
Although, in my opinion, the non-mathematical ideas are also somewhat obvious,
I did not manage yet to find anyone drawing the same conclusions
in the same way.

My argument can be considered a statistical approach to Aquina's fifth way.
It tries to avoid the issues that other statistical approaches for showing
our universe's design (e.g. the fine-tuning argument) have.

Sections \ref{probabilities-section} and \ref{sec:ordinals}
briefly present some mathematical background for people interested in it.
Section \ref{sec:possible-universes} introduces universe approximations
and a few other related notions.
Section \ref{sec:axioms} introduces the argument's axioms,
while section \ref{sec:valid-options} uses them to make predictions
about our universe, then verifies if they match our observations.
Sections \ref{sec:ordered-universe} and
\ref{sec:fine-tuning} present this paper's relations to
previos work.

\section{The ordered universe argument}
\label{sec:ordered-universe}

%TODO: Urmatoarele trei alineate sunt nasoale.
The \svn{great?} Catholic theologian Thomas Aquinas,
in his \ghilimele{fifth way}, attempts to show God's existence from
the order of the universe, i.e. that almost all bodies, almost always
behave according to simple natural laws. %TODO: citation.
One can find a good exposition of this argument in
\citetitle{swinburne1968}%\citeauthor{swinburne1968} \citeyear{swinburne1968}
\parencite{swinburne1968}, but let us look at a few ideas which are
interesting in the context of this \paper{}.
%Metaphysics by Peter van Inwagen, from 2015, says that it's a powerful and sophisticated defense.
% 67 de citari https://scholar.google.ro/scholar?cites=18062876017816040453&as_sdt=2005&sciodt=1,5&hl=en

There are two types of order which we may consider, the spatial order
and the temporal order.
The former is the order that can be seen in (a part of) the universe
at a given moment in time, e.g. that planets, living bodies, and other things
are ordered.
The latter can be seen in the behaviour of things, including the laws of
nature, and it's the one that will be used in this \paper{}.

\citetitle{swinburne1968} addresses Hume's
objections \svn{Need to introduce them. * Re-read The Argument From Design}
against Aquinas' argument, but one in particular is interesting here:
the argument is based on analogy, which limits its strength.
I think that this \paper{}'s argument, although it builds on the same
foundation, needs only an extremely weak analogy
(see Section \ref{sec:design-probability}).

The following quote from \textcite{swinburne1968}, made when addressing
Hume's objection that the order which can be observed in this universe
is just an accident, makes a nice introduction for the argument
described in this \paper{}:
\begin{quote}
But if we say that it is chance that in 1960 matter is behaving in a
regular way, our claim becomes less and less plausible as we find that in
1961 and 1962 and so on it continues to behave in a regular way. An appeal
to chance to account for order becomes less and less plausible
the greater the order.
\end{quote}

However, there were many people who were surprised by the existence of laws
for our universe in one way or another and who didn't try to use it
as an argument for the existence of a Designer.
As an example, see this quote of \textcite{Feynman2009}:
\begin{quote}
Incidentally, the fact that there are rules at all to be checked
is a kind of a miracle; that it is possible to find a rule,
like the inverse square law of gravitation, is some sort of miracle.
It is not understood at all, but it leads to the possibility of
prediction --- that means it tells you what you would expect in
an experiment you have not yet done.
\end{quote}

% TODO: Maybe list some of Hume's issues with this argument.

\section{The fine-tuning argument and other objections}
\label{sec:fine-tuning}
\svn{
  Consider deleting this section or moving it to the objections section or
  correcting it.
}

Although the argument in this paper does not resemble the fine-tuning argument
much, it uses similar methods, which may raise similar objections. Indeed,
when writing this paper I had to carefully avoid these objections, so readers
may find it useful to have a short summary of the fine-tuning argument and its
objections which are also relevant to this paper, together with an explanation
on how they were avoided.

In the recent past people started noticing that life can exist only for
narrow ranges of various parameters of our universe, so our universe
seems fine-tuned for life.
But that seems unlikely unless there is a designer for our universe,
a designer that wanted it to have (intelligent) life.
Examples of such parameters include the early cosmic density and
the expansion speed, the strength of the weak nuclear force, and
the ratio between electromagnetism and gravity \parencite{Leslie2003}.

Although this may seem intuitively reasonable, people have found various
potential problems with this argument, some of which are presented below.
For possible replies to some of these in the fine-tuning context see,
e.g., \textcites{Leslie2003}{Swinburne2003}{Monton2006}{Kotzen2012}.
% TODO: Cite \ghilimele{Should we care about fine-tuning} by Jeffery Koperski.
% TODO: Actually check if these issues apply to my argument.

Probabilities do not apply to the values of the cosmic parameters:
if one assumes that all cosmic parameter ranges have the same probability
then the probabilities become non-normalizable and the argument
would apply to any finite range (coarse-tuning).
If the cosmic parameter ranges do not all have the same probability,
then we must consider many possible probability distributions.
For some of them the life-permitting ranges have low probability,
for others they have high probability and we don't have a good way
of choosing one of them \parencites{Manson2009}{McGrew2001}.

Indeed, the most reasonable assumption is that they all have the same
probability, i.e. $0$, which implies that, if our mainstream physics theories
provide a reasonable approximate model for our universe, then the probability
that our universe is not designed is $0$.

The current argument also analyses the other case, having unequal probabilities
for individual events, where, in order to believe that our universe is not
created, one needs to either ignore the evidence that we have about our
universe's consistency, or believe in a rather weird and implausible structure
of meta-universes, without having any proof for its existence.

There is an observation selection effect:
we can't observe a universe which is not life-permitting,
so our observations are biased.
If we take this additional piece of information into account,
the probability of our universe not being designed is not longer
small \parencite{Sober2009}.
This is particularly interesting if multiple universes
exist \parencite{Manson2009}.

All universes considered in this paper contain intelligent life. Even
more, the next source of bias, those beings' perception, was taken into account
by describing universes as they are seen by the said beings.

There is a third possible source of bias which is considered here: it can be
argued that, in a non-created universe, beings might be intelligent
only if their intelligence is useful to them. But this likely means that those
beings live in a timespace region which seems consistent from their point of
view. This is also addressed in the argument: we observe non-essential
consistency, which can only be explained by design.

The multiple universes objection raises an interesting issue in the context
of the current argument.
When multiple universes exists, the probability of an intelligent being
living in a universe having a finite observable description is still $0$, or
it needs the same implausible assumptions as the case of a single universe.
In virtually all universes this argument\svn{axiom ??? instead of
\ghilimele{this argument}} would not work, since the infinite complexity
of the universe would be easily observable.
For the intelligent beings living in the other universes, as long as they have
a hypothesis which is more plausible, it would be irrational to think that
their universe is not designed (assuming that the argument presented in
this \paper{} is correct).

Non homogenous spacetime:
another possible case similar to the above is when our universe
is much larger than the part that we can observe and, outside
of the part that we can observe now, it contains spacetime regions
with different values for the various cosmic parameters \parencite{Manson2003}.

The current argument takes into account the fact that different
spacetime regions can have different descriptions. Indeed, the fact that this
is logically possible is used to show that the timespace region of our universe
in which we live is implausible in a non-created universe.

Unknown Designer's intent:
from a natural theology point of view, one can't
know what the Designer wanted
(e.g. one can't know that a universe designer would want to create
a universe having life) \parencites{Sober2009}{Narveson2003}, so, by default,
any argument showing that the probability of our universe is small
if it's not designed would also show that the probability is small
even if it is designed.
To fix this, one would need an independent way to show that the Designer
wanted the universe to have life \parencite{Sober2003}.

Although this argument is still based on analogy, note that the weakest
analogy that still has some trace of plausibility is enough, i.e. we just need
that \ghilimele{a Designer of world wants intelligent beings} has non-zero
probability. \svn{Rephrase this - the counterargument has nothing about
analogies. Maybe add something about revelations.}

Design criteria: minds may be the product of natural laws,
but we intuitively distinguish something natural from something
designed by checking if their production involves a mind or it is only
a result of natural laws.
There is no similar way of telling if the creation of the laws of nature
themselves is natural or not \parencite{Narveson2003}.
\svn{Think carefully about this. This is not a FT counter-argument and
I should make sure that I'm understanding this correctly. Think carefully.}{}

I will rephrase this as: we distinguish designed object by being much
more likely to be produced by processes involving minds than by natural laws.
As an example, with our current understanding of physical laws, a watch could
materialize out of thin air, or it could happen that its various pieces are
the product of chance, and they came together by chance,
but that's way too implausible, while it is more likely that any watch that we
observe was designed.

In the same way, does not show that the non-design hypothesis for our universe
is logically impossible, it just shows that, under any plausible
\ghilimele{natural laws}, our universe is much less plausible than if it was
designed.

\svn{Should this be here?} Note that, under the design assumption, one may have
information from outside the universe about the designer's intent. \svn{A good
question is how can we tell that from information from inside the universe.}

\svn{10} In order to avoid this objection, the current argument works with
conceivable universes.

First, let us note that using conceivable universes makes sense. If we are
trying to make a predictiona about how a designed / non-designed universe would
look like, we would have to imagine that we are outside any universe, and we are
about to observe one that is (say) non-designed. What could we say about it?
Not much, maybe that it makes logical sense (i.e. that it can be modelled
mathematically), maybe it has something cloase to time and space, but
everything else is possible.

As a parenthesis, one could go one step further and, by using the same proof
structure as used in this paper, show that, if there is no designer of worlds,
then the hypothesis that only our universe exists is a peculiar one, so it is
extremely unlikely.

There is a second reason why we should believe that many different universes
exist: \svn{11} \svn{12} \svn{13} - Do I need this? Does it help?

\textcite[][Section \ghilimele{Why a world with human bodies is unlikely
if there is no God}]{Swinburne2003} comes sort of close to the argument
presented in this \paper{}, but while \citeauthor{Swinburne2003}
argues that human bodies are unlikely, I am arguing that, in the context
of all possible universes that could have human-like beings,
our universe is extremely unlikely.
Since the existence of human bodies is not directly related to the subject
of this \paper{}, I will not discuss that section more
than it is strictly needed.
Note that  \citeauthor{Swinburne2003} says that individual sets of laws
have non-zero probability while I'm claiming that their probability is $0$.
It seems to me that \citeauthor{Swinburne2003} implicitly assumes that
such a set has a finite number of laws, while I am explicitly removing
that constraint, so both can be right within their contexts.

\section{Possible universes and their descriptions}
\label{sec:possible-universes}

If our universe is created, then it's likely to be the way it is because
its Creator wanted it to have certain properties.
In order to understand why our universe works the way it does,
one would need to understand the intent of its Creator.
While that is interesting in itself, I will not try to pursue it here,
except for a few limited ideas.
% TODO: use here the \ghilimele{design implies non-continous probabilities idea}

For most of the remainder of this \paper{}, let us consider the other case.
Let us assume the hypothesis that our universe is not designed and created
and let us try to make a prediction based on it.
How would a non-created universe look like?
Would it be similar to our universe?
Maybe an infinity of universes exist and ours is just one of many,
or maybe our universe it the only one that exists.
Even if ours is the only one, one could easily imagine that it worked
in a different way, e.g. maybe some constant like the speed of light would be
different, or maybe gravity would work differently.

There are people who claim that all logically possible universes exist,
either because they think that it simply makes sense, or because they want to
give a good account of modality, or because they need
an ad hoc\svn{Do I need the \ghilimele{ad-hoc} part?}{} hypothesis
to argue against, say, fine-tuning.
If that's the case, it seems, at first sight, that making predictions about
non-designed universes is rather hard.
However, this paper argues that there are certain things that can be said
regardless of how many universes exist.
The argument in this \paper{} does not assume that more than
one universe exists, but it should work either way.

A possible universe
could have exactly the same fundamental laws as ours, but with matter
organized differently.
It could have similar laws, but with different universal constants.
It could have different fundamental particles (or fields, or whatever the basic
building blocks of our universe are, assuming that there are any).
Or it could be completely different, i.e. different in all possible ways.

It could be that our logic and reasoning are universal instruments,
but it could also be that some of these possible universes are
beyond what our reasoning can grasp and others have properties
for which our logic is flawed.
Even if that's the case, let us see if we can say anything about
the possible universes that we could understand and could model in some way.

In the following, the \definitie{possible universes} term will denote
only the possible universes which we could model.

This notion of model is not precise enough.
Let us restrict the possible universes term even more,
to the possible universes that could be modelled mathematically,
even if that may leave out some of them.
This may seem too restrictive,
especially since this paper only needs universes which can be approximated
by mathematical models.
However, for simplicity, let us, for now, consider only universes
which are fully modellable with sets of axioms that are at most countable.
\svn{Do I need this?}

Let us restrict the universes we are considering even further, to universes that
have something remotely resembling time and space, for which
\ghilimele{the state of the universe at a given moment in time}, or something
close, makes sense, and which can plausibly contain intelligent beings.
Any such universe is, for the purpose of this paper, a conceivable universe.

To keep the exposition simple, in the following I will use
\ghilimele{the state of the universe at a given moment in time},
but one should replace it with one's favourite alternative concept.

Let us define a \definitie{universe description} to be a mathematical
theory that has
a set of axioms which is at most countable and which allows making
predictions about the future state of the universe given its state
at a certain moment in time. A \definitie{universe region description}
is something similar, but only for a given space-time region of a universe.
In the best case, for a deterministic universe, there might exist
a description which allows one to correctly predict the entire future state
given the state at any moment in time, but a universe description
as defined here does not have to predict everything and,
even when it predicts something, it does not have to always be correct.

Note that the data available for making predictions is dependent
on who is making the prediction. As an example, if we assume that
all predictions are about things that can be perceived, then
each kind of intelligent beings (e.g. humans) will make predictions
about the universe projected through their senses. If a universe contains
multiple kinds of intelligent beings, with different kinds of
sense organs, then that universe may have descriptions which are
very different.
Of course, things that are not observable directly can sometimes be mapped
to things that are observable, but this may not be always true.

In order to handle this dependence on who observes the universe
in a reasonable way, in the reminder of this paper we will work with universes
that contain intelligent beings
and all predictions will be relative to what these intelligent beings
could observe.
If there are multiple kinds of intelligent beings in the
universe, when talking about its description, we'll assume that we picked
one such kind.
I.e. although we
will talk about universes and their descriptions, we'll actually mean
(universe, intelligent-being-kind) pairs and their corresponding descriptions.

Next, let us try to specify how good an univese description
should be. First, our predictions speak about the future, but expecting them to
predict everything until the end of the universe (if any) may not be
reasonable. We may want to fix a minimum amount of time $\Delta t$
and to evaluate the predictions about things that are at most
$\Delta t$ in the future. Second, we should't expect to
be able to describe everything with full precision, so we may want to
have a precision $\eta>0$ for all the values that we predict.
Third, we shouldn't expect our predictions to always be correct, so
we should require that they are true with probability $p>0$.
The exact meaning of \ghilimele{true with probability} here is left open.
Of course, we may add other similar constraints if needed.

Then let us say that an \definitie{approximate universe description} with a
\definitie{level of approximation} ($\eta>0$, $p>0$ and $\Delta t>0$)
is a universe description which allows one to approximate the future
state of the universe with a precision $\eta$, with a probability
$p>0$ for a prediction to be correct
\svn{Do I need the probability of the prediction to be correct
     or do I need to cover a fraction p of the world?}
and for a limited amount of time $\Delta t$.
\svn{I must answer the following questions: exactly what does that
     probability of correct prediction mean? Is it an aggregate among
     all possible predictions? Does it mean that all of them need a p
     chance of being correct? Do I need a fraction of the universe here?
     All writable predictions are countable, so they all have a non-zero
     weighth, so the aggregate probability most likely can't be zero.
     Am I talking about provable predictions or about predictions
     which are true on models?}

There is a distinction that we should make.
When predicting (say) weather we can't make long-term precise predictions,
and this happens because weather is chaotic, that is, a small difference
in the start state can create large differences over time.
This could happen even if the universe is deterministic
and we know the laws of the universe perfectly, as long as we don't know
the full current state of the universe.
However, high precision predictions may be possible if the full state
is taken into account.

For a given universe or region of a universe,
given a level of approximation, we will pick a canonical description
in the following way: Let $S$ be the set of descriptions which approximate
the universe with the given level of aproximation. If $S$ contains
at least one finite description, then we pick the shortest such
description as \ghilimele{the canonical description}, breaking ties by
using the lexicographic order. Otherwise, we simply say that the
universe (region) has an infinite description, and we will abuse the
terminology a bit by picking $S$ as the canonical description (we could
also pick a random description from the set).
If the level of approximation is obvious from the context, we will call
this canonical description \definitie{the universe's description}
or the \definitie{universe region's description}.

One could use a well-ordering on the real numbers to choose the
lowest description as the universe's description, but that will
complicate things without any benefit.

\section{Options for our universe}

The reminder of this paper will analyze what we can reasonably believe about
the following issues:
\begin{itemize}
  \item Our universe is created or not.
  \item Our universe has finite or infinite space.
  \item Our universe has finite or infinite time.
  \item Our universe has a finite or infinite description.
  \item Option 1: There is a meta-$\beta$ universe for each countable ordinal
        $\beta$ such that our universe is the meta-$0$ one and the meta-$beta$
        universe includes, directly or not, all meta-$\alpha$ universes with
        $\alpha < \beta$. Option 2: there is an ordinal $\beta$ where this
        stops being the case.
  \item Life is or is not compatible with a finite chunk of space-time having
        an infinite observable description.
\end{itemize}

\section{Axioms}
\label{sec:axioms}

Throughout this paper we will only use probability measures that can measure
singletons (i.e. single-element sets).

\subsection{Observing events}
\svn{Add the definition for "generic" and add a reference to the section + peculiar}
\begin{axiom}
  If $P$ is a probability over the set of real numbers
  (or a set with the same cardinal)\footnote{Readers of this paper should
  keep in mind that, in most cases throughout this paper, what is being said
  about the set of real numbers is similarly valid for any set with the same
  cardinality.}
  which can measure singletons,
  we observe $x\in\reale$, and $P(x)=0$, then $x$ is generic.
\end{axiom}

Note that here, and in all the axioms in this paper, it is not required that
$P$ is be a probability over the
Borel algebra of $\reale$, although that is, in many cases, implicitly
assumed when talking about probabilities over $\reale$.

The set of events for which $P(y)$ is
greater than $0$ is at most countable, and, if we remove them from $\reale$, we
get a set with the same cardinality as $\reale$.
On this later set, the probability
of all peculiar events taken together is $0$, so there is no chance of us
observing one.
In other words, the probability of all generic events is $1$,
so we can be sure that we observed a generic event.

Of course, the (logical) possibility of observing a peculiar event still exists,
but, practically, we will not observe it as long as the set of our
observations is at most countable.

This axiom could be generalized to probabilities for which singletons are not
measurable, in those cases where we have a set of disjoint measurable events
with the same cardinality as $\reale$, but, for the purpose of this paper,
the current version should be enough.
\svn{Keep or drop this paragraph?}

\begin{axiom}\label{noprobability}
  If we observe $x\in\reale$, when we could have
  observed any real number, and there is no probability distribution that could
  describe how $x$ was chosen, then $x$ is generic.
\end{axiom}

Note that this axiom does not say that we do not know that probability
distribution, it says that there is no such probability distribution.
Anyone believing that this cannot happen should treat the cases where
this axiom applies as invalid.

Also note that this cannot happen when using subjective probabilities.

If there is nothing that could favor
peculiar numbers over generic ones, it's absurd to think that we could have
observed an element of such a tiny set among something infinitely larger.
Also, the case with a probability above suggests that this is the only
reasonable assumption in this case.

\subsection{Design probability}
\label{sec:design-probability}

\begin{axiom}
  The probability that our universe is designed is greater than $0$.
\end{axiom}
\svn{quote/cite Swinburne or \ghilimele{This section parallels Swinburne}}

In this section we are interested only in whether a probability is $0$ or not,
so the word \ghilimele{likely} can be replaced with
\ghilimele{has a non-zero probability} without decreasing
the validity of the argument.

First, let us note that something that is designed follows the intent of
its designer. A designer may want to design everything possible, but then
this shouldn't be properly called \ghilimele{design}, since it does not involve
choice.

A designer is likely
to divide all the possible things in classes of similarity, and design at most
a small number of things from each class (or at most one per class for a very
good designer). In other words, between possible designs there will be small
differences that do not really matter, so a normal designer will focus on the
important parts and would not make explicit choices for each of these small
diferences. And a very good designer would optimize and make the best choice
even for these small things that do not matter much, but these
would not cause more things to be designed.

Assuming that we have some classes of similarity which may be interesting for
a given designer, it is acceptable to believe that their set is at most
countable, and, if so, that we have a positive probability for each of them.

Extrapolating from what we know about ourselves, although that may
not give us much, it is still acceptable to believe that
finite rational beings,
capable themselves of design, would be interesting to a designer of worlds.
Therefore, we can assume that it's not completely implausible that a designer
would design a world in order to be inhabited by the said beings.

But then, that world is likely to be simple enough that
those beings would find it possible to use their reason to
discover, navigate and interact fruitfully with their world.
\svn{Hidden assumption that those beings are finite.}

First, something close to this issue was already addressed by
\textcite{Swinburne2003}, which argues that, if God exists,
there is a fairly good chance that humans can understand their universe:

\begin{quote}
  So, in order to have significant freedom and responsibility, humans need
  at any time to be situated in a \ghilimele{space} in which there is a
  region of basic control and perception, and a wider region into which
  we can extend our perception and control by learning which of our
  basic actions and perceptions have which more distant effects and causes
  when we are stationary, and by learning which of our basic actions cause
  movement into which part of the wider region.
  If we are to learn which of our basic actions done where have which
  more distant effects (including which ones move us into which parts
  of the wider region), and which distant events will have which basically
  perceptible effects, the spatial world must be governed by laws of nature.
  For only if there are such regularities will there be recipes for changing
  things and recipes for extending knowledge that creatures can learn and
  utilize.
  So humans need a spatial location in a law governed universe in which to
  exercise their capacities, and so there is an argument from our being thus
  situated to God.
\end{quote}

However, a designer designing a universe for intelligent beings would likely
make it intelligible to those beings, so it's likely that most of
the space and time accessible to those beings is fairly consistent.

But then, it is possible that the entire universe has a finite approximate
description.

To summarize, it is reasonable to assume that peculiar descriptions have
a non-zero probability when a Designer is involved.

\svn{5. Use Swinburne's argument about why would God create human-like beings.}

\subsection{$\reale^4$ universe}

\svn{Add axioms for:

* anything physical that is part of our universe, including empty space, can
  have a set of real-number coordinates attached to it that approximates its
  location (how well?). - this should be included in the things below. Maybe
  add it as an explanation.
}

\svn{Kind of redundant. Do I need all of these axioms?}

This set of axioms attempts to say that our universe can be modelled
on top of something close to $\reale^n$.

Let us define a \definite{generalized rational number} as being either a
rational number, or one of $-\infty$ and $+\infty$.

\begin{axiom}\label{unitycovering}
  For all $q\in\rationale\cup\multime{-\infty, +\infty}$,
  the subset of hypercubes of size $q$ and corners
  having generalized
  \svn{Check that generalized rationals are needed, or we can use
       $q\in Q$}
  rational coordinates is countable and covers our universe.
\end{axiom}
\svn{Do I Need this axiom?}

This axiom states that our universe can be covered by a countable
set of hypercubes of a given size.
While this could certainly be false, it is implied by some
basic beliefs that we have about our universe, e.g. that there is a
level of approximation for which the space and time anywhere in the universe
can be modelled locally by an open subset of $R^n$.
\svn{More explanation}

\begin{axiom}\label{rationalcovering}
  The set of (generalized) cuboids having their corners' coordinates in
  $\rationale\cup\multime{-\infty, +\infty}$,
  is countable and covers our universe.
\end{axiom}

If our universe's spacetime is $\reale^\alpha$ with
$\alpha$ finite, then, indeed,
the set of cuboids in the above axiom is countable.

\begin{axiom}\label{finitecovering}
  Any finite part of our universe can be covered with a finite set of cuboids
  having their corners' coordinates in
  $\rationale$.
  \svn{Could be a definition.}
\end{axiom}

\svn{Generalize: a countable set of coordinates covering the space,
  coordinates become finer and finer.}

\subsection{Logically possible universes}

\begin{axiom}\label{uncountable}
  For any level of approximation $L$ there is a set $D_L$
  of universe descriptions such that the following are true:
  \begin{enumerate}
    \item $D_L$ has the same cardinality as $\reale$.
    \item For all elements $d\in D_L$ there is at least one conceivable
          \svn{possible?}
          universe $U_d$ which
      \begin{enumerate}
        \item has a time-space or something similar enough;
        \item can plausibly contain intelligent beings that use mathematics;
        \item for the intelligent beings mentioned above and
              for the level of approximation $L$,
              $d$ is a description of $U_d$.
      \end{enumerate}
    \item $D_L$ contains our universe's description.
          \svn{Do I need to make such a strong claim?}
  \end{enumerate}
  The same is true for universe regions, excetp that $D_L$ may have a lower
  cardinality.
\end{axiom}

This axioms states that, for a given level of approximation,
there is a large set of conceivable universes, which in some narrow respects
are similar to ours, but which are, in general, wildly different.
Also, our universe belongs to this set.
\svn{Mention that all countable sets of axioms form a continuum.
I did that below, but it's a bit unclear.}

To see why that is reasonable, let us first note that,
%in a similar way to axiom \ref{nonessentialhypercube},
most likely, we have an approximate description for our universe
given by classical mechanics, maybe with some additions.
Alternately, one could use a description based on, say, quantum field theory.

Next, for almost any countable axiom system that still
has $R^n$ as a base, one could imagine an alternate universe
which is exactly like ours inside (say) Mars' orbit,
but what is what is outside of Mars' orbit is described by that axiom set.

Some of these axiom sets would describe laws of nature which are similar enough
to ours to allow us to observe what happens outside of Mars' orbit,
but different enough that
we would notice (e.g. gravity could work differently, depending on the region
of space in which one travels).

For any approximation level $L$, and any region that is not trivial
for $L$
(the meaning of \ghilimele{trivial} is left open, but, as an example,
if we use an approximation level that does not measure things larger
than a size $l$, then the region must be significantly larger than $l$),
there are multiple possible descriptions that
are distinct for $L$ (not defined!). By splitting an infinite timespace
into disjoint regions defined by a finite set of rational coordinates
and taking all possible ways of assigning descriptions to these regions
we get a set of universe descriptions with infinitely countable axioms
which we will denote by $\descriptions$.
Each of the conceivable universes
having one of these descriptions could contain intelligent beings that use
mathematics, and so on.\svn{It's not about axiom sets, but about axiom sets
that also include our region. Actually it's not about those axiom sets, but
about their equivalence classes. Either reword everything, or, if choosing the
lowest description soves the issue, make an explicit comment about it.}

\subsection{Neighborhood modelling}

\svn{Move before the previous section.}
\begin{axiom}\label{finiteneighborhood}
  There is a large time-space region of our universe that includes our solar
  system, which has a finite approximate description.
\end{axiom}

Above I argued that, for non-created universes, the probability of having
a non-trivial finite approximate mathematical description is zero,
i.e. $P(F | \negat{C}) = 0$.
As mentioned above, many people assume, implicitly or explicitly,
that our universe has such a description, e.g. when claiming
that the universe is around $14$ billion years old.
This means that they assume that $P(F) > 0$, which implies that $P(F | C) > 0$.
It also implies that the probability of our universe not being created is $0$,
i.e. $P(\negat{C} | F) = \frac{P(F | \negat{C}) P(\negat{C})}{P(F)} = 0$.
\svn{
    Think about deterministic vs non-deterministic modelling.
    Think about overlapping approximate descriptions - I can't have
    equivalence classes because a description that works for
    a fraction of the universe can be equivalent to two incompatible
    descriptions for the other fraction. Does the explanation in
    \ghilimele{Logically possible universes} still work?
}

\section{Valid options for our universe}
\label{sec:valid-options}

This section will try to develop the axioms above in order to find out what
is reasonable to believe about: our universe being designed or not;
space and time being finite or infinite; the laws that we use to describe
the behaviour of the universe working only locally, around us,
or in the entire universe;
our universe being contained in an uncountable chain of meta-universes;
life being
compatible with a finite chunk of space-time having an infinite
observable description.

We will focus mostly on what happens when our universe is not created
since in this case it is easier to make predictions about our universe
and to falsify them.

\subsection{Finite Descriptions and Uncountable Meta-universes}
\label{fdaumu}

Let us assume first that our universe has a finite
description, which we will denote by $x$, with $x\in\descriptions$.
Let us also conside that, perhaps, our universe is contained
in a meta-universe, which is, perhaps, contained in
a meta-meta-universe, and so on, these meta-universes corresponding to
ordinals. We will assign the ordinal $0$ to the first meta-universe.

A meta-universe here is something that contains our universe, and which
influences one way or another which of the logically possible universes exist.
As an example, if all possible universes exist, we can take their
set/collection/grouping to be a meta-universe. For the purpose of this paper,
the most important thing related to a meta-universe is whether there is a
probability measure over the descriptions of the universes
contained in it, specifying which ones exist and in which proportion.
The existence of
such a probability implies the existence of a meta-universe, but not the other
way around. Alternately, this probability may be the subjective probability
which we assign to various universe descriptions.
\svn{Axiom 2 - limit to countable?}
\svn{
  De gandit la faptul ca regula ca toate universurile posibile exista e
  peculiar, dar, poate, rezonabila.
  Poate nu.
  Oricum, eu lucrez cu distributii de probabilitate.
  Pot da ca exemplu si comenta exact regula ca
  (fiecare univers posibil exista /
  pentru fiecare descriere exista un univers)
  si ca am sa nu distributie de probabilitate.
  De urmarit cum se aplica axiomele - in articol sau ca anexa.
}

One example of a probability measure is that which assigns $1$ to our
universe's description and $0$ to everything else.
However, if, say,
our universe is the only one that exists, but there is no reason for it
existing
\svn{Did I explain what it means?}
and for the other universes not to exists, then there is no
probability measure in the sense mentioned above.

Returning to associating meta-universes with ordinals,
if there are ordinals which do not correspond to any meta-universe, and
the smallest one is countable, let us denote it by $\alpha$.
Otherwise, let $\alpha$ be the lowest uncountable ordinal.

If $\alpha = 0$ then there is no probability distribution over universe
descriptions, so $x$ is generic (axiom \ref{noprobability}).

Otherwise, if $\alpha > 0$, but, at level
\svn{Did I say what \ghilimele{level} means?}
$0$, there is no probability
distribution over universe descriptions, $x$ is generic (again, from axiom
\ref{noprobability}).
\svn{Either use the proper wording everywhere, or explain that,
     for easier writing I am treating $P(X)=1$ that as
     logical necessity, but the
     reader should keep in mind that the actual meaning is closer to
     \ghilimele{practically true}.}

If the probability distribution at level $0$ is continuous,
then the probability of all peculiar descriptions is $0$,
so the only reasonable conclusion is that our universe has a generic
approximate description.
\svn{Axiom??}

For many people, this result may be good enough.
However, since this
probability distribution does not have to be continuous,
let us let us assume that for some levels $\beta$ with
$0 < \beta \le \alpha$ there is a probability
$p_\beta$ over $\descriptions$, with $p_\beta(y)$ being
the probability that $y$ is a discontinuity
for all $p_\gamma$ with $\gamma < \beta$.
\svn{Really confusing. Maybe a footnote with an example would help.}

Let $\beta$ be the minimum (if any) for which $\beta < \alpha$ and
$p_\beta$ does not exist or $p_\beta(x) = 0$.

% If such a $\beta$ exists, then let us denote by $P$ the joint probalility for
% all the ordinals up to $\beta$ and including it.
% Then $p_\gamma(x) = P(x|\gamma)$. We have $P(x|\gamma)>0$ for all
% $\gamma<\beta$ and $P(x|\beta) = 0$. Then $P(x) = 0$

If such a $\beta$ exists, then, although in all the lower universes
$\gamma < \beta$ we may be able to bet on a peculiar description for our
universe, at level $\beta$ we can be sure that a bet on peculiar descriptions
would fail.

Since, until $\beta$, we made an at-most-countable number of
observations\footnote{Each meta-universe between $\beta$ and us counts as an
observation}, we
do not expect to observe a peculiar event with $0$ probability, so we must be
in a chain of (meta-)universes where our universe's description is
generic.
\svn{+quote. Qhat did I mean by \ghilimele{quote}?}
% Since we only placed an at-most-countable number of bets until now,
%we don't expect certain failure.
%\svn{Why we don't expect failure?}

If such a $\beta$ does not exist, up to $\alpha$, without including it,
$p_\alpha$ exists and $p_\alpha(x) > 0$.

If $\alpha$ is countable and there is no meta-universe above $\alpha$,
then, similar to the above reasoning, as far as $\alpha$ is concerned,
$x$ is generic.

This means that, if our universe is not created, the only option that
has a chance of being reasonable is that $\alpha$ is uncountable and,
for all $\beta < \alpha$, $p_\beta$ exists and $p_\beta(x) > 0$.

To summarize, in order to claim that $x$ is peculiar, one needs to postulate the
existence of an uncountable chain of meta-universes, all of them favouring a
peculiar $x$, which, by default, is unlikely for any of them. Also, although the
current argument does not work anymore when the chain of
meta-universes becomes uncountable, intuitively the peculiarness problem
still remains.

From now on, I will assume that the possible objection in the preceding
paragraph is unreasonable, which means that, practically speaking,
either $x$ is not peculiar or our universe is designed.

Next, let us consider if a universe with finite space-time can have an
infinite description that is compatible with intelligent life. If this is not
possible, then either our universe is designed, or it has an infinite
space-time, or it has a finite spacetime with an infinite description.

\subsection{Peculiar descriptions for universe regions}

In this section, we will only work with cuboids that have their
corners' coordinates
in $\rationale\cup\multime{-\infty, +\infty}$, although we may limit
them at times to smaller sets (e.g. finite coordinates).

Let us take a countable covering with (generalized) cuboids of our universe.
From axiom \ref{rationalcovering}, this covering exists.
Let us take all finite unions of such cuboids. Their set is countable.

Let us consider the case when a finite chunk of spacetime can
have an infinite approximate
description which is compatible with inteligent life in the universe.
Let us take such a chunk of space and let us cover it with a finite union of
finite cuboids (axiom \ref{finitecovering}).
We can then apply the same argument as in section \ref{fdaumu} to any
such union of cuboids, by considering it as a mini-unverse inside
our universe.

The set of all possible finite unions is countable,
which means that the probability that any of them has a peculiar
description is $0$. The only reasonable options that we have are:
either each such union has a generic description,
or finite chunks can only have peculiar descriptions, or our universe
is designed.

However, we usually believe that we can have peculiar/finite
approximate descriptions
for the region of universe around us (axiom \ref{finiteneighborhood}), so we
will disregard the possibility that each finite union of cuboids has an
infinite approximate description. Even if we are unwilling to disregard it,
the argument in the next section also works when using the full set of
descriptions, not only the finite ones.

This leaves us with the following
two options: finite unions of cuboids that cover a finite chunk of space can
only have finite descriptions in a universe with intelligent life, or our
universe is designed.

However, from section \ref{fdaumu}, our universe should have an
infinite description, which means that either our universe is infinite,
or it is designed.

If our universe is infinite,
then finite unions of cuboids that cover an infinite
amount of time-space can have infinite descriptions. The set of all finite
unions of cuboids is countable.
Applying the argument
in section \ref{fdaumu}, we get that either each such union has a generic
description, or our universe is designed.

To summarize, either
\begin{itemize}
  \item our universe is designed, or
  \item our universe is infinite, all finite unions of
        cuboids that cover an infinite chunk of spacetime have infinite
        descriptions, while finite unions of cuboids that cover a finite chunk
        of space can have only finite descriptions.
\end{itemize}

\subsubsection{Improbable consistency}

Let us focus on the second case above, infinite universe with infinite
description, while finite unions of cuboids have finite approximate
descriptions.

Let $A_1$ be the set of possible approximate descriptions for
a hypercube of size, say, $1\;second \times meter^3$.
Unless our approximation is extremely coarse, $A_1$ will have multiple elements.
For any description $a\in A_1$, let $P(a)$ be the probability of encountering
a hypercube with $a$ as its description in our universe.

\svn{Finite union covering infinite space has infinite description
     iff
     infinite hypercube covering infinite space has infinite description
     iff ???}

Note that there are multiple ways of picking this probability. Examples
\svn{TODO: Remove the examples. They are at odds with the reminder of this
     section (subjective vs objective probabilities).
     }
include:
\begin{itemize}
  \item Take the entire spacetime that we could reach until the termic death of
        our universe and divide it into hypercubes of size $1$. This spacetime
        is, most likely, finite, so we can assign equal weights to each
        hypercube, and we can sum these for each description to get
        the probability of each description.
        \svn{Shouldn't one take the description probability among all universes?
             Have an option where this is a subjective probability
             }
  \item Abuse the term \ghilimele{encountering} and take any weight
        function over space-time divided in hypercubes of size $1$ that does
        not assign a $0$ weight to any hypercube such that all weights add to
        $1$, perhaps we would
        assign the maximum probability to \ghilimele{now and here},
        decaying exponentially over any infinite dimension. As above,
        we would add these weights to get the probability
        ************ of any description in the entire space.
\end{itemize}

If $A_1$ would be finite, we could, by default, pick the probability
distribution that assigns equal probabilities to all elements of $A_1$.
Howewer, $A_1$ can be countable, so let us consider how such a probability
distribution would look like.
It would be unreasonable to assume that the probability is zero for all
elements except one.
This is suggested by two things: first, there are too many
possible universes where this does not happen; second, right now we are
examining the case when our universe has an infinite description, so not all
hypercubes share the same description\footnote{
This may not be true when considering only descriptions
within a finite region of space-time, but it's still unreasonabe to believe
that there is only one description within that space}.

If the description that we use for the universe around us
has probability $p_1 < 1$, then the probability of observing $n$ hypercubes
with this description without observing any other description is $p_1^n$.
\svn{This assumes independence, but that's reasonable among all possible
     universes.
     }

Even if $p_1$ is very close to $1$, $p_1^n$ converges very quickly to $0$,
so the consistency of a small spacetime region around us is enough
to make this hypothesis unlikely enough (compared to the others, assuming
that at least one other has a non-zero probability) to disregard it.

TODO: Move to the right\footnote{
  If $N=2$, $k=10$ is enough to go below one to one thousand odds,
  $k=20$ is enough to go below one to one million, $k=30$ goes below one to one
  billion
  }
place\footnote{
  Let $ND$ be the hypothesis that our universe id not designed,
  $D$ be the hypothesis that it is designed, $OUR$ be our universe,
  $our$ be the region of space around us that has the same description.
  Note that $P(our) \ge P(OUR)=P(OUR|D)\cdot P(D) > 0$.
  Then $P(D|our) = \frac{P(our|D)\cdot P(D}{P(our)}) > 0$.
  Similarly,
  $P(ND|our) = \frac{P(our|ND)\cdot P(ND}{P(our)})
    = \frac{P(ND)}{N^k} \cdot \frac{1}{P(our)}$.
  To compare the two we have to compare
  $P(our|D)\cdot P(D)$ with $\frac{P(ND)}{N^k}$.
  }.

\section{Conclusion}

We have, then, the following reasonable possibilities:

\begin{enumerate}
\item Our universe is designed.
\item Our timespace is infinite and each infinite chunk has an infinite
      description. Any approximate description for a limited chunk of timespace
      around us is finite (infinite descriptions are impossible or
      incompatible with life). At the same time, one of the following
      would be true:
      \begin{enumerate}
      \item We are in a very unlikely chunk of our universe, where,
            given a reasonable decomposition in small pieces (e.g.
            hypercubes of size $1$), all pieces around us that we observe
            have the same approximate description. The probability
            of this happening converges quickly to $0$ when the number
            of observed pieces goes up.
      \item The universe pieces around us do not have the same
            approximate description. This means that
            we can't know much about the past and we can't
            predict much about the future. This is not what we
            observe.
      \end{enumerate}
\end{enumerate}

The most reasonable option seems to be that our universe is designed.

\svn{
    + explicatie ca universul nostru e unul din multele posibile bazate pe QM
    means that the meta-universe is too simple.
}

\section{Background}
\subsection{Probabilities}
\label{probabilities-section}

Let $A$ be a set with the same cardinality as $\reale$. Let $F$ be the set of
all mathematical predicates of one variable over $A$ that can be written as a
finite formula. If $f$ is such a predicate then let $A_f$ be the subset of
$A$ where $f$ is true, i.e. $A_f=\multime{a\in A\mid f(a) \mbox{ is true}}$.
Given a probability distribution over $A$, we define $P(f)=P(A_f)$.

Let $F_0$ be the set of all elements of $F$ such that their probability is
$0$ for all continuous probability distributions over $A$, i.e.
$$F_0=\multime{
  f\in F
  \mid P(A_f)=0
    \mbox{ for all continuous probability distributions over } A}.
$$
As an example, any predicate which is true for a finite subset of elements,
i.e. $A_f$ is finite, would belong to $F_0$. Let us identify by $A_0$ the
set of elements of $A$ for which at least one predicate of $F_0$ is true, i.e.
$$A_0=\multime{a \in A\mid \exists f\in F_0 \mbox{ with } f(a)\mbox{ true} }.$$
Let us also denote by $F_1$ and $A_1$ the complements of $F_0$ and $A_0$,
respectively.

Let $P$ be a continuous probability distribution.

Since $F$ is countable, $F_0$ must be at most countable. Then,
obviously, $P(F_0)=0$, so the probability of its complement, $P(F_1)$, is $1$.
Similarly, $P(A_0) = 0$ and $P(A_1) = 1$, which justifies the indexes used for
these.

Let us say that an element $a\in A_1$ is \definitie{generic}
and an element $a\in A_0$ is \definitie{peculiar}. Then we could rewrite
the equalities above to $P(x\mbox{ is generic}) = 1$ and
$P(x\mbox{ is peculiar}) = 0$.
\svn{I should actually use the \ghilimele{is finite} predicate or something
     like that.}

Obviously, $P(E)=0$, where $E\subset A$, does not mean that observing
an element of $E$ is logically impossible, it just means that,
if we make a set of (independent) observations that is at most countable
of elements of $A$, we have no chance at all of observing an element of $E$.

\subsection{Ordinals}
\label{sec:ordinals}

Ordinals are generalizations of natural numbers. Natural numbers can be
defined by identifying each natural number with the set of natural numbers less
than it. Then $0$ is identified with the empty set $\emptyset$,
$1$ with $\multime{0}=\multime{\emptyset}$,
$2$ with $\multime{0, 1}=\multime{\emptyset, \multime{\emptyset}}$,
and so on.

Each of these natural numbers is an ordinal. To define the smallest ordinal
that is not a natural number, denoted by $\omega$, we will use the
same rule: let $\omega$ to be the set of ordinals smaller than it, i.e
the set of natural numbers, $\omega=\multime{0, 1, 2, \dots}$.

Of course, the next ordinal, called $\omega + 1$, will be the set
$\multime{0, 1, 2, \dots \omega}$ and the next one,
$\omega+2$, will be $\multime{0, 1, 2, \dots \omega, \omega+1}$. We can
continue and, in the same way, define $\omega\cdot 2=\omega+\omega$ to be
the ordinal that comes after all $\omega+n$ where n is a natural number
\svn{Use finite ordinal instead of natural number.}.

Then we can define $\omega\cdot 3$, $\omega\cdot 4$ and so on, and we can
take $\omega\cdot \omega$ to be the ordinal that comes after all the ones
defined by using the above rules.

Let us note that $\omega$ is countable, and that all the ordinals mentioned
above that come after it are also countable. By using the same kind of
reasoning as above we can produce other countable ordinals like
$\omega^\omega$ (from ordinals like $\omega\cdot\omega\cdot\dots\cdot\omega$)
and $\epsilon_0$ (from $\omega^{\omega^{\cdots^\omega}}$).

After going through many similar processes, at some point we obtain the
smallest uncountable ordinal, $\omega_1$, which is the set of all
countable ordinals.

Let us note that some ordinals, like all the finite ones except $0$,
and like $\omega+1$, can be obtained from the previous one by using
a succesor relation, i.e. $succesor(\alpha) = \alpha\cup\multime{\alpha}$.
All ordinals have a succesor, but not all are succesors, some, like
$\omega$ and $\omega\cdot 2$ can be defined only as the set of all
smaller ordinals. The former are called \ghilimele{successor ordinals},
the later are called \ghilimele{limit ordinals}. Note that $0$ is a limit
ordinal.

Transfinite induction is a generalization of induction, where, if we can
prove that a property holds for an ordinal $\alpha$ from the fact that
it holds for all ordinals $\beta<\alpha$, then this property holds for
all ordinals.
\svn{am I using this?}

In many cases transfinite induction proofs are done separately for
succesor ordinals (maybe in the form $p(\alpha)$ implies $p(\alpha+1)$)
and for limit ordinals.
\svn{am I using this?}

\end{document}
